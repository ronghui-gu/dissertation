
\chapter{Related Work}
\label{chap-rel}

\paragraph{Hoare-style program verification} Hoare logic~\cite{hoare69}
and its modern variants~\cite{reynolds02,boogie05,nanevski06} were
introduced to prove strong (partial or total) correctness properties
of programs annotated with pre- and postconditions. A
total-correctness Hoare triple $[P]C[Q]$ often means a refinement
between the implementation $C$ and the specification $[P,Q]$: given
any state $S$, if the precondition $P(S)$ holds, then the command $C$
can run safely and terminate with a state that satisfies $Q$. Though
not often done, it is also possible to introduce auxiliary/ghost
states to serve as ``abstract states'' and prove that a program
implements a specification via a simulation.
  
Our layer language can be viewed as a novel way of imposing a module
system over Hoare-style program verification. We insist on using
interfaces with deep specifications and we address the ``conflicting
abstract states'' problem mentioned in
Section~\ref{sec:seq:overview}. Traditional program verification does not
always use deep specification (for pre- and post-conditions) so the
module interfaces (\eg, $[P,Q]$) may allow some safe but unwanted
behaviors. Such gap is fine if the goal is to just prove safety (as in
static type-checking), but if we want to prove the strong contextual
correctness property across module boundaries, it is important that
each interface accurately describes the functionality and scope of the
underlying implementation.

In addition to the obvious benefits on compositionality, our layered
approach also enables a new powerful way of combining programming- and
specification languages in a single setting. Each layer interface
enables a new programming language at a specific abstraction level,
which is then used to implement layers at even higher
levels. As we move up the layer hierarchy, our programming language
gets closer and closer to the specification language---it can call
primitives at higher abstraction levels but it still supports
general-purpose programming (\eg, in ClightX).

Interestingly, we did not need to introduce any program logic to
verify our OS kernel code. Instead, we verify it directly using the
ClightX (or LAsm) language semantics (which is already conveniently
parameterized over a layer interface).  In fact, unlike Hoare logic
which shows that a program (\eg, $C$) refines a specification (\eg,
$[P,Q]$), we instead show there is a downward simulation from the
specification to the program. As in CompCert, we found this easier to
prove and we can do this because both our specification and language
semantics are deterministic relative to external events.

\paragraph{Stepwise program refinement}
Dijkstra~\cite{Dijkstra72} proposed to ``realize'' a complex program
by decomposing it into a hierarchy of linearly ordered ``abstract
machines.''  Based on this idea, the PSOS team at SRI~\cite{psos80}
developed the Hierarchical Development Methodology (HDM) and applied
HDM to design and specify an OS using 20 hierarchically organized
modules. HDM was difficult to be rigorously applied in practice,
probably because of the lack of powerful specification and proof
tools. In this paper, we advance the HDM paradigm by using a new formal
layer language to connect multiple layers and by implementing all
certified layers and proofs in a modern proof assistant. We also
pursued decomposition more aggressively since it made our
verification task much easier.

Morgan's refinement calculus~\cite{morgan94} is a formalized approach
to Dijkstra's stepwise refinement. Using this calculus, a high-level
specification can be refined through a series of
correctness-preserving transformations and eventually turned into an
efficient executable. Our work imposes a new layer language to enhance
compositional reasoning. We use ClightX (or
LAsm) and the Coq logic as our ``refinement'' language,
and use a certified layer (with deep specification)
to represent each such correctness-preserving transformation.
All our ClightX and LAsm instances have executable semantics and can be
compiled and linked using our new CompCertX compiler.

\paragraph{Separate compilation for CompCert} Compositional 
compiler correctness is an extremely challenging
problem~\cite{BentonH09,hur12}, especially when it involves an open
compiler with multiple languages~\cite{perconti14}.  In the context of
CompCert, a recent proposal~\cite{beringer14}
aims to tackle the full Clight language but it has not been
fully implemented in the CompCert compiler.  While our CompCertX
compiler proves a stronger correctness theorem for each ClightX layer,
the ClightX language is subtly different from the original
full-featured Clight language. Within each ClightX layer, all locally
allocated memory blocks (\eg, stack frames) cannot be updated by
functions defined in another layer. This means that ClightX does not
support the same general ``stack-allocated data structures'' as in
Clight. This is fine for our OS kernels since they do not allocate any
data structures on stack, but it means that CompCertX can not be
regarded as a full featured separate compiler for CompCert.


As for the concurrency, \citet{stewart15} developed a new compositional extension of the
original CompCert compiler~\cite{compcert} with the goal of providing
thread-safe compilation of concurrent Clight programs.  Their
interaction semantics also treats all calls to synchronization
primitives as external calls. Their compiler does not support a layered
ClightX language as our CompCertX does, so it is unclear how it can
be used to build concurrent layers involving a stack of atomic objects.

%
% Previous work on object invariants have also considered how to specify
% invariants for layered objects. They rely on some auxiliary flags to
% specify when an invariant should hold and when it could fail. 
% 
% Some programming languages (e.g., Spec\#, JML, Eiffel) support rich
% language constructs for specifying contracts, pre- and post conditions
% and object invariants. 
%
% Another class of specification languages are verification intermediate
% languages like Boogie. They make the prescription of verification
% conditions natural and convenient. It contains both mathematical
% and imperative component, but the imperative component is only used
% for verification and it cannot be compiled to generate executable code. 
%
% Dafny is a programming language with classes and datatypes and
% specification constructs for describing intended behaviors. It is
% intended for proving functional correctness.
% 
% Hoare type theory describes a rich dependently typed language whose
% rich types can be used to describe Hoare assertions, but their
% implements relation is still a typechecking relation, not a 
% simulation relation. They do not describe 
%
% Formal specification languages such as TLA, Z, Alloy, and AADL, can
% build sophisticated models and reason about deep properties but they
% are not concerned with connecting to the actual running code.  
%
% Data abstraction is abstraction over shallow specification. The
% representation independence states that the behavior of client
% programs should only depend on the functionality of an abstract
% datatype, not the actual representation type.
%
\paragraph{OS kernel verification} There has been a large body
of recent work on OS kernel verification including
seL4~\cite{klein2009sel4,klein14}, Verve~\cite{hawblitzel10},
and Ironclad~\cite{ironclad14}. None of these works have addressed the
issues on concurrency with fine-grained locking. Very recently,
\citet{xu16} developed a new verification framework based on RGSim
and Feng~{et~al.}'s program logic~\cite{feng08:aim} for reasoning
about interrupts; they have successfully verified many key modules
(in C) in the $\mu$C/OS-II kernel, though so far, they have not proved
any progress properties.

\ignore{
\paragraph{OS kernel verification} The seL4
team~\cite{klein2009sel4} were the first to build a proof of
functional correctness for a realistic microkernel.  The seL4 work is
impressive in that all the proofs were done inside a modern mechanized
proof assistant. They have shown that the behaviors of 7500 lines of
their C code always follow an abstract specification of their
kernel. To make verification easier, they introduced an intermediate
executable specification to hide C specifics. Both their abstract and
executable specifications are ``monolithic'' as they are not divided
into layers to support abstraction among different kernel modules.
These kernel interdependencies led to more complex invariants which
may explain why their effort took 11 person years.

The initial seL4 effort was done completely at the C level so it does
not support many assembly level features such as address translation.
This also made verification of assembly code and kernel initialization
difficult (1200 lines of C and 500 lines of assembly are still
unverified). It is also unclear how to use their verified kernel
to reason about user-level programs since they would be running in a
different address space. Our certified kernels, on the other hand,
directly model assembly-level machines that support all kernel/user
and host/guest programs. Memory access to a user-level address space
must go through a page table, and memory access in a guest virtual
machine must go through a nested page table. We thus had no problem
verifying our kernel initialization or assembly code.


Recently, Sewell {\em et al.}~\cite{sewell13} used translation
validation to build a refinement proof between the
semantics of the seL4 binary (compiled by GCC) and the C source
code. This is an intriguing way of establishing the correctness of a
specific compilation run, as it did not rely on any certified
compiler, but it seems to be more suitable for a fixed code base. 
This might be suitable for seL4 since it is a minimal microkernel
not subject to much modification or extension.

This last point illustrates a key high-level difference between our
two projects. Our compositional approach is developed not just for
building a ``one-off'' lump of a verified microkernel, but instead, it
aims to support advanced development of all kinds of verified kernels.
In particular, we want to apply our layered framework to significantly
enhance the modularity and extensibility of OS kernels.  In place of
the rigid safety conditions used in traditional extension mechanisms
(e.g., that kernel extensions must be
type-safe~\cite{bershad95,hunt07}), our approach will assure both the
safety and semantic {\em correctness} of extensions with respect to
appropriate specifications.
}

\paragraph{Modular verification of low-level code} 
Vaynberg and Shao~\cite{vaynberg12} also used a layered approach to
verify a small virtual memory manager. Their layers are not linearly
ordered; instead, their seven abstract machines form a DAG with
potential upcalls (\ie, calls from a lower layer to upper ones). As a
result, their initialization function (an upcall) was much harder to
verify. Their refinement proofs between layers are insensitive to
termination, from which they can only prove partial correctness but
not the strong contextual correctness property which we prove in our
current work.

Feng {\em et al.}~\cite{feng08:vstte} developed OCAP, an open
framework for linking components verified in different
domain-specific program logics. They verified a
thread library with hardware interrupts and
preemption~\cite{feng08:aim} using a variant of concurrent
separation logic~\cite{ohearn:concur04}. They decomposed the thread
implementation into a sequential layer (with interrupts disabled)
and a concurrent layer (with interrupts enabled).
Chlipala~\cite{BedrockPLDI11} developed Bedrock, an automated
Coq library to support verified low-level programming. All these
systems
are purely sequential, 
and lack the support for concurrency.

\ignore{\paragraph{Certified abstraction layers} \citet{dscal15}
presented the first formal account of certified abstraction layers and
showed how to apply layer-based techniques to build certified system
software. The layer-based approach differs from Hoare-style program
verification~\cite{hoare69,reynolds02,boogie05,nanevski06} in several
significant ways. First, it uses the termination-sensitive forward
simulation techniques~\cite{Lynch95,compcert} and proves a stronger
contextual correctness property rather than simple partial or total
correctness properties (as done for Hoare logics). Second, the overlay
interface of a certified layer object completely removes the internal
concrete memory block (for the object) and replaces it with an
abstract state suitable for reasoning; this abstract state differs
from auxiliary or ghost states (in Hoare logics) because it is
actually used to define the semantics of the overlay abstract machine.
Third, as we move up the abstraction hierarchy by composing more
layers, each layer interface provides a new programming language that gets
closer to the specification language---it can call primitives at
higher abstraction levels while still supporting general-purpose
programming (e.g., in ClightX or LAsm).

Our new formal framework for certified concurrent layers follows the
same layer-based methodologies. Each time we introduce a new concrete
concurrent object implementation, we replace it with an abstract
atomic object in its overlay interface. All shared abstract states are
represented as a single global log, so the semantics of each atomic
method call would need to {\em replay} the entire global log to find
out the return value.  This seemingly ``inefficient'' way of treating
shared atomic objects is actually great for compositional
specification. Indeed, it allows us to apply game-semantic ideas and
define a general semantics that supports parallel layer composition.}

\paragraph{Abstraction for concurrent objects}
\citet{herlihy90} introduced {\em linearizability} as a key technique
for building abstraction over concurrent objects. Developing
concurrent software using a stack of shared atomic objects has since
become the best practices in the system
community~\cite{Herlihy08book,ospp11}. Linearizability is quite
difficult to reason about, and it is not until 20 years later that
\citet{filipovic10} showed that linearizability is actually equivalent
to a termination-insensitive version of the contextual refinement
property. \citet{Gotsman12concur} showed that such equivalence also
holds for concurrent languages with ownership
transfers~\cite{ohearn:concur04}.  \citet{liang13,lili16} showed that
linearizability plus various progress properties~\cite{Herlihy08book}
for concurrent objects are equivalent to various termination-sensitive
versions of the contextual refinement property. These results
convinced us that we should follow \citet{dscal15} and prove
termination-sensitive (contextual) simulation when building certified
concurrent layers as well.

\paragraph{RGSim} Building contextual refinement proofs
for concurrent programs (and program transformations) is challenging.
Liang~{et~al.}~\cite{RGSim,Liang14lics,lili16} developed the
Rely-Guarantee-based Simulation (RGSim) that can support both parallel
composition and also prove contextual refinement of concurrent
objects. Our contextual simulation proofs between two concurrent
layers can be viewed as an instance of RGSim if we extend RGSim with
auxiliary states such as environment contexts and shared logs. This
extension, of course, is the main innovation of our new compositional
layered model. Also, all existing RGSim systems are limited to reasoning
about atomic objects at one layer; their client program context cannot
be the method body of another concurrent object, so they cannot
support the same general vertical layer composition as our work does.

%%%% The following paragraph might be optional %%%%
\citet{lili16} also developed a program logic called LiLi that can
directly prove both the linearizability and starvation-freedom (or
deadlock-freedom) properties. Their ``rely'' conditions are specified
over shared states only, so they cannot express temporal properties. To
prove progress, they have to introduce a separate temporal ``rely''
condition called {\em definite actions}.  This made it difficult to
provide a standalone (total) specification for each lock acquire
method.  Indeed, all examples in their paper are code fragments that
must acquire a lock, then perform critical-section tasks, and then release the
lock. In contrast, our environment context can specify the full
strategies (\ie, both the past and the future behaviors) of all
environment threads and the scheduler, so we can readily impose
temporal invariants over the environment. Within each thread-modular
layer $L[t]$, we can show that each lock acquire primitive (\eg, for
ticket locks) always returns as long as its environment is cooperative
(\eg, always releases its acquired lock), even if $t$ itself may not
be cooperative.
In other words, the termination of $t$'s lock acquire
operation does not depend on whether $t$ itself will release the lock
after first acquiring it.


\paragraph{Treatment of parallel composition}
Most concurrent languages (including those used by RGSim) use a
parallel composition command $(C_1 \| C_2)$ to create and terminate
new threads.  In contrast, we provide thread spawn and join
primitives, and assign every new thread a unique $id$ (\eg, $t$, which
must be a member of the full thread-$id$  domain set $D$). Parallel layer
composition in our work is always done over the whole program $P$ over
all members of $D$. This allows us to reason about the current
thread's behaviors over the environment's full strategies (\ie, both
its past and future behaviors). Even if a thread $t$ is never
created, the semantics for running $P$ over $L[t]$ is still well
defined since it will simply always query its environment context to
construct a global log.

\paragraph{Program logics for shared-memory concurrency}
A large body of new program
logics~\cite{ohearn:concur04,brookes:concur04,SAGL,vafeiadis:marriage,LRG,verifast,gotsman13,Turon13popl,Turon13icfp,nanevski13,nanevski14,sergey15,sergey15pldi,pinto14,iris15,pinto16}
have been developed to support modular verification of shared-memory
concurrent programs. Most of these follow Hoare-style logics so they
do not prove the same strong contextual simulation properties as RGSim
and our layered framework do. Very few of them (\eg,~\cite{pinto16})
can reason about progress properties. Nevertheless, many of these
logics support advanced language features such as high-order functions
and sophisticated non-blocking synchronization, both of which will be
useful for verifying specific concurrent objects within our layered
framework.  Our use of a shared global log and an environment context
resembles the recent work~\cite{sergey15} on using compositional
subjective auxiliary states to represent history traces; the main
difference is again that our environment context can talk about both
past and future behaviors but a history trace can only specify past
behaviors. 


\paragraph{Connection to game semantics} Even though we have used
the game-semantic concepts (\eg, strategies) to describe the
construction of our environment contexts and compositional semantics,
our concurrent machine is still defined using traditional small-step
semantics.  This is in contrast to several past
efforts~\cite{ghica08,nishimura13,rideau11,abramsky99} of modeling
concurrency in the game semantics community where they use games to
define the semantics of a complete language. Modeling higher-order
sequential features as games is great for proving full abstraction,
but it is still unclear how it would affect large-scale program
verification as done in the certified software community.  We do
believe that there are great synergies between the two communities
waiting to be explored and we hope that our current work can serve as
a basis for promoting such interaction.

