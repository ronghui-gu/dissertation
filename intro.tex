% Introduction section

\chapter{Introduction}
\label{chap-intro}

\epigraph{{\em A distributed system is one in which the failure of a 
computer you didn't even know existed can render 
your own computer unusable.}{\\ \hfill --- Leslie Lamport}}

Modern hardware and software systems are constructed using a series of
abstraction layers (e.g., circuits, microarchitecture, ISA
architecture, device drivers, OS kernels, hypervisors, network
protocols, web servers, and application APIs), each defining
an interface that hides the implementation details of a particular set
of functionality.  Client programs built on top of each layer can be
understood solely based on the interface, independent of the layer
implementation. Two layer implementations of the same interface should
behave in the same way in the context of any client code.

The power of abstraction layers lies in their use of a very rich class
of specifications, which we will call {\em deep specifications} in
this paper. A deep specification, in theory, is supposed to capture
the precise functionality of the underlying implementation as well as
the assumptions which the implementation might have about its client
contexts. In practice, abstraction layers are almost never formally
specified or verified; their interfaces are often only documented in
natural languages, and thus cannot be rigorously checked or
enforced. Nevertheless, even such informal instances of abstraction
over deep specifications have already brought us huge
benefits. Baldwin and Clark~\cite{baldwin00} attributed such use of
abstraction, modularity, and layering as the key factor that drove the computer
industry toward today's explosive levels of innovation and growth
because {\em complex products can be built from smaller subsystems
  that can be designed independently yet function together as a
  whole}.

Abstraction and modularity have also been heavily studied
in the programming language community~\cite{reynolds98,pierce02}.
The focus there is on abstraction over ``shallow''
specifications. A module interface in existing languages cannot
describe the full functionality of its underlying implementation,
instead, it only describes type specifications, augmented sometimes
with simple invariants.  Abstraction over shallow
specifications is highly desirable~\cite{mitchell86}, but
client programs cannot be understood from the interface alone---this 
makes modular verification of correctness properties
impossible: verification of client programs must look beyond the
interface and examine its underlying implementation, thus breaking the
modularity.

Given the obvious importance, formalizing and verifying abstraction 
layers are highly desirable, but they pose many challenges:
\begin{itemize} \itemsep 0pt
\item
{\em Lack of a language-based model.}  It is unclear how to
  model abstraction layers in a language-based setting and how they
  differ from regular software modules or components.  Each layer
  seems to be defining a new ``abstract machine;'' it may take an
  existing set of mechanisms (e.g., states and functions) at the layer
  below and expose a different view of the same mechanisms. For
  example, a virtual memory management layer---built on top of a
  physical memory layer--- would expose to clients a different view of
  the memory, now accessed through virtual addresses.
%%%%%%%%%%%%%
\item {\em Lack of good language support.} Programming an abstraction
  layer formally, by its very nature, would require two languages: one
  for writing the layer implementation (which, given the low-level
  nature of many layers, often means a language like C or assembly);
  another for writing the formal layer specification (which, given the
  need to precisely specify full functionality, often means a rich
  formal logic). It is unclear how to fit these two different
  languages into a single setting. Indeed, many existing formal
  specification languages~\cite{znot92,lamport94,jackson12} are
  capable of building accurate {\em models} with rich specifications,
  but they are not concerned with connecting to the actual running code.
%%%%%%%%%%%%%
\item {\em Lack of compiler and linking support.} Abstraction layers
  are often deployed in binary or assembly. Even if we can verify a
  layer implementation written in C, it is unclear how to compile it
  into assembly and link it with other assembly layers. The CompCert
  verified compiler~\cite{compcert} can only prove the correctness of
  compilation for whole programs, not individual modules or
  layers. Linking C with assembly adds a new challenge since they may
  have different memory layouts and calling conventions.
\end{itemize}

In this paper, we present a formal study of abstraction layers that
tackles all these challenges. We define a {\em certified abstraction
  layer} as a triple $\layer{L_1}{M}{L_2}$ plus a mechanized proof
object showing that the layer implementation $M$, built on top of the
interface $L_1$ (the {\em underlay}), indeed faithfully {\em
  implements} the desirable interface $L_2$ above (the {\em overlay}).
Here, the {\em implements} relation is often defined as some
{\em simulation} relation~\cite{Lynch95}.  A certified layer can be
viewed as a ``parameterized module'' (from interfaces $L_1$ to $L_2$),
{\em a~la} an SML functor~\cite{milner97}; but it enforces a stronger
contextual correctness property: a correct layer is like a
``certified compiler,'' capable of converting any {\em safe} client program
running on top of $L_2$ into one that has the same behavior but runs
on top of $L_1$ (e.g., by ``compiling'' abstract primitives in $L_2$ into
their implementation in $M$). 

% A regular module $M$, which builds on top of $L_1$ and
% implements $L_2$, may not enjoy such a property. A client program $P$
% may invoke functions defined in $M$ and in another module $M'$.  Such
% $M'$ may share some state with $M$ but imposes state invariants that
% are in conflict with those assumed by $L_2$.  An abstraction layer
% does not allow such a client $P$, instead, such $M'$ must be either
% built on top of $L_2$ (thus respecting the invariants in $L_2$), or
% below $L_2$ (in which case, $L_2$ itself must be changed).

A regular software module $M$ (built on top of $L_1$) with interface
$L_2$ may not enjoy such a property because its client may invoke
another module $M'$ which shares some states with $M$ but imposes
different state invariants from those assumed by $L_2$. An abstraction
layer does not allow such a client, instead, such $M'$ must be either built
on top of $L_2$ (thus respecting the invariants in $L_2$), or below
$L_2$ (in which case, $L_2$ itself must be changed).

%
% Each layer interface $L$ contains a {\em deep} specification of a set
% of functionality in that given any client program $P$ (sanctioned by
% $L$), its execution on top of $L$ always yield deterministic behaviors
% (relative to external events, e.g., I/O, scheduling~\cite{leroy06,sevcik13})
%

Our paper makes the following new contributions:
\begin{enumerate}
\item We present the first language-based account of certified
  abstraction layers and show how they correspond to a {\em rigorous} 
  form of abstraction over deep specifications used widely
  in the system community. A certified layer interface
  describes not only the precise functionality of any underlying
  implementation but also clear assumptions about its client contexts.
  Abstraction over deep specifications leads to the powerful
  {\em implementation independence} property
  (see Sec.~\ref{sec:overview}): 
  any two implementations
  of the same layer interface have contextually equivalent behaviors.
%%%%%%%%%%%%%
\item We present a new layer calculus showing how to formally specify,
  program, verify, and compose certified abstraction layers 
  (see Sec.~\ref{sec:layer}). Such a
  layer language plays a similar role as the module language in
  SML~\cite{milner97}, but its interface checking is not just
  typechecking or signature matching; instead, it requires formal
  verification of the {\em implements} relation in a proof assistant.
%%%%%%%%%%%%%
\item We have instantiated the layer calculus on top of two core
  languages (see Sec.~\ref{sec:prog} and ~\ref{sec:lasm}): {\bf
    ClightX}, a variant of the CompCert Clight
  language~\cite{blazy-leroy-clight}; and {\bf LAsm}, an x86 assembly
  language.  Both ClightX and LAsm can be used to program certified
  abstraction layers.  We use the Coq logic~\cite{coq} to develop all
  the layer interfaces.  Each ClightX or LAsm layer is parameterized
  over its underlay interface, implemented using CompCert's external
  call mechanisms.  We developed new tools and tactic libraries to
  help automate the verification of the {\em implements} relation.
%%%%%%%%%%%%%
\item We have also modified CompCert to build a new verified compiler,
  {\bf CompCertX}, that can compile ClightX abstraction layers into
  LAsm layers (see Sec.~\ref{sec:comp}). 
  CompCertX is novel because it can prove a stronger
  correctness theorem for compiling individual functions in each
  layer---such a theorem requires reasoning about {\em memory
    injection}~\cite{leroy08} between the memory states of the source
  and target languages.  To support linking between ClightX and LAsm
  layers, we show how to design the {\em implements} relation so that it
  is stable over memory injection.
%%%%%%%%%%%%%
\item Using these new languages and tools, we have successfully
  constructed several feature-rich certified OS kernels in Coq (see
  Sec.~\ref{sec:kernel}).  A certified kernel
  $\layer{L_{\raw}}{K}{L_{\sys}}$ is a verified LAsm implementation
  $K$, built on top of $L_{\raw}$, and it {\em implements} the set of
  system calls as specified in $L_{\sys}$.  The correctness of the
  kernel guarantees that if a user program $P$ runs {\em safely} on
  top of $L_{\sys}$, running the version of $P$ linked with the kernel
  $K$ on $L_{\raw}$ will produce the same behavior.  All our certified
  kernels are built by composing a collection of smaller layers.
  The most realistic kernel consists of 37 layers, took less than one
  person year to develop, and can boot a version of Linux as a guest.
\end{enumerate}

\noindent{}The {\em POPL Artifact Evaluation Committee} reviewed the
full artifact of our entire effort, including ClightX and LAsm,
the CompCertX compiler, and the implementation of all
certified kernels with Coq proofs. 
The reviewers unanimously stated
that our implementation {\em exceeded their expectations}, with one
reviewer rating the paper as one of the most solid papers he/she has
ever seen. Additional details about our work can be found in
the companion technical report~\cite{dscal14tr}.
%
%\begin{verbatim}
%   flint.cs.yale.edu/publications/aec14.html
%\end{verbatim}
%

\paragraph{Roadmap.}
This thesis is organized as follows.
$\S$\ref{sec:overview} explains why layers support 
abstraction over deep specifications better than regular modules do.
$\S$\ref{sec:layer} presents the formal layer calculus.
$\S$\ref{sec:prog} and $\S$\ref{sec:lasm} describe the new ClightX and
LAsm languages and show how to use them to build certified layers.
$\S$~\ref{sec:comp} presents the new CompCertX compiler and shows how
to link ClightX layers with LAsm layers.  $\S$\ref{sec:kernel} gives
an overview of the certified OS kernels we have constructed.
$\S$\ref{sec:related} and $\S$\ref{sec:concl} discuss related work and
then conclude.



Abstraction layers (e.g., circuits, microarchitecture, ISA, device
drivers, OS kernels, hypervisors, network protocols, and application
APIs)~\cite{salzer09,baldwin00} are widely used in modern computer
systems to help reduce the complex interdependencies among components
at different levels of abstraction.  An abstraction layer defines an
interface that hides the implementation details of its underlying
software or hardware components. Client programs built on top of each
layer can be understood solely based on the interface, independent of
the layer implementation.

As multicore hardware and multithreaded programming become more
pervasive, many of these abstraction layers also become {\em
  concurrent} in nature.  Their interfaces not only hide the concrete
data representations and algorithmic details, but also create an
illusion of {\em atomicity} for all of their methods: each method call
can be viewed as if it completes in a single step, even though its
implementation contains complex interleavings with operations done by
other threads.  Herlihy~{\em{}et~al.}~\cite{herlihy90,Herlihy08book}
advocated using these concurrent atomic objects as the key building
blocks for constructing large-scale shared-memory concurrent software
systems.

\citet{dscal15} showed that specifying and verifying large-scale
complex system software (e.g., OS kernels and hypervisors) can also
benefit greatly from aggressive layer decomposition. They developed a
formal theory of certified (sequential) abstraction layers, as well as a
set of new certified programming tools.  A {\em certified abstraction
  layer} is defined as a triple $\layer{L'}{M}{L}$ plus a mechanized
proof object for the predicate $\ltyp{L'}{R}{M}{L}$, showing that the layer
implementation $M$, built on top of the interface $L'$ (the {\em
  underlay}), indeed faithfully {\em implements} the desirable
interface $L$ above (the {\em overlay}).  The {\em implements}
relation $R$ is defined as a {\em simulation}~\cite{Lynch95} but it
enforces a stronger {\em contextual} correctness property: a correct
layer is like a ``certified compiler,'' capable of converting any {\em
  safe} client program $P$ running on top of $L$ into one that has the
same behavior but runs on top of $L'$ (e.g., by ``compiling'' abstract
primitives in $L$ into their implementation in $M$); if
$\sem{L}{\cdot}$ denotes the state machine for layer $L$, the
correctness property (over context $P$) can be written formally as
$\sem{L'}{P\oplus{}M} \leq_R \sem{L}{P}$.

Unfortunately, the theory developed by \citet{dscal15} does not easily
carry over to the concurrent setting. Concurrent abstraction layers
pose many new technical challenges:
\begin{itemize} \itemsep 0pt
\item 
  {\em Lack of a general compositional model.} It is unclear how to
  define a general thread-modular semantic framework so that the
  behavior of each concurrent abstraction machine (e.g.,
  $\sem{L}{\cdot}$ for a layer interface $L$) can be constructed from the
  behaviors of each of its threads or CPU cores. More specifically, if
  $D$ denotes the set of all possible threads and CPU cores, $A$ is a
  subset of $D$, and $L[A]$ denotes the layer $L$ with members of $A$
  being {\em active} (i.e., members outside $A$ are considered as the
  {\em environment}), can we construct the semantics of the abstract
  machine $\sem{L[A]}{\cdot}$ from those of $\sem{L[t]}{\cdot}$
  for all $t\in{}A$? Note for readability, we have abbreviated
  $\{t\}$ as $t$, so $L[t]$ really denotes $L[\{t\}]$.
%%%%%%%%%%%%%
\item {\em Lack of a general parallel layer composition rule.}
  \citet{dscal15} supports both horizontal and vertical layer
  composition, but even with a compositional concurrency model, and
  supposing we can perform thread-modular reasoning to prove the
  ``contextual'' simulation property $\ltyp{L'[t]}{R}{M}{L[t]}$ for
  each $t\in{}D$, we still need a general ``parallel layer composition''
  rule to show that the contextual simulation holds for the whole
  concurrent machine $\ltyp{L'[D]}{R}{M}{L[D]}$.
  Liang~{\em{}et~al.}\cite{RGSim,Liang14lics,lili16} developed a
  Rely-Guarantee-based Simulation (RGSim) that is compositional over
  simple parallel composition constructs (e.g., $C_1 \| C_2$), but
  it is unclear how to extend it to support richer language constructs
  and vertical composition of multiple concurrent layers.
%%%%%%%%%%%%%
\item {\em Lack of compiler and linking support.} \citet{dscal15}
  developed a new CompCertX compiler that can compile certified
  ClightX layers into certified assembly layers. However, neither
  CompCertX nor the original CompCert compiler~\cite{leroy08} supports
  concurrent programs. It is unclear how to turn CompCertX into a
  thread-safe verified compiler, and how to use it to compile a
  certified concurrent C program into assembly and then link it with
  other concurrent assembly layers.
\end{itemize}

In this paper, we present a new formal framework that tackles all of
the above challenges for specifying, programming, compiling, and
composing certified concurrent abstraction layers. We extend each
layer interface $L$ in \citet{dscal15} with a collection of abstract
{\em atomic} objects~\cite{Herlihy08book}, each of which provides a
set of atomic methods. Unlike calls to thread-private abstract objects
(as in \cite{dscal15}) which are not {\em observable} by other
threads, each method call to an atomic object is recorded as an 
observable event in the concurrent machine semantics
($\sem{L}{\cdot}$), and is appended to the end of a global log $l$---a
single shared abstract state which we maintain for all the atomic
objects in $L$. The semantics $\sem{L}{P}$ can then be defined based
on the set of shared logs (i.e., {\em{}traces}, possibly of
infinite length) generated from running the concurrent program $P$
over $L$.

We develop a general compositional (operational) model for
$\sem{L}{P}$ based upon ideas from the game semantics
community~\cite{gsinvite}. Each run of $P$ over $L$ can be viewed as
playing a game involving members of $D$ (plus a scheduler). Each
participant $t$ contributes its play by appending an event into the
global log; its {\em strategy} $\strat{t}$ can be defined as a
deterministic partial function from the current log $l$ to its next
move $\strat{t}(l)$ when the control is transferred to $t$
(defined by the last event in $l$). The
eventual log for executing $P$ over $L$ can then be calculated by
combining the strategies of all the participants.

A layer interface $L[A]$ can be viewed as a concurrent machine
parameterized over its {\em environment context} $\oracle$, which is a
strategy for its environment (i.e., the scheduler plus those
participants not in $A$).  Given an environment context $\oracle$
which also contains a particular scheduler strategy, the execution of
$P$ over $L[A]$ should be deterministic; the concurrent machine will
run $P$ when control is transferred to any of the participants in $A$, but
will instead ask the environment context $\oracle$ for the next move when 
control is transferred to a member of the environment.

The new compositional model allows us to prove a general parallel
layer composition rule. Furthermore, $L[t]$ behaves like a
thread-local ``sequential'' machine (albeit parameterized over an
environment context).  This makes it possible to adapt the CompCertX
compiler so that it can be used to compile a certified ClightX layer
$(L'[t],M,L[t])$ into a certified assembly layer. We can then apply
both the horizontal and vertical layer composition rules (as developed
by \citet{dscal15}) along with the new parallel composition rule to 
construct fully certified concurrent software.
 
Our paper makes the following new contributions:
\begin{itemize} \itemsep 0pt
%%%%%%%%%%%%%
\item We present the first language-based account of certified
  concurrent abstraction layers and show that complex concurrent
  software can indeed be decomposed into layers of simple concurrent
  atomic objects while still supporting thread-modular reasoning.
  Following \citet{dscal15}, for each certified layer, we prove a {\em
    termination-sensitive} contextual correctness property. In the
  concurrent setting, this means that every certified concurrent
  object satisfies not only a safety property (e.g., {\em
    linearizability})~\cite{herlihy90,filipovic10} but also a
  progress property (e.g., {\em
    starvation-freedom})~\cite{liang13,lili16}.
%%%%%%%%%%%%%
\item We present a new compositional semantic model for shared-memory
  concurrent abstract machines and prove a general parallel layer
  composition rule. Our new framework is general in that it can be
  used to specify and verify the behaviors of various concurrency
  constructs at different levels of abstraction, such as a multicore
  machine with hardware scheduler, implementation of fine-grained
  synchronization primitives (e.g., spinlocks, queuing locks,
  condition variables), low-level thread primitives (e.g., $\yield$,
  $\sleep$, $\wakeup$), and multi-threading with software scheduler and I/O
  interrupts.
%%%%%%%%%%%%%
\item We show how to apply standard {\em simulation} techniques
  (as in \cite{compcert,dscal15}) to verify
  both the contextual correctness and liveness properties in a unified setting.
  Following RGSim~\cite{RGSim}, we can impose
  {\em invariants} both over the environment contexts (i.e., the ``rely'')
  and also over the active threads themselves (i.e., the ``guarantee'').
  However, unlike RGSim,
  because each environment context (i.e., the opponent's strategy)
  specifies not just the environment's past behaviors
  but also {\em future behaviors}, we can readily impose temporal
  invariants such as fairness requirements (for schedulers) or
  {\em definite actions}~\cite{lili16} (for releasing locks). This allows
  us to give full specifications for lock primitives and support vertical
  composition of starvation-free atomic objects, none of which have ever
  been done before~\cite{lili16}.
%%%%%%%%%%%%%
\item We have successfully extended the layered programming languages
  ClightX (a C variant) and LAsm (an x86 assembly language) by
  \citet{dscal15} with support for concurrent layered programming.  We
  have also developed a new {\em thread-safe} version of the CompCertX
  compiler that can compile certified concurrent ClightX layers into
  LAsm layers.  To support certified parallel composition of
  thread-modular layers (i.e., $L[t]$ for all $t$), we have developed
  a new extended algebraic memory model (for CompCertX) whereby stack
  frames allocated for each thread can be combined to form a single
  coherent CompCert-style memory.
%%%%%%%%%%%%%
\item Using these new languages and tools, we have successfully
  constructed a certified concurrent OS kernel in the Coq proof
  assistant~\cite{coq}. Our certified kernel supports not just user-
  and I/O concurrency on a single CPU, but also multicore parallelism
  with fine-grained locking. We prove both the contextual
  correctness of all system calls, as well as a liveness property 
  saying that all system calls will eventually return.  As far as we 
  know, this is the first proof of functional correctness of a complete,
  general-purpose concurrent OS kernel with fine-grained locking.
\end{itemize}

The full artifact for our entire effort, including the new ClightX and
LAsm languages for concurrent layered programming, the thread-safe
CompCertX compiler, and the implementation of our certified concurrent
OS kernel have been uploaded as supplementary materials on the POPL
2017 submission site.

The rest of this paper is organized as follows.
Sec.~\ref{sec:informal} gives an overview of a few key high-level
ideas for building certified concurrent layers.  Sec.~\ref{sec:layer}
presents the formal framework for specifying, verifying, and composing
concurrent layers.  Sec.~\ref{sec:mach}-\ref{sec:prog}
describe our layered concurrent machine models and  
show how to use them to build certified
concurrent layers.  Sec.~\ref{sec:comp} presents the new thread-safe
CompCertX compiler and shows how to link ClightX layers
with concurrent LAsm layers.  Sec.~\ref{sec:kernel} gives an overview
of the certified concurrent OS kernel that we have constructed.
Sec.~\ref{sec:related} discusses related work and then concludes.

Finally, note that throughout this paper, all of our concurrent 
abstract machines assume strong sequential consistency for atomic 
primitives. Supporting relaxed memory models is left as future work. 




