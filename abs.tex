% Abstract

\begin{abstract}

Modern computer systems consist of a multitude of abstraction layers
(\eg, OS kernels, hypervisors, device drivers, network protocols),
each of which defines an interface that hides the implementation
details of a particular set of functionality. Client programs built on
top of each layer can be understood solely based on the interface,
independent of the layer implementation. Despite their obvious
importance, abstraction layers have mostly been treated as a system
concept; they have almost never been formally specified or
verified. This makes it difficult to establish strong correctness
properties, and to scale program verification across multiple layers.

In this paper, we present a novel language-based account of
abstraction layers and show that they correspond to a strong form of
abstraction over a particularly rich class of specifications which we
call {\em deep specifications}. Just as {\em data abstraction} in
typed functional languages leads to the important {\em representation
independence} property, abstraction over deep specification is
characterized by an important {\em implementation independence}
property: any two implementations of the same deep specification must
have {\em contextually equivalent} behaviors.  We present a new layer
calculus showing how to formally specify, program, verify, and compose
abstraction layers. We show how to instantiate the layer calculus in
realistic programming languages such as C and assembly, and how to
adapt the CompCert verified compiler to compile certified C layers
such that they can be linked with assembly layers. Using these new
languages and tools, we have successfully developed multiple certified
OS kernels in the Coq proof assistant, the most realistic of which
consists of 37 abstraction layers, took less than one person year to
develop, and can boot a version of Linux as a guest.

\end{abstract}
