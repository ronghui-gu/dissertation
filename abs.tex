% Abstract

\begin{abstract}

Operating System (OS) kernels form the backbone of all system
software. They can have a significant  impact on the resilience,
extensibility, and security of today's computing hosts. 
However, modern OS kernels are complex and consist of a multitude of 
abstraction layers with concurrency; 
unfortunately,
concurrent abstraction layers have almost never been formally specified or
verified. This makes it difficult to establish strong correctness
properties, and to scale program verification across multiple layers.

Recent efforts
have demonstrated the feasibility of building large scale
formal proofs of functional correctness for simple general-purpose
kernels, but the cost of such verification is still prohibitive,
and it is unclear how to use their verified kernels to reason about
user-level programs and other kernel extensions.
Furthermore, they
have ignored the issues of concurrency, which include not
just user- and I/O concurrency on a single core, but also
multicore parallelism with fine-grained locking.

This thesis presents \CTOS{}, an extensible architecture
for building certified sequential and concurrent OS kernels.
\CTOS{} proposes a novel language-based account of concurrent
abstraction layers and show that they correspond to a strong form of
abstraction over a particularly rich class of specifications, which are
called {\em deep specifications}.
To formally specify, program, verify, and compose
concurrent abstraction layers,
\CTOS{} instantiates
 the certified abstraction layer framework
in
realistic programming languages such as C and assembly, and
adapts the CompCert verified compiler to compile certified C layers
such that they can be linked with assembly layers. 

To evaluate \CTOS{}, we  have successfully developed multiple
practical sequential and concurrent OS kernels.
The most realistic sequential hypervisor kernel is written in 6000 lines of C and x86
assembly, and can boot a version of Linux as a guest. 
The general-purpose concurrent OS kernel 
with fine-grained locking
can boot on a quad-core hardware.
For all the certified kernels,
their abstraction layers
and (contextual)
functional correctness properties are specified and verified in the Coq proof assistant. 

\ignore{
\paragraph{POPL15}
Modern computer systems consist of a multitude of abstraction layers
(\eg, OS kernels, hypervisors, device drivers, network protocols),
each of which defines an interface that hides the implementation
details of a particular set of functionality. Client programs built on
top of each layer can be understood solely based on the interface,
independent of the layer implementation. Despite their obvious
importance, abstraction layers have mostly been treated as a system
concept; they have almost never been formally specified or
verified. This makes it difficult to establish strong correctness
properties, and to scale program verification across multiple layers.

In this paper, we present a novel language-based account of
abstraction layers and show that they correspond to a strong form of
abstraction over a particularly rich class of specifications which we
call {\em deep specifications}. Just as {\em data abstraction} in
typed functional languages leads to the important {\em representation
independence} property, abstraction over deep specification is
characterized by an important {\em implementation independence}
property: any two implementations of the same deep specification must
have {\em contextually equivalent} behaviors.  We present a new layer
calculus showing how to formally specify, program, verify, and compose
abstraction layers. We show how to instantiate the layer calculus in
realistic programming languages such as C and assembly, and how to
adapt the CompCert verified compiler to compile certified C layers
such that they can be linked with assembly layers. Using these new
languages and tools, we have successfully developed multiple certified
OS kernels in the Coq proof assistant, the most realistic of which
consists of 37 abstraction layers, took less than one person year to
develop, and can boot a version of Linux as a guest.

\paragraph{POPL17}
Concurrent abstraction layers are ubiquitous in modern computer
systems because of the pervasiveness of multithreaded programming and
multicore hardware.  Abstraction layers are used to hide the
implementation details (e.g., fine-grained locking for building
concurrent atomic objects) and reduce the complex dependencies among
components at different levels of abstraction. Despite their obvious
importance, concurrent abstraction layers have almost never been
treated formally.  This severely limits the applicability of
layer-based techniques and makes it difficult to scale formal
verification across multiple concurrent layers.

In this paper, we present a formal language-based account of
concurrent abstraction layers and show that complex concurrent
software can indeed be decomposed into layers of simple concurrent
atomic objects while still supporting thread-modular reasoning. We
present a new layer framework showing how to formally specify,
program, verify, and compose concurrent layers. We show how to build
layered concurrent machine models and support layered concurrent
programming in C and assembly. We have built a thread-safe version of
CompCert that can compile certified C layers and link them with other
concurrent assembly layers.  Using these new languages and tools, we have
successfully developed a certified concurrent OS kernel in the Coq
proof assistant. As far as we know, this is the first proof of
functional correctness of a complete, general-purpose concurrent OS
kernel with fine-grained locking.

\paragraph{OSDI16}
Operating System (OS) kernels form the backbone of all system
software. They can have a significant impact on the resilience
and security of today's computers. Recent efforts
have demonstrated the feasibility of building formal proofs of
functional correctness for simple general-purpose kernels, but they
have ignored the important issues of concurrency, which include not
just user- and I/O concurrency on a single core, but also
multicore parallelism with fine-grained locking.  Many 
researchers believe that building a certified concurrent kernel is
very challenging, and even if it is possible, its cost would far
exceed that of verifying a single-core sequential kernel.

In this paper, we present a new compositional approach for building
certified concurrent OS kernels. Concurrency allows interleaved
execution of kernel/user modules across different layers of
abstraction. Each such layer can have a different set of observable
events. We insist on formally specifying these layers and their
observable events, and then verifying each kernel module at its proper
abstraction level. To support certified linking with other CPUs,
threads, or processes, we prove a strong contextual refinement
property for every kernel function, which states that the
implementation of each such function will behave like its
specification under any kernel/user context with any valid
interleaving. We have successfully developed a
practical concurrent OS kernel and verified its (contextual)
functional correctness property in the Coq proof assistant. Our
certified hypervisor kernel is written in 6000 lines of C and x86
assembly, can boot a version of Linux as a guest. To our knowledge,
this is the first proof of functional correctness of a complete,
general-purpose concurrent OS kernel with fine-grained locking.

\paragraph{SOSP15}
Operating System (OS) kernels form the backbone of all system
software. They can have the greatest impact on the resilience,
extensibility, and security of today's computing hosts. Recent effort
on seL4 has demonstrated the feasibility of building large scale
formal proofs of functional correctness for a general-purpose
microkernel, but the cost of such verification is still prohibitive,
and it is unclear how to use such a verified kernel to reason about
user-level programs and other kernel extensions.

In this paper, we present a new compositional approach for building
certified OS kernels.  Because the very purpose of an OS kernel is to
build layers of abstraction over hardware resources, we insist on
uncovering and specifying these layers formally, and then verifying
each kernel module at its proper abstraction level. To support
reasoning about user-level programs and linking with other certified
kernel extensions, we prove a strong contextual refinement property
for every kernel function, which states that the implementation of
each such function will behave like its specification under any
kernel/user (or host/guest) context. To demonstrate the effectiveness
of our new approach, we have successfully implemented and specified a
practical OS kernel and verified its (contextual) functional
correctness property in the Coq proof assistant. We show how to extend
our base kernel with new features such as virtualization and ring-0
processes and how to quickly adapt existing verified layers to build
new certified kernels for different domains. Our certified hypervisor
OS kernel is written in 5500 lines of C and x86 assembly, and can
successfully boot a version of Linux as a guest. The entire
specification and proof effort took less than 1.5 person years.
}

\end{abstract}
