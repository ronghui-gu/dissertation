\chapter{Conclusion}
\label{chap-concl}

Abstraction layers are key techniques used in building large-scale
computer software and hardware. In this paper, we have presented a
novel language-based account of abstraction layers and shown that they
are particularly suitable for supporting abstraction over deep
specifications, which is essential for compositional verification of
strong correctness properties. We have designed a new layer language
and imposed it on two different core languages (ClightX and LAsm). We
have also built a verified compiler from ClightX to LAsm. By
aggressively decomposing each complex abstraction into smaller
abstraction steps, we have successfully developed several certified OS
kernels that prove deeper properties (contextual correctness), contain
smaller trusted computing bases (all code verified at the assembly
level), require significantly less effort (3000 lines of C and
assembly code proved in less than 1 person year), and demonstrate
strong support for extensibility (layers are heavily reused in
different certified kernels). We expect that both deep specifications
and certified abstraction layers will become critical technologies and
important building blocks for developing large-scale certified system
infrastructures in the future.



Abstraction layers are key techniques used in building large-scale
concurrent software and hardware. In this paper, we have presented a
novel language-based account of certified concurrent layers and showed
how to build compositional semantic models for our layered abstract
machines. We have also developed new languages and tools for
supporting layered concurrent programming and thread-safe verified
compilation.  By aggressively decomposing complex concurrency features
into a stack of simple atomic objects, we have successfully built a
certified concurrent OS kernel that supports both fine-grained locking
on multicore machines and blocking synchronization primitives using
thread $\sleep$ and $\wakeup$ primitives. We expect that our new concurrent
layer-based techniques will become critical technologies and important
building blocks for developing large-scale certified system
infrastructures in the future.