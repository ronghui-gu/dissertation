\section{Compositional Concurrent Abstract Machine}
\label{sec:layer}
The sequential abstraction layer
is understood 
in terms of languages and program transformations,
and this is the starting point
of the concurrent abstraction layer.
We use the following kind of abstract machine.

\begin{definition}[Machine]
\label{def:mach}
A \emph{machine} is a tuple $\Mach = (S, \rightarrow, I, F)$
where $S$ is a set of states,
$\rightarrow \, \subseteq S \times S$ is a transition relation,
$I \subseteq S$ is a set of initial states, and
$F \subseteq S$ is a set of final states.
\end{definition}

\ignore{
\begin{definition}[Safety]
Given a machine $\Mach$,
we say that $s \in S$ is a \emph{stuck state}
if $s \notin F$ and there is no $s'$ such that $s \rightarrow s'$.
Then, a \emph{safe state} is a state $s$ such that there is no
stuck state $s'$ with $s \rightarrow^* s'$.
%A \emph{safe machine} is a machine which has at least one initial state,
%and for which all initial states are safe.
\end{definition}
}

\begin{definition}[Refinement]
We say that {\small$\Mach_1$} refines {\small $\Mach_2$}
and write {\small $\Mach_1 \refines \Mach_2$}
if there is a simulation relation {\small $R \subseteq S_1 \times S_2$}
such that:
\begin{itemize}
\item for any $s_1 \in I_1$,
	%if $I_2$ is non-empty then
	there exists an $s_2 \in I_2$
	such that $s_1 \ R\ s_2$;
\item for any related states $s_1\ R\ s_2$, % with $s_2$ safe,
	if $s_1 \in F_1$ then $s_2 \in F_2$ as well;
\item for any related states $s_1\ R\ s_2$, % with $s_2$ safe,
	if there is a step $s_1 \rightarrow s_1'$ in $\Mach_1$,
	then there is a series of steps $s_2 \rightarrow^+ s_2'$ in $\Mach_2$
	such that $s_1'\ R\ s_2'$.
\end{itemize}
\end{definition}

Based on these definitions,
we can make precise our notion of languages and
certified program transformations.

\begin{definition} \label{def:certified-trans}
A \emph{language} is a set of programs $\mathbb{P}$
together with a semantics operator $\llbracket - \rrbracket$,
which gives a machine $\llbracket P \rrbracket$ for each $P \in \mathbb{P}$.

A \emph{certified transformation} between languages $\mathbb{P}_1$ and $\mathbb{P}_2$
is a function $f : \mathbb{P}_1 \rightarrow \mathbb{P}_2$ such that
$\forall P \in \mathbb{P}_1 \,.\,
\llbracket f(P) \rrbracket \refines \llbracket P \rrbracket$
\end{definition}

\subsection{Certified concurrent abstraction layers}
The setting of languages and certified program transformations
offers limited compositionality:
two transformations can be vertically composed,
corresponding to stacking abstraction layers
or sequencing the passes of a certified compiler.
To make modular verification and linking more practical, Gu et
al. \cite{dscal15} introduce the notion of \emph{layer interfaces} $L$
specifying abstract states and associated primitives,
and \emph{certified modules} $L_1 \vdash_R M : L_2$,
where a piece of code $M$
implements an \emph{overlay interface} $L_2$ on top of an
\emph{underlay interface} $L_1$ through a refinement relation $R$.
This setting supports both \emph{horizontal} and \emph{vertical} composition rules,
which can be used to decompose the implementation of a layer:

\vspace{-5pt}
{\small
\begin{mathpar}
\inferrule{
  L \vdash_R M_{1} : L_1 \\ L \vdash_R M_{2} : L_2
}{
  L \vdash_{R} M_1 \oplus M_2 : L_1 \oplus L_2
}
\end{mathpar}
\vspace{-5pt}
\begin{mathpar}
\inferrule{
  L_1 \vdash_R M : L_2 \\ L_2 \vdash_S N : L_3
}{
  L \vdash_{R \circ S} M \oplus N : L_3
}
\end{mathpar}
\vspace{-5pt}
}%

Gu et al. instantiate this basic framework for
$\text{ClightX}(L)$, a modular formal semantics for the CompCert
Clight subset of C parameterized with layer interface $L$, and
$\text{LAsm}(L)$, a modular formal semantics for the CompCert x86
assembly language. For each language, they then define
the module system's judgement as:
\begin{definition} \label{def:popl15-layers}
$L_1 \vdash_R M : L_2$ is defined as a per-function simulation
$L_2 \leqslant_R \llbracket M \rrbracket L_1$. For
  each primitive $f$ defined in $L_2$, either a primitive $f$ in $L_1$
  is defined and simulates $f$ in $L_2$, or there exists a function
  $f$ in $M$, calling primitives in $L_1$, and whose behavior
  simulates that of the primitive $f$ in $L_2$.
\end{definition}
They then prove the horizontal compositionality
property
and prove that their compositional
verification framework is sound with respect to \emph{contextual
  refinement}:

\begin{theorem}[Contextual refinement]
Let $L_1 \vdash M : L_2$ be a certified abstraction layer. Then, the
transformation $P \mapsto (M \oplus P)$ is certified in the sense of
Definition~\ref{def:certified-trans}, i.e. for any program $P$:
\[ \mathcal{L}(M \oplus P, L_1) \refines \mathcal{L}(P, L_2) \]
\end{theorem}
\noindent where, for any program 
{\small $P$, $\mathcal{L}(P, L) = (S,
\stackrel{L}{\rightarrow}, I(P), F)$} is the machine describing the
whole-program (small-step) semantics of the program $P$ when external
function calls are interpreted as primitives in $L$.

\subsection{Partial machines}

We define our own setting of concurrent abstract machines
with support for parallel composition,
denoted as \emph{partial machine}.
\begin{definition}[Partial machine]
\label{def:partialm}
A \emph{partial machine} is a tuple $\Mach = (A, R, G, S, \Delta, I, F)$
where:
\begin{itemize}
\item
$A \subseteq T$ is the machine's \emph{active thread set},
\item
$R \subseteq \{ l \in \Ev^* \ |\ \kw{target}(l) \notin A \}$ is a ``rely'' log invariant,
\item
$G \subseteq \{ l \in \Ev^* \ |\ \kw{target}(l) \in A \}$ is a ``guarantee'' log invariant,
\item
$S$ is the set of \emph{local states} of $\Mach$,
\item
$\Delta : S \times G \rightarrow \mathcal{P}(S \times \Ev^*)$
specifies the \emph{local transitions},
\item
$I \subseteq S$ is the set of \emph{initial local states}, and
\item
$F \subseteq S$ is the set of \emph{final local states}.
\end{itemize}
$\Mach$ will execute with a log $l$ that always satisfies the invariant $R \cup G$.
The transition relation $\Delta$ must preserve it event by event:
\[ \Delta(s, l) \ni (s', \vec{e}) \Rightarrow (\forall \vec{e}_0 \,.\, l \cdot \vec{e}_0 \sqsubset \vec{e} \Rightarrow \vec{e}_0 \in G) \wedge (l \cdot \vec{e} \in R \cup G) \]
An \emph{environment context} $\Env : R \rightharpoonup \Ev$
of $\Mach$ can provide for any log $l$
which satisfies the ``rely'' invariant
a move which satisfies:
\[ l \in R \land \Env(l) = e \Rightarrow l \cdot e \in R \cup G \]
We write $\EC{\Mach}$ for the set of all environment contexts of $\Mach$.
\end{definition}

In order to provide an execution model for partial machines,
we show how a regular machine can be constructed
from a partial machine
by supplementing it with an environment context.

\begin{definition}[Instantiated partial machine]
Given a partial machine $\Mach$ and an environment context $\mathcal{E} \in \EC{\Mach}$,
we define the machine:
\[ \Mach \langle \Env \rangle =
	(S \times \Ev^*, \rightarrow_\Env, I \times \{\epsilon\}, F \times \Ev^*) \,, \]
where the step relation $\rightarrow_\Env$ is generated by:
\[
	\AxiomC{$\Delta(s, l) \ni (s', \vec{e})$}
	\UnaryInfC{$(s, l) \rightarrow_\Env (s', l \cdot \vec{e})$}
	\DisplayProof
	\quad
	\AxiomC{$\Env(l) = e$}
	\UnaryInfC{$(s, l) \rightarrow_\Env (s, l \cdot e)$}
	\DisplayProof
\]
We say that $\Mach$ takes a local step when the first rule is used,
and that $\Mach$ takes an environment step when the second rule is used.
\end{definition}

Because partial machines
running at different levels of abstraction
will be run alongside different kinds of environment contexts,
simulations of partial machines are stated relative to
a function $f$ which maps
environment contexts of one machine into
environment contexts of the other.
We define them as follows.

\begin{definition}[Refinement of partial machines]
Given two partial machines $\Mach_1$, $\Mach_2$ and
a function $f : \EC{\Mach_1} \rightarrow \EC{\Mach_2}$,
we say that $\Mach_1$ refines $\Mach_2$ when:
\[ \Mach_1 \refines_f \Mach_2
	\ \stackrel{\text{def.}}{\Leftrightarrow}\ 
	\forall \Env \in \EC{\Mach_1} \,.\,
		\Mach_1 \langle \Env \rangle \refines
		\Mach_2 \langle f(\Env) \rangle \]
\end{definition}

%\begin{definition}[Simulation of partial machines]
%We say that $M_1 \le_f M_2$ if there exists a relation
%$R \subseteq (S_1 \times \Ev^*) \times (S_2 \times \Ev^*)$
%between the (local and global) states of $M_1$ and $M_2$,
%as well as a function $f : (\Ev^* \rightarrow \Ev) \rightarrow (\Ev^* \rightarrow \Ev)$
%mapping
%environment contexts intended for $M_1$ into
%environment contexts intended for $M_2$,
%such that:
%\begin{itemize}
%\item for any local initial state $s_1 \in I_1$ there exists a local initial state $s_2 \in I_2$
%	such that $(s_1, \epsilon) R (s_2, \epsilon)$;
%\item for any related pair of states $(s_1, l_1) R (s_2, l_2)$,
%	if $s_1 \in F_1$ then $s_2 \in F_2$;
%\item for any related pair of states $(s_1, l_1) R (s_2, l_2)$,
%	if $M_1$ takes a local step $\Delta_1(s_1, l_1) \ni (s_1', \vec{e}_1)$
%	then there exists a state $(s_2', \vec{e}_2)$
%	such that $(s_1', l_1 \cdot \vec{e}_1) R (s_2', l_2 \cdot \vec{e}_2)$
%	and such that $M_2$ takes a local step $\Delta(s_2, l_2) \ni (s_2', \vec{e}_2)$;
%\item for any related pair of states $(s_1, l_1) R (s_2, l_2)$,
%	if $M_1$ takes an environment step
%	then $M_2$ takes an environment step as well,
%	and the resulting states
%	$(s_1, l_1 \cdot \mathcal{E}(l_1)) R (s_2, l_2 \cdot f(\mathcal{E})(l_2))$
%	are related.
%\end{itemize}
%\end{definition}
%
%\begin{lemma}[Instantiation preserves simulations]
%Basically, $M \le_f M' \Rightarrow M(\mathcal{E}) \le M'(\mathcal{E})$.
%\end{lemma}

Now we present two different ways in which
we can concretize the environment context of a partial machine:
(1) by providing complementary partial machines to replace environment transitions;
or (2) by replacing the environment context with nondeterministic choices.
Let us first consider the composition of partial machines.

\begin{definition}[Linked partial machine]
\label{def:link:partial}
Given two partial machines $\Mach_1$ and $\Mach_2$ with
$A_1 \cap A_2 = \varnothing$,
whose invariants satisfy:
\begin{gather*}
\{ l \in R_1 \ |\ \kw{target}(l) \in A_2 \} \subseteq G_2 \quad
\{ l \in R_2 \ |\ \kw{target}(l) \in A_1 \} \subseteq G_1 \\
\{ l \in R_1 \ |\ \kw{target}(l) \notin A_2 \} = \{ l \in R_2 \ |\ \kw{target}(l) \notin A_1 \}\,,
\end{gather*}
we define the composite partial machine:
\[ \Mach_1 \Join \Mach_2 =
	(A_1 \cup A_2,
	 R_1 \cap R_2,
	 G_1 \cup G_2,
	 S_1 \times S_2,
	 \Delta,
	 I_1 \times I_2,
	 F_1 \times F_2) \,. \]
The local transition relation is defined as:
\begin{align*}
\Delta(s_1, s_2, l) =
	&\ \{ (s_1', s_2, \vec{e}) \ |\ (s_1', \vec{e}) \in \Delta_1(s_1, l) \} \\ \cup
	&\ \{ (s_1, s_2', \vec{e}) \ |\ (s_2', \vec{e}) \in \Delta_2(s_2, l) \} \,,
\end{align*}
\end{definition}

The linking operator replaces some of the environment steps of one machine
with local steps of the other.
It preserves simulations in the following sense:

\begin{lemma}[Monotonicity of $\Join$]
\label{lemma:mono}
Given $\Mach_1 \refines_f \Mach_1'$ and $\Mach_2 \refines_f \Mach_2'$ we have
$\Mach_1 \Join \Mach_2 \refines_f \Mach_1' \Join \Mach_2'$.
%\begin{proof}
%TODO: prove.
%\end{proof}
\end{lemma}

The other way $\Env$ can be concretized
is to let it range over all environment contexts
and to build a nondeterministic machine.
This is useful when the environment
does not contain any concurrent threads,
but only the hardware scheduler.

\begin{definition}[Nondeterminized partial machine]
Given a partial machine $\Mach$,
we define the machine:
\[ \bigcup \Mach =
	(S \times \Ev^*, \textstyle\bigcup\nolimits_{\mathcal{E} \in \EC{\Mach}} \rightarrow_\mathcal{E},
		I \times \{ \epsilon \}, F \times \Ev^*) \,, \]
where $\rightarrow_\mathcal{E}$ is defined as before.
\end{definition}

\begin{lemma}[Monotonicity of $\bigcup \Mach$]
\[ \Mach_1 \refines_f \Mach_2 \Rightarrow \bigcup \Mach_1 \refines \bigcup \Mach_2 \]
%\begin{proof}
%\end{proof}
\end{lemma}

\subsection{Putting it together}

Since linking operates in the setting of partial machines
while the module system operates in its own setting,
we need to connect them
via the common setting of (regular) machines and refinements.

Consider a fixed program $P$
from which we build two partial machines $\Mach_1$ and $\Mach_2$.
$\Mach_2$ corresponds to an execution of the program
on top of a high-level specification of the subsystem
we with to verify,
whereas $\Mach_1$ runs the program $P$ together with the subsystem's implementation $M$
on top of a low-level machine.
Both machines execute a number concurrent threads $t \in T$.

In a first verification step,
we use the setting of partial machines and partial machine refinements to 
decompose $\Mach_1$ and $\Mach_2$ on a per-thread basis
by expressing them as linked machines:
\[
	\Mach_1 \refines_\id \bigJoin_{t \in T} \Mach_1^t  \hspace{3em}
	\bigJoin_{t \in T} \Mach_2^t \refines_\id \Mach_2
\]
Thanks to Lemma~\ref{lemma:mono},
it is sufficient to show that for each thread $t$:
\[
		\Mach_1^t \refines_f \Mach_2^t \,
\]
We expand the definition of $\refines_f$
to move back to the setting of regular machines,
so that we now need to prove for each thread $t$:
\[
	\forall \Env \in \EC{\Mach_1^t}\,.\,
		\Mach_1^t \langle \Env \rangle \refines
		\Mach_2^t \langle f(\Env) \rangle
\]
Then, in order to prove this per-thread property,
we carefully define a pair of layer interface specifications
$L_1(t, \mathcal{E})$ and $L_2(t, \mathcal{E}')$,
parametrized by a thread identifier and an environment context,
so that the following refinements hold for all $t, \mathcal{E}, \mathcal{E}'$:
\begin{gather*}
	\Mach_1^t \langle \Env \rangle \refines \kw{LAsm}(L_1(t, \Env), P \oplus M) \\
	\kw{LAsm}(L_2(t, \Env'), P) \refines \Mach_2^t \langle \Env' \rangle
\end{gather*}
\ronghui{Fix LAsm machine}
If we can show that
$ \ltyp{L_1(t, \mathcal{E})}{R}{M}{L_2(t, f(\mathcal{E}))} $,
then the soundness theorem of the module system
can be used to bridge the remaining gap.


