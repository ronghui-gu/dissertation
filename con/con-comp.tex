
\section{Certified Thread-safe Compilation}
\label{sec:comp}

In our approach, interaction between threads happens only through the 
abstract log, and concrete memory is always thread-local.
In Section~\ref{sec:con:base}, we already saw how this yields a big
improvement over the previous state of the art, because it lets us
give a \emph{C level} specification to primitives that handle
communication between threads. In particular, we can write the
implementation of the $\yield$ primitive in assembly, and because the
behavior is deterministic (given the log in the abstract state) we can
write a sequential-style specification in which $\yield$ immediately
returns with the abstract state suitably modified.
In order to take advantage of this, we also need a C compiler which
can compile the programs that call $\yield$. 
In this section we show how to adapt the CompCertX verified
separate compiler for this task.

In fact, most of the compiler can be reused as is. This is because we
do not need to consider parallel composition of thread behaviors
\emph{at the C level}. As the source language, we reuse our
ClightX, which is the CompCert
Clight subset of C parameterized with layer interfaces.
Each program is compiled into a
target program in the per-thread TAsm language, and we prove that the
compiled code refines its source. Then our concurrent linking
theorems (Lemmas~\ref{lemma:mono},~\ref{thread_composition}, and~\ref{thread_compose})
state that the individual threads compose into a
single, concurrent TAsm program. %, and that theorem can be stated entirely in terms
% of assembly code.

The final thread composition is subtle even at the assembly level,
because each thread's stack and thread-local memory must be combined
into a single global memory state in the CompCert memory model. To
prove that this is possible, we have to maintain a detailed invariant
about how the memory is partitioned, and to extend the abstract state
with extra information to make block allocations deterministic.

%%%%%%%%%%%%%%

% These paragraphs are Tahina's previous section introduction, can 
% reuse as needed.
\ignore{
If the specification of the primitive being implemented already
involves any need for communication with other threads, then such need
is recorded as events and operations on the abstract state of the
overlay interface, which the implementation will need to refine. We
saw in Section~\ref{sec:prog} that, if the primitive is implemented in
assembly, then it is enough to prove refinement between this primitive
and its assembly implementation calling primitives of the underlay
interface where communication with other threads is also recorded as
events and operations on the abstract state of the underlay interface.

In this section, we argue that this principle is still valid if the
primitive is implemented in C rather than assembly. Indeed, we reuse
Gu et al.'s ClightX \cite[\S 4]{dscal15}, the formal semantics of the
CompCert Clight subset of C parameterized with layer interfaces, in
the same way as for sequential C code, with abstract state and
primitives also serving to model concurrency and thread
communication. Then, similarly to the sequential setting by Gu et al.,
we prove that, on the one hand, each C function refines the primitive
it claims to be implementing. Then, on the other hand, we argue that
Gu et al.'s CompCertX verified separate compiler \cite[\S 6]{dscal15}
can be reused in the same way here, to prove that the compiled
assembly code refines its C source. Thus, by transitivity, the
compiled assembly code will refine the primitive its source C code
claims to implement.

This way, just like an assembly thread, we can consider the behavior
of a C thread as if it were running standalone sequentially, by
abstracting other threads away through the abstract state and
primitives (logs, etc.): the $\yield$ primitive can be called by a
C function, and it is \emph{deterministic} (given the log in the
abstract state) and returns with the abstract state suitably modified
by other threads (thus allowing communication between threads.)

However, contrary to assembly code, we do not need to consider any
parallel composition of thread behaviors \emph{at the C level}, thanks
to verified separate compilation and linking, since it is enough to
link the behaviors of assembly threads for the purposes of refining
our concurrent abstraction layers.
}

\paragraph{Verified separate compilation and linking with CompCertX} 
Replacing  LAsm target with our per-thread TAsm assembly
language, we first recall the CompCertX correctness statement
pictured in Figure~\ref{fig:compcertx}: starting from a memory state $m$
(provided by the caller of the primitive being implemented), if the
ClightX function $f$ runs on top of a layer $L$ and produces a
concrete memory state $m'$, then the compiled function $f$ in TAsm
runs and produces a memory state $m''$, introducing some \emph{memory
  injection} $j$ between $m'$ and $m''$ (written
$\inject{j}{m'}{m''}$). This memory injection is a memory
transformation \cite[\S 5.4]{leroy08} due to the different handling of
stack frames between ClightX and TAsm.


\begin{figure}
\[
\xymatrix@R=3pt@C+=3pt{
& v, m', a'
\ar@<-3ex>@{_{(}-->}[dd]^j
\ar@{_{(}-->}[dd]^j
\ar@<3ex>@{==}[dd]
\\
\genfrac{}{}{0pt}{0}{l}{\rho}, m, a
\ar[ur]^{\llbracket f \rrbracket_{\text{ClightX}}(L)}
\ar@{-->}[dr]^{ \llbracket f \rrbracket_{\text{TAsm}}(L)} \\
l \approx m(\rho(\textsf{ESP}))
& \rho', m'', a'
}
\]
\caption{CompCertX per-thread correctness statement} \label{fig:compcertx}
\hrulefill
\end{figure}

Then, to implement the primitives of an overlay interface $L_2$ on top
of an underlay interface $L_1$ by some ClightX module $M_C$ for some
primitives of $L_2$ and some TAsm module $M_{\text{Asm}}$ for other
primitives of $L_2$, for any fixed thread $t$ and environment context
$\mathcal E$, we follow the sequential setting, first splitting $L_2$ into C-style
and assembly-style primitives by writing $L_2 = L_{2, \text{C}} \oplus
L_{2, \text{Asm}}$ , then implementing $L_{2, \text{Asm}}$ with TAsm
code $M_{\text{Asm}}$, and $L_{2, \text{C}}$ with ClightX code
$M_{\text{C}}$, so as to prove per-thread refinement at the assembly
level using forward downward simulations as described in \cite[\S
  3.3]{dscal15} and Definition~\ref{def:popl15-layers}, $L_{2,
  \text{Asm}} \leqslant_R \llbracket M_{\text{Asm}} \rrbracket L_1$
(and thus $L_{1} \vdash_R M_{\text{Asm}} : L_{2, \text{Asm}}$), and at
the C level, $L_{2, \text{C}} \leqslant_R \llbracket M_{\text{C}}
\rrbracket L_1$. Then, we compile $M_{\text{C}}$ into TAsm using
CompCertX, thus obtaining $\llbracket M_{\text{C}} \rrbracket L_1
\leqslant_{j} \llbracket \text{CompCertX}(M_{\text{C}}) \rrbracket
L_1$ for some memory injection $j$, and thus by transitivity $L_{2,
  \text{C}} \leqslant_{j \circ R} \llbracket
\text{CompCertX}(M_{\text{C}}) \rrbracket L_1$, which can be rewritten
into $L_{1} \vdash_R \text{CompCertX}(M_{\text{C}}) : L_{2, \text{C}}$
since $j \circ R = R$. Finally, by horizontal composition, we link the
obtained assembly codes to obtain per-thread layer refinement, for any
fixed thread $t$ and environment context $\mathcal E$: $L_1 \vdash_R
M_{\text{Asm}} \oplus \text{CompCertX}(M_{\text{C}}) : L_2$.

\paragraph{Thread-safety of CompCertX: parallel composition and concrete memory states}

We then have to show that the output of CompCertX is compatible with
parallel composition of threads at the TAsm level, and in particular
that the per-thread specifications of concurrent primitives such as
$\yield$ are compatible with verified compilation a la CompCert.  This
might sound trivial: as we presented it so far, $\yield$ only modifies
the abstract state, which should not interact with the compiled
assembly code. But unfortunately, the exact semantics of primitive
calls actually also have to modify the concrete memory state. This is because
of a small snag which we glossed over until now: stack frames.

\ignore{
We argue that Gu et al.'s CompCertX is sound to be used for per-thread
verified separation \vilhelm{what's ``verified separation''?} and linking, in a way compatible with parallel
composition of threads at the TAsm level. The key reason is due to the
fact that the per-thread specifications of concurrent primitives such
as $\yield$ are in fact compatible with verified compilation a la
CompCert, which we are explaining in this paragraph.
}

\begin{figure}
\begin{small}
\begin{tabular}{ll}
$\load{m}{\ell} = \some{x}$ & reading from location $\ell$ in memory $m$ succeeds and value $x$ is read \\
$\store{m}{\ell}{v} = \some{m'}$ & storing value $v$ to location $\ell$ in  $m$ succeeds and yields $m'$ \\
$\alloc{m}{l}{h}$ & allocating a new block in $m$ with permissions for offsets from $l$ to $h$ \\
$\nextblock{m}$ & number of memory blocks already allocated in $m$ \\
$\liftnextblock{m}{n}$ & allocating $n$ empty memory blocks \\
&(satisfies $\liftnextblock{m}{n+1} = \liftnextblock{\alloc{m}{0}{0}}{n}$ and $\liftnextblock{m}{0} = m$) \\
$\extends{m}{m'}$ & memory extension between $m$ and $m'$ \cite[\S 5.2]{leroy08} \\
$\inject{j}{m}{m'}$ & memory injection between $m$ and $m'$ \cite[\S 5.4]{leroy08} \\
\end{tabular}
\end{small}
\caption{The CompCert memory model \cite{leroy08}: notations} \label{fig:mem}
\hrulefill
\end{figure}

Modeling the $\yield$ primitives when performing parallel
composition at the assembly level (from TAsm to LAsm, Lemma~\ref{thread_composition}) requires
maintaining an invariant on the concrete global memory state $m$ of
the LAsm machine and the per-thread concrete memory state $m_i$ of
each thread $i$'s TAsm machine. The concrete memory state of a thread
only contains thread-private memory, in particular its stack. Whenever
a function is called, a fresh memory block has to be allocated\footnote{We use the CompCert memory model \cite{leroy08}(notations in Fig.~\ref{fig:mem}), where a memory state consists in finitely many blocks within which finitely many memory locations can be accessed in such a way that pointer arithmetics can be performed only within one memory block at a time.} in the
concrete memory for its stack frame. This means that, at the TAsm
level, a function called within a thread will allocate its stack frame
into the thread-private memory state, and conversely, a thread is
never aware of any newer memory blocks allocated by other
threads. However, at the LAsm level, all stack frames have to be allocated
in the global memory regardless of which thread they belong to;
thus, in the TAsm to LAsm parallel composition proof, we need to account
for all such stack frames.

One possible way could be to consider maintaining memory injections
from each TAsm per-thread memory state to the LAsm global memory state
(so that per-thread $\yield$ would not modify the thread-private
memory at all); but proofs involving memory injections are notoriously
hard since they involve transformations of pointer values.

Thus, we instead chose an easier solution \cite[\S 5.2]{leroy08}. Contrary to memory injections,
memory extensions (written $\extends{m_1}{m_2}$, following the
notation in \cite[\S 5.2]{leroy08}) are memory transformations that do
not require any transformation on the values
stored in memory states. The main issue is that if there is a memory
extension between $m_i$ and $m$, they need to have the same number of
memory blocks allocated, even though the actual permissions of memory
locations within those memory blocks may differ (more permissions in
$m$ than in $m_i$) \cite{compcert-mem-v2}. Thus, to make our parallel composition proof go
through, we need to modify the semantics of per-thread $\yield$ so
that it allocates enough empty memory blocks (within which no memory
locations have any permissions) corresponding to the stack frames
allocated by other threads. So, for each thread $i$, we only need to
maintain memory extension between $\liftnextblock{m_i}{\nextblock{m} -
\nextblock{m_i}}$ (the thread-local memory $m_i$ adjusted by
allocating enough empty blocks to match the number of blocks in $m$)\footnote{for any memory state $m'$,
$\nextblock{m'}$ is the total number of blocks allocated in $m'$, and
$\liftnextblock{m'}{n}$ allocates $n$ empty blocks in $m'$},
and the global memory state $m$.

\ignore{
\tahina{Here, introduce the new per-thread semantics of $\yield$ with
  the accurate handling of $m$ with $\liftnextblockOP$. --- More
  generally, can we try to describe the correctness proof of parallel
  composition (TAsm to LAsm) in Sec. \ref{sec:layer} and/or
  \ref{sec:prog}, first \emph{pretending} that $\yield$ would not
  change the thread-private concrete memory state (and so, that no
  further memory blocks are ever allocated either in thread-privates
  or in the global memory), then here in this section, ``refine'' our
  explanation by showing how that correctness proof changes
  wrt. conveying the $\nextblockOP$ into the log events?}
}

More formally, in Figure~\ref{fig:algmem}, we introduce a notion of \emph{algebraic memory model}
to model the relationship $\disjointunion{m_1}{m_2}{m}$ between the private memory states $m_1, m_2$ of two disjoint
thread sets and the global memory state $m$ after parallel composition.
\begin{definition}[Algebraic memory model] \label{def:algmem}
A memory model based on the CompCert memory model \cite{leroy08} is
said to be \emph{algebraic} if, and only if, there is a \emph{disjoint
  union} predicate $\disjointunion{m_1}{m_2}{m}$ satisfying
the axioms of Figure~\ref{fig:algmem}.
\end{definition}



\begin{figure}
\begin{mathpar}
\inferrule{
  \disjointunion{m_1}{m_2}{m}
}{
  \nextblock{m} = \max(\nextblock{m_1}, \nextblock{m_2})
}(\textsc{Disj-Nb})
\and
 \inferrule{
  \disjointunion{m_1}{m_2}{m}
}{
  \disjointunion{m_2}{m_1}{m}
}(\textsc{Disj-Comm})
\and
\inferrule{
  \disjointunion{m_1}{m_2}{m} \\
  \nextblock{m_1} \leq \nextblock{m_2}
}{
  \disjointunion{m_1}{\liftnextblock{m_2}{n}}{\liftnextblock{m}{n}}
}(\textsc{Disj-Liftnb-Right})
\and
\inferrule{
  \disjointunion{m_1}{m_2}{m} \\
  \nextblock{m_1} \leq \nextblock{m_2}
}{
  \disjointunion{\liftnextblock{m_1}{n}}{m_2}{\liftnextblock{m}{n - (\nextblock{m} - \nextblock{m_1})}}
}(\textsc{Disj-Liftnb-Left})
\and
\inferrule{
  \disjointunion{m_1}{m_2}{m} \\
  \nextblock{m_1} \leq \nextblock{m_2} 
}{
  \disjointunion{m_1}{\alloc{m_2}{l}{h}}{\alloc{m}{l}{h}}
}(\textsc{Disj-Alloc})
\and
\inferrule{
  \disjointunion{m_1}{m_2}{m}
}{
  \disjointunion{m_1}{\store{m_2}{\ell}{v}}{\store{m}{\ell}{v}}
}(\textsc{Disj-Store})
\and
\inferrule{
  \disjointunion{m_1}{m_2}{m} \\
  \load{m_2}{\ell} = \some{v}
}{
  \load{m}{\ell} = \some{v}
}(\textsc{Disj-Load})
\end{mathpar}
\caption{Algebraic memory model} \label{fig:algmem}
\hrulefill
\end{figure}

Rules in Figure~\ref{fig:algmem} define the notion of algebraic memory
we use to prove the correctness of single-processor parallel
composition: any memory operation in a per-thread memory state shall
reflect in the corresponding global memory state. The condition
$\nextblock{m_1} \leq \nextblock{m_2}$ ensures that a non-empty memory
block can be allocated only in the memory state of the \emph{active}
thread, so that if a thread calls $\yield$, then, once it regains
control, it is guaranteed that all new memory blocks were allocated by
other threads. Similarly to (\textsc{Disj-Store}), $\disjointunionOP$
is compatible with $\mathsf{free}$, which only clears the permissions
of memory locations without changing the number of memory blocks.

Rule \textsc{Disj-Liftnb-Left} is critical to our parallel composition
proof, since it allows to maintain the disjoint union invariant when a
thread yields and gets back control, allocating empty memory blocks in
its private memory state $m_1$ to account for blocks allocated in $m$
by other threads.

Based on our disjoint union for two memory states, we then use rule
\textsc{Disj-Liftnb-Right} to generalize to $N$ threads by saying that
$m$ is a disjoint union of the private memory states $m_1, \dots, m_N$
of $N$ threads (on a single processor) if, and only if, there exists a
memory state $m'$ such that $m'$ is a disjoint union of $m_1, \dots,
m_{N-1}$ and $\disjointunion{m_N}{m'}{m}$ holds.

\begin{lemma}
Let $m_1, m_2$ two partial memory states containing the thread-private memories of two disjoint thread sets. Then, we say that global memory state $m$ is \emph{a disjoint union} of $m_1$ and $m_2$ (written $\disjointunion{m_1}{m_2}{m}$) if, and only if, all the following conditions hold:
$\mathsf{nb}(m) = \max(\mathsf{nb}(m_1), \mathsf{nb}(m_2))$; $\forall i$, memory extension from $\mathsf{liftnb}(m_i, \mathsf{nb}(m) - \mathsf{nb}(m_i))$ to $m$; and no valid common memory locations in both $m_1$ and $m_2$; then we satisfy the algebraic memory model axioms of Figure~\ref{fig:algmem}.
\end{lemma}

Remember that, similarly to CompCert, CompCertX can only compile
ClightX code that calls primitives that are deterministic and are
preserved by memory transformations (extensions and injections)
introduced by compilation passes. To maintain per-thread determinism
of $\yield$ while the exact behaviors of other threads is not
known, it is necessary to record the number of missing blocks
allocated by other threads in each per-thread log in its abstract
state, so that per-thread $\yield$ can remain
deterministic. Finally, having $\yield$ allocate a number of
blocks only determined by the per-thread log instead of any concrete
memory state makes it compatible with memory transformations
introduced by CompCertX. Thus, $\yield$ is valid to be used in
ClightX code being compiled by CompCertX.
