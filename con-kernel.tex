\chapter{Case Study: Building Certified Concurrent OS Kernels}
\label{chap:conkernel}

To demonstrate our concurrent framework,
we extend the \mCTOS{} single-core verified kernel into a
concurrent kernel \cCTOS{} running on multi-core hardware.
\ronghui{Fix the road map}

\section{Overview of the \cCTOS\ kernel}
\label{sec:con:overview}

\begin{figure}[t]\centering
\includegraphics[scale=.78]{figs/sysarch}
\caption{System architecture for the \cCTOS\ kernel.}
%\rule[0in]{\columnwidth}{.15mm}
\label{fig:sysarch}
\hrulefill
\end{figure}

Figure~\ref{fig:sysarch} shows the system architecture of the
\cCTOS\ kernel. The \cCTOS\ system was initially developed in the
context of a large DARPA-funded research project.  
\ronghui{Fix}
It is a concurrent
OS kernel that can also double as a hypervisor.  It runs on a military
land vehicle with a multicore Intel Core i7 machine. On top of the
\cCTOS\ kernel, we run 6 Ubuntu Linux systems as guests (one each on
the first six cores). Each virtual machine runs several RADL (The
Robot Architecture Definition Language) nodes that have fixed hardware
capabilities such as access to GPS, radar, \etc{}  The kernel also
contains a few simple device drivers (\eg, interrupt controllers,
serial and keyboard devices). More complex devices are either
supported at the user level, or directly passed through (via
IOMMU) to various guest Linux VMs. By running different RADL nodes in
different VMs, \cCTOS\ provides strong isolation support so that even
if attackers take control of one VM, they still cannot break into
other VMs to compromise the overall mission of the drone.
%\vilhelm{Maybe cut this paragraph, and drop the ``-like''? The figure does not mention 
%the VMs, GPS, etc, and the text creates the impression that 
%we ran the certified version on the vehicle. }

Within \cCTOS, we have various shared objects such as lock modules
(Ticket, MCS), sleep queues (SleepQ) for implementing queueing locks
and condition variables, message queues (MsgQ) for waking up a
thread on another CPU, container-based physical and virtual memory
management modules (Container, PMM, VMM), a Lib Mem module for
implementing shared-memory IPC, synchronization modules (FIFOBBQ,
CV), and an IPC module. Within each core (the purple box), we have
the per-CPU scheduler, the kernel-thread management module, the process
management module, and the virtualization module (VM Monitor). Each
kernel thread has its own TCB, context, and stack.

\paragraph{What have we proved?}
Using \CTOS{} framework, we have successfully built a fully certified version of
the \cCTOS\ kernel and proved its contextual refinement property with
respect to a high-level deep specification for \cCTOS.  This important
functional correctness property implies that all system calls and
traps will strictly follow the high-level specification and always run
{\em safely} and {\em terminate} eventually; and there will be no data
race, no code injection attacks, no buffer overflows, no null pointer
access, no integer overflow, \etc{}

\begin{figure}[t]
%\includegraphics[scale=0.1]{figs/layer_diagram.png}
\includegraphics[width=1.0\textwidth]{figs/layer_diagram.pdf}
\caption{Layer hierarchy of \cCTOS{} kernel}
\label{fig:layer_diagram}
\hrulefill
\end{figure}

\ignore{In addition to all the concurrent objects described in Sec. \ref{sec:prog},
the kernel also implements an MCS lock \cite{mcs91}, paging-based dynamically
allocated virtual memory,
a synchronous inter-process communication (IPC) protocol implemented using the
queuing lock, and a shared-memory IPC protocol with a shared page.
Using the techniques and strategies presented in the paper, we have
successfully specified and verified the \cCTOS{} kernel in the Coq proof assistant.}

Figure~\ref{fig:layer_diagram} shows the layer hierarchy of  \cCTOS{}.
The gray boxes denote machine models, on top of which the
per-CPU (blue) and per-thread (yellow) layers are built. Orange boxes
are user threads.

The bottom-left portion of the figure illustrates the contextual refinement
among machine models (\cf Section~\ref{sec:mach}), where we gradually turn the nondeterministic
multicore machine model with arbitrary interleavings
among different processors into a CPU-local machine model that is parameterized
over the behaviors of other processors,
where the switch points only appear at shared operations. This new machine
model allows us to reason about programs running on different processors locally,
and later compose them formally to reason about the whole program.

On top of this abstract machine model, the code of the concurrent kernel is specified and 
verified through 59 abstraction (logical) layers. For each CPU,
we introduce the atomic (ticket and MCS) spinlock objects
(\cf Section~\ref{sec:base:lock} and \ref{sec:con:mcs}).
On top of that, the device drivers running inside kernel are verified
(\cf Section~\ref{sec:con:device}).
Then there are multiple layers used to introduce memory management units (\cf Section~\ref{sec:con:mem}),
the thread context, atomic queue object (\cf Section~\ref{sec:con:queue}), and scheduler methods (\cf Section~\ref{sec:con:thread}).
The \texttt{PThread} layer is the topmost layer built for a particular processor.
There, the scheduler primitives like \texttt{yield} and \texttt{sleep}
are specified in a small-step manner, similar to how they are implemented
in C and assembly. Their specifications do not follow the C calling conventions
and thus cannot be called by C code. Above \texttt{PThread}, we then build up the 
per-thread layers that support thread-local reasoning for each CPU.
We first introduce the layer \texttt{PHThread} which defines big-step semantics for
the scheduler primitives that can be invoked from the C level (\cf Section~\ref{sec:con:thread}).
Finally, above the \texttt{PHThread} layer, we verify the
IPC and trap handler modules.

The verified kernel source code (both C and assembly) is extracted using 
Coq's extraction mechanism. The C source code is then compiled by the
extracted verified compiler and merged with the extracted assembly source to
produce the final assembly source code for our verified concurrent kernel. 


More importantly, because $\sem{\rm{}x86mc}{K\join{}P}$ refines
$\sem{\rm\cCTOS}{P}$ for any program $P$, we can also derive the
important {\em behavior equivalence} property for $P$, that is,
whatever behavior a user can deduce about $P$ based on the high-level
specification for the \cCTOS\ kernel $K$, the actual linked
system $K\join{}P$ running on the concrete x86mc machine would indeed
behave exactly the same.  All global properties proved at the
system-call specification level (including information-flow
security~\cite{costanzo16}) can be transferred down to the lowest
assembly machine.


\ignore{
Finally, we also proved that there is no stack overflow or memory
exhaustion in the kernel using recent techniques developed by
Carbonneaux~{\em et al}~\cite{veristack,ccrb15}.
\vilhelm{Maybe omit this paragraph, it draws attention to the fact that stack usage is 
  not tracked by CompCert. We mention it later in the text when talking about the Thread management module.}
}



%\sectskip
\vspace{-5pt}
\section{Certifying the {\mCTOS} Kernel}
\label{sec:base}
%\asectskip

\ignore{
\begin{figure}
\begin{minipage}{.22\textwidth}
\lstinputlisting [language = C] {source_code/lock_producer.c}
\end{minipage}\hfill
\begin{minipage}{.22\textwidth}
\lstinputlisting [language = C] {source_code/lock_consumer.c}
\end{minipage}
\ifTR{}{\vspace*{-5pt}}
\caption{Lock-based producer-consumer implementation}
\label{fig:exp:lock}
\end{figure}
}


\begin{comment}
\begin{figure*}
\begin{center}
\begin{scriptsize}
\begin{tabular}{ |l|l||l|p{4.5cm}| }
  \hline
  \multicolumn{2}{|c||}{\textbf{Memory Management}} 
  & \multicolumn{2}{|c|}{\textbf{Thread and Process Management}} \\
  \hline
  \hline    
  \multicolumn{2}{|l||}{\textbf{abstract state}} 
  & \multicolumn{2}{|l|}{\textbf{abstract state}}\\
  \hline
  \verb"AT" & physical page allocation table
  & \verb"kctxp" & kernel context (\verb"kctx") pool\\
  \hline 
  \verb"PFInfo" & save the address and \verb"PC" that page fault occurs
  & \verb"Ltdqp" & low abstract thread queue pool\\
  \hline
  \verb"ptp" & page table (\verb"pt") pool 
  
  & \verb"Htdqp"& high abstract thread queue (\verb"Htdq") pool\\ 
  \hline
   \verb"ipt"& whether \verb"pt"'s invariant  should hold or not
  
  & \verb"uctxp" & user context pool\\
  \hline
\verb"PT" & index of the current \verb"pt"
  & \verb"chanp" & channel pool\\
  \hline
  \verb"pbit" & bit map for free \verb"pt" indexes
  & \verb"Htcbp"& high abstract TCB pool \\
  \hline
  \multicolumn{2}{|l||}{\textbf{primitive}} 
  & \multicolumn{2}{|l|}{\textbf{primitive}}\\
  \hline	
  \verb"setcr3" & set the starting address of the \verb"pt"
  & \verb"kctx_new" & allocate the first free \verb"pt" and \verb"kctx"\\
  \hline
  \verb"meminit" & initialize the allocation table
  & \verb"Henqueue" & append a thread to the \verb"Htdq"\\
  \hline
  \verb"palloc" & allocate a page 
  & \verb"thread_kill" & kill and free a thread\\
  \hline
  \verb"pt_insrt" & insert a page map into a given \verb"pt" 
  & \verb"thread_sleep" & sleep, schedule to the 1st ready thread\\
  \hline
  \verb"pt_resv" & allocate a page for a given linear addr 
  & \verb"kctx_switch" & switch \verb"kctx" between threads\\
  \hline 
  \verb"PTInit" & init kernel's \verb"pt" and enable paging 
  & \multirow{2}{*}{\texttt{resv\_chan}} & 
  receive msg from the channel, wake\\
  \cline{1-2}
  \verb"pt_new" & allocate the first free \verb"pt" & & up the first sleeping thread
    \\	  
  \hline
  \hline
  \multicolumn{2}{|c||}{\textbf{Virtualization}}
  &\multicolumn{2}{|c|}{\textbf{Trap Handler}} \\
  \hline
  \hline    
  \multicolumn{2}{|l||}{\textbf{abstract state}} 
  & \multicolumn{2}{|l|}{\textbf{primitive}} \\
  \hline
  \verb"npt" & nested page table for guest
  & \verb"trap_arg" & get arguments of system calls\\
  \hline
  \verb"hctx"& host context
  & \verb"hpagefault" & page fault handler\\ 
  \hline
  \verb"vmcb" & virtual machine (\verb"VM") control control block
  & \verb"sys_yield" & system calls for yielding\\
  \hline
  \verb"xvmst" & registers not saved in \verb"vmcb" 
  & \verb"sys_wait_chan" & system calls to sleep on a channel\\
  \hline
  \multicolumn{2}{|l||}{\textbf{primitive}} 
  & \verb"sys_run_vm" & system calls to run \verb"VM"\\
  \hline	
  \verb"npt_insrt" & insert into the nested page table
  & \verb"sys_proc_create" & system calls to create a process\\
  \hline
  \verb"switch2guest" &  switch to guest mode 
  & \verb"sys_getexitinfo" 
  & get the information about \verb"VM" exit\\
  \hline
  \verb"set_vmcb" & set value in virtual machine control block
  & \verb"sys_injectevent" & inject interrupt and exception to \verb"VM"\\
  \hline 
  \verb"run_vm" & save host context, restore \verb"vmcb", start \verb"VM" 
  & \verb"kernel_init" & initialization function of the kernel\\  
  \hline  

\end{tabular}
\end{scriptsize}
\caption{Key abstract states and primitives for \mCTOSbase{} and \mCTOShyper{}}
\label{table:layers}
\end{center}
%\vspace*{-14pt}
\end{figure*}
\end{comment}

\begin{comment}
\begin{figure}
\includegraphics[scale=0.38]{figs/memory_management_layer}	
\caption{Layers of PreInit and memory management}
\label{fig:base:mm:layers}
\vspace*{-14pt}
\end{figure}

\begin{figure}
\includegraphics[scale=0.37]{figs/process_management_layer}	
\caption{Layers of process management}
\label{fig:base:pm:layers}
\vspace*{-14pt}
\end{figure}

\begin{figure}
\includegraphics[scale=0.37]{figs/trap_management_layer}	
\caption{Layers of trap management}
\label{fig:base:tm:layers}
\vspace*{-14pt}
\end{figure}
\end{comment}

\ignore{
\begin{figure}
\includegraphics[scale=0.34]{figs/mctos_layer}	
\caption{Layers of \mCTOSbase{}}
\label{fig:base:ctos:layers}
\vspace*{-14pt}
\end{figure}
}

In this section, we present the main components of the certified {\mCTOS} kernel. 
The pre-initialization module is the bottom layer that connects to the
 \emph{CPU-local machine model} $\mach{loc}$, instantiated with a 
 particular \emph{active CPU} (\cf Sec.~\ref{subsec:spec:seq}).
The trap handler contains the top layer that provides system call interfaces
and serves as a specification of the whole kernel,
instantiated with a particular active thread
running on that active CPU.
Our main theorem states that any global properties proved at the topmost
abstraction layer can be transferred down to the lowest hardware machine.
In this section, we explain selected components
in more details.

\ignore{
\paragraph{Pre-initialization module}
\label{sec:base:preinit}
only contains the bottommost layer
\code{PreInit}. It axiomatizes the x86 hardware behaviors of a particular active CPU.
These behaviors include page table walk upon memory load when paging is turned on, 
saving and restoring the trap frame in the case of interrupts and exceptions (\eg, page fault), 
and the data exchange between devices and memory.
\ignore{The abstract state of the \code{PreInit} layer consists of control registers, FLAGS registers,
registers of devices, an E820 memory map \code{MM} (set up by the bootloader),
and a CPU-mode flag (either kernel or user mode).
\ignore{Its primitives consist of
getter-setter functions for control registers and \code{MM},
and a function models the transition between user and kernel mode.}
}

As shown in Fig.~\ref{fig:spec:memmodel}(a),
the hardware memory management unit (MMU) is modeled 
in a way that mirrors the paging hardware.
When paging is enabled (as indicated by \code{CR0}),
memory accesses made by both the kernel and the user programs
are translated using the page map pointed to by \code{CR3}.
When a page fault occurs,
the fault information is stored in \code{CR2},
the CPU mode is switched from user mode to kernel mode,
and the page fault handler is triggered.
}
\ignore{
Some privileged
memory regions (\eg, allocation table) and 
instructions (\eg, modifying control registers)
are only available in kernel mode.}
\ignore{
\code{CR0} selects the memory protection mode,
\code{CR2} stores the Page Fault Linear Address (PFLA)
as well as the address of the instruction that caused the page fault, and
\code{CR3} stores the starting point of the page map.
}
\ignore{
The switch function models the change of the \code{ikern} flag
and the remaining tasks involved with trap handling,
such as saving and restoring user and kernel contexts,
and dispatch over the trap type,
are verified at the assembly level.
}


\ignore{
{\color{red}Jan: Do we need the following paragraph?}
The initialization primitive at this bottom-most layer is the bootloader,
which initializes \code{MM} and necessary drivers
(tsc, disk, console, timer, keyboard, serial, {\it etc.}),
loads the kernel into the memory,
and sets the initialization flag to be \code{true}.
}

\paragraph{Lock module}
\label{sec:base:lock}
of the {\mCTOS} kernel provides fine-grained lock objects
as the base of synchronization mechanisms.

\begin{figure}
\lstinputlisting [language = C, multicols=2] {source_code/ticket_lock.c}
\vspace{-5pt}
\caption{Pseudocode of ticket lock implementation.}
\label{fig:exp:ticket_lock}
\vspace{-10pt}
\end{figure}

\textbf{Ticket Lock}
depends on an \emph{atomic ticket object}, which consists of two 
fields: \code{ticket} and \code{now}.
Figure~\ref{fig:exp:ticket_lock} shows one implementation of a
ticket lock. The atomic increment to the ticket is achieved
through the atomic \texttt{fetch-and-increment} (FAI) operation (implemented using
the \texttt{xaddl} instruction with the \texttt{lock} prefix in x86).
As described in Sec.~\ref{subsec:spec:seq}, 
the \emph{{\intptext}s} at this abstraction level have been shuffled and merged so that there is exactly one {\intptext}
in front of each atomic operation. 
Thus, the lock implementations generate a list of events;
\ignore{The atomic operations generate ticket-object events;}
for example, when CPU $t$ attempts to acquire the lock $i$, it continuously generates 
the event ``$\event{t.get\_now\ i}$" (line~9)
until the latest \code{now} is increased
to the ticket value returned by the event ``$\event{t.inc\_ticket\ i}$" 
(line~8), and then followed by the event ``$\event{t.pull\ i}$" (line~10):\vspace{-5pt}
\[
\includegraphics[scale=.6]{figs/ticket_log}
\vspace{-5pt}
\]
\ignore{
The event list is as below:
$$[\intp,\event{t.inc\_ticket\ i},\intp,\event{t.get\_now\ i},\cdots,\intp,\event{t.get\_now\ i}]$$
}

Certifying the atomic ticket lock object consists of two main proof obligations:
(1) the lock guarantees \emph{mutual exclusion}, and (2) the lock  
is \emph{starvation-free}.

\emph{Mutual exclusion}
is straightforward for a ticket lock.
At any time, only the thread whose ticket is equal to
the current serving ticket (\ie, \code{now})
can hold the lock. 
Furthermore, each thread's ticket is unique
as the $\texttt{fetch-and-increment}$ operation is atomic
(line 8). 
Thanks to this \emph{mutual exclusion} property,
it is safe to \emph{pull} the shared memory associated with
the lock $i$ to the local copy
at line 10.
Before releasing the lock, the local copy is \emph{pushed}
back to the shared memory at line 13.

\emph{Starvation-freedom} can be proved
by showing that lock acquire eventually succeeds.
To prove this, we impose a set of invariants on the environment
context at this layer.

\vspace{-3pt}
\begin{invariant}[Invariants for ticket lock]
\label{inv:lock}
An environment context that holds the lock $i$,
(1) never acquires lock $i$ again before releasing it;
and (2) always releases lock $i$ within $k$ steps
(for some $k$).
\end{invariant}
Furthermore, we assume the hardware scheduler $\hardoracle$ is \emph{fair},
meaning that between any two consecutive events from the same thread,
there are less than $m$ events generated by other threads (for some $m$).

\vspace{-3pt}
\begin{lemma}[Starvation-freedom of ticket lock]
\label{lemma:lock}
Acquiring ticket-lock in the {\mCTOS} kernel eventually succeeds.
\ignore{\proof 
Let $n$ be the maximum number of the threads in the system.
Then (1) there are at most $n$ threads waiting before the current one;
(2) the thread holding the lock releases the lock within $k$ steps,
which generates at most $k$ events;
(3) environment context generates less than $m$ events between each step of the lock holder;
and (4) every \code{get\_now} generates one event.
Hence there are at most
$n\times m\times k$ loop iterations at line 10 in Fig.~\ref{fig:exp:ticket_lock},
and so acquiring lock always succeeds.
\qed}
\end{lemma}
Once we prove starvation freedom,
we abstract the lock implementation into atomic specifications in
the higher layers such that acquiring a lock only generates a single
event ``$\event{t.acq\_lock\ i}"$. To compose the specification with
the environment context, we prove that Invariant~\ref{inv:lock}
holds on the current CPU at overlay.

\ignore{
\begin{figure}
\lstinputlisting [language = C, multicols=2] {source_code/mcs_lock.c}
\caption{MCS Lock Implementation}
\label{fig:exp:mcs_lock}
\end{figure}
}

\textbf{MCS Lock} is known to have better scalability than ticket lock
on large numbers of CPUs.
In the {\mCTOS} kernel, we have also implemented a version of MCS 
locks~\cite{Mellor-Crummey:mcs-locks}.
The starvation-freedom proof is similar to that of the ticket lock. 
The difference is that the MCS lock release operation waits in a loop until the next waiting thread (if it exists)
has added itself to a linked list, so we need similar proofs for both acquire and release.

\vspace{-3pt}
\paragraph{Device drivers}
\label{sec:base:device}
Chen {\it et al} \cite{chen16} present a framework to allow interrupts
inside the kernel
with device drivers. \ignore{The core idea is to treat device drivers of
each device as if they were running on a ``logical'' CPU dedicated to that device,
and build a framework that systematically enforces the isolation among different
``devices'' and the rest of the kernel.} The work has been successfully ported
into our setting relatively easily thanks to the fact that their event-based
 model is consistent with our interleaving machine model.


\ignore{\subsection{Memory management}
\label{sec:base:memm}

The memory management of {\mCTOS} consists of  the
\emph{physical memory management} (4 layers), 
\emph{virtual memory management} (7 layers), and
\emph{shared memory management} (3 layers).
}

\ignore{
\begin{figure}
\includegraphics[scale=0.35]{figs/dynamic}	
\caption{The state transition of page object}
\label{fig:base:dynamic}
\vspace*{-14pt}
\end{figure}
}

\vspace{-3pt}
\paragraph{Physical memory management}
\label{sec:base:memm}
maintains the physical pages and provides a \emph{dynamic page allocator}
\texttt{palloc} (\cf Fig.~\ref{fig:exp:palloc}).
As the page allocation table $\texttt{AT}$ is shared among different CPUs, 
we associate it with a lock $\texttt{lock\_AT}$.
The dynamic page allocator is then refined into an atomic object where
the \texttt{palloc} implementation is proved to satisfy an atomic interface,
with the proof that
lock utilization for $\texttt{lock\_AT}$ satisfies Invariant~\ref{inv:lock}.
Once the dynamic page allocator is introduced as
 an atomic object, the lock acquire and lock release 
for $\texttt{lock\_AT}$
are \emph{not allowed to be invoked} at higher layers. 
Thus, in this layered approach, it is not possible
that a thread holding a lock defined in a lower layer tries to acquire another lock
introduced in a higher layer, \ie, the order that a thread acquires different
locks is guided by the order that the locks are introduced in the layers.
This implicit order of lock acquisitions prevents \emph{deadlocks} in the
{\mCTOS} kernel.

\begin{figure}
\lstinputlisting [language = C, multicols=2] {source_code/palloc.c}
\vspace{-5pt}
\caption{Pseudocode of \texttt{palloc}}
\label{fig:exp:palloc}
\vspace{-10pt}
\end{figure}



Another function of the physical memory management is to dynamically
track and bound the memory usage of each thread. A \emph{container}
object is used to record information for each thread (array \code{cn}
in Fig.~\ref{fig:exp:palloc}); one piece of information tracked is the
thread's \emph{quota}. Inspired by the notions of containers and
quotas in HiStar~\cite{zeldovich06}, a thread
in {\mCTOS} is spawned with some quota specifying the maximum number
of pages that the thread will ever be allowed to allocate. As can be
seen in Fig.~\ref{fig:exp:palloc}, \code{palloc} returns an error
code if the requesting thread has no remaining quota (lines~4 and~5), 
and the quota is decremented when a page is successfully allocated (line~17).
Quota enforcement allows the kernel to prevent a denial-of-service attack,
where one thread repeatedly allocates pages and uses up all available
memory (thus denying other threads from allocating pages). From a security
standpoint, it also prevents the undesirable information channel between 
different threads that occurs due to such an attack.


\begin{figure*}[th]
$$
\begin{array}{c|c}
\begin{array}{cc}
(a) &\
      \begin{array}{c}
			\includegraphics[scale=.33]{figs/mem_model_1} 
		\end{array}
\end{array}
& 
\begin{array}{cc}
(b) & 
\begin{array}{c}
\includegraphics[scale=.35]{figs/mem_model_2}
\end{array}
\end{array}
\vspace*{-14pt}
\end{array}
$$
\caption{(a) Hardware MMU using two-level page map; (b) Virtual address space $i$ set up by page map $i$.}
\label{fig:spec:memmodel}
\vspace*{-10pt}
\end{figure*}

\vspace{-3pt}
\paragraph{Virtual memory management}
provides consecutive virtual address spaces on top of physical memory management.
Because much of the code assumes that 
the memory management sets up the virtual address space properly,
initialization has been a sticking point.
We proved not only that the primitives of virtual memory management
manipulate the page maps correctly,
but also that the \emph{initialization procedure} sets up the two-level page maps properly
in terms of the hardware address translation.


\begin{invariant}
\label{inv:virtual}
1) paging is enabled only after the initialization of virtual memory management;
2) the memory regions that store kernel-specific data must have the kernel-only 
permission in all page maps;
3) the page map used by the kernel is an identity map
4) the non-shared parts of user processes' memory are isolated (see 
Sec.~\ref{security}).
\end{invariant}


By Inv.~\ref{inv:virtual}, we show that it is safe to
run both the kernel and user programs in the virtual 
address space when paging is enabled.
In this way, memory accesses at higher layers
operate on the basis of
the high-level, abstract descriptions of address spaces
rather than concrete page directories and page tables stored in the memory
itself.


\vspace{-3pt}
\paragraph{Shared memory management} provides a protocol to share physical
pages among different user processes. 
It provides an infrastructure to map a physical page into multiple
processes' page maps (\ie, processes' address spaces).
We prove that the shared page can only be freed after 
all processes release their ownership of that page.

\ignore{
Its verification makes use
of the ownership relation. 
For example, a user process $k_1$ can share its private physical page $i$
to another user process $k_2$ through the shared memory protocol,
and the owner set of page object $i$ will become
$\{\text{process object }k_1, \text{process object }k_2\}$.
}


\ignore{\subsection{Process management}


Process management  introduces the
\emph{abstract queue library} (4 layers),
\emph{thread management} (6 layers),
\emph{condition variable} (3 layers),
and \emph{IPC} module (2 layers).}

\ignore{\begin{figure}
\lstinputlisting [language = C, multicols=1] {source_code/enqueue.v}
\vspace{-5pt}
\caption{Specifications of local queue operations.}
\label{fig:exp:queue}
\vspace{-10pt}
\end{figure}}


\vspace{-3pt}
\paragraph{Abstract queue library}
\label{sec:base:procm} abstracts the queues implemented as \emph{doubly-linked lists}
into \emph{abstract queue states} (\ie, Coq lists).
The local {\it enqueue} and {\it dequeue} operations
are specified over the abstract lists.
As usual, we associate each shared queue with a lock.
The atomic interfaces for shared queue operations are represented by
queue events $``\event{t.enQ\ i\ e}"$ and 
$``\event{t.deQ\ i}"$,
which can be replayed to construct the shared queue. 
For instance, starting from an empty initial queue,
the resulting queue constructed from
the log of the following interleaving
is $[2,5]$. \vspace{-5pt}
\ignore{and the last atomic dequeue operation returns 3.}
\ignore{ 
if the current local log of $i$'th shared queue is
$[\intp,\event{t_0.enQ\  i\ 2},\intp,\event{t_0.deQ\ i}]$,
and the event lists generated by the \emph{environment context} at two {\intptext}s are
$[\event{t_1.enQ\  i\ 3},\event{t_1.enQ\  i\ 4}]$ and
$[\event{t_1.enQ\  i\ 5}]$, respectively,
then the complete logical log for the shared queue $i$ is:}
\[
\includegraphics[scale=.65]{figs/queue_log}
\]
\ignore{
$$[\event{t_1.enQ\  i\ 3},\event{t_1.enQ\  i\ 4},\event{t_0.enQ\  i\ 2},\event{t_1.enQ\  i\ 5},\event{t_0.deQ\ i}]$$
}

\ignore{Thus, by replaying the log, the shared queue state is calculated as $[4,2,5]$,
and the last atomic dequeue operation returns 3.}


\paragraph{Thread management}
introduces the thread control block (TCB)
and manages the resources of dynamically spawned threads (\eg, quotas) and
their meta-data (\eg, parent, children, thread state).
For each thread, one page (4KB) is allocated for its \emph{kernel stack}.
We use an external tool~\cite{veristack}
to prove that the stack usage of our compiled kernel is much
less than 4KB,
so stack overflows cannot occur inside the kernel.
\ignore{
One interesting aspect of the thread module is the context switch function. 
This assembly function saves the register set
of the current thread and restores the register set from 
the kernel context of another thread on the same CPU.
Since the instruction pointer register (\code{EIP}) and stack pointer register (\code{ESP}) 
are saved and restored in this procedure,
this kernel context switch function is verified at 
the assembly level,
and linked with other code that is verified at the C~level
and then compiled by CompCertX. }

The thread scheduling is done by the three primitives \code{yield}, 
\code{sleep}, and \code{wakeup}, using the abstract queue library
(\cf Fig.~\ref{fig:exp:fig:scheduler}). 
Each CPU has a \emph{private ready queue} \code{ReadQ}
and a \emph{shared pending queue} \code{PendQ}.
The context CPUs can insert threads to the current CPU's pending queue.
The {\mCTOS} kernel also provides a set of \emph{sleeping queues} \code{SleepQs}, which are
shared among all CPUs.
As shown in Fig.~\ref{fig:exp:fig:scheduler},
the \code{yield} primitive moves thread from
the pending queue to the ready queue
and then switches to the next ready thread.
The \code{sleep} primitive simply adds the running thread to the sleeping
queue and runs the next ready thread.
The \code{wakeup} primitive contains two cases.
If the thread to be woken up belongs to the current CPU,
then the primitive adds the thread to its ready queue.
Otherwise, \code{wakeup} adds the thread to the pending queue of the CPU it belongs to.
Except for the ready queue,
all the other thread queue operations are protected by \emph{fine-grained} locks.

\begin{figure}
\begin{center}
\includegraphics[scale=.72]{figs/scheduler} 
\end{center}
\vspace{-10pt}
\caption{Scheduling procedures of \texttt{yield}, \texttt{sleep},
and \texttt{wakeup}.}
\label{fig:exp:fig:scheduler}
\vspace{-10pt}
\end{figure}

\begin{figure}
\lstinputlisting [language = C, multicols=2] {source_code/thread_management.c}
\caption{Implementation of the Scheduler Module}
\label{fig:exp:scheduler}
\end{figure}

\vspace{-3pt}
\paragraph{Thread-local machine model}
can be introduced based on the thread management.
The first step is to 
extend the environment context
with a \emph{software scheduler} (\ie, abstracting the concrete
scheduling procedure), resulting in a new environment context $\oracle_{ss}$.
The scheduling primitives 
generate the $\event{yield}$ and
$\event{sleep}$
events, and $\oracle_{ss}$ responds with the next
thread ID to execute.
One invariant we have to require on $\oracle_{ss}$
is that a sleeping thread can be rescheduled only after 
a $\event{wakeup}$ event is generated.
The second step is to introduce
the \emph{active thread set}
to represent the target threads on the \emph{active CPU},
and extend the $\oracle_{ss}$
with the \emph{context threads},
\ie, the rest of the threads running on 
the active CPU. The composition structure is similar to the one of Lemma~\ref{lemma:compose}.
In this way,
higher layers can be built upon a thread-local machine model
with a single active thread on the active CPU
(\cf Fig.~\ref{fig:spec:refine_layer}).

\vspace{-3pt}
\paragraph{Starvation-free condition variable}
A \emph{condition variable} is a synchronization object
that enables a thread to efficiently wait for a change to be made 
to a shared state (protected by a lock). 
To make sure there is no starvation inside the \mCTOS\ kernel,
every thread waiting on a condition variable (CV) needs to be signaled within 
a certain bounded number of execution steps.
We have implemented a starvation-free version of condition variable
using a FIFO blocking bounded queue (FIFOBBQ) of condition variables,
following a similar idea to that presented in \cite[fig~5.14]{ospp11}.
However, we have found a bug in the implementation shown in that textbook \cite{anderson16}.
In some cases, the system can get stuck by allowing all the signaling and
waiting threads to be asleep simultaneously, or the system can arrive at
a dead end where the threads on the remove queue (rmvQ) can no longer be woken up.
In {\CTOS}, this issue is fixed by postponing the removal of
the CV of a waiting thread from the queue, until the waiting thread finishes its
work. As shown in Fig.~\ref{fig:exp:fifo}, the remover is now responsible
for removing itself from the rmvQ (line 23) and waking up the next element
in the rmvQ (line 26).

\begin{figure}
\lstinputlisting [language = C, multicols=2] {source_code/fifoq.c}
\vspace{-5pt}
\caption{Implementation of remove method of FIFOBBQ.}
\label{fig:exp:fifo}
\vspace{-10pt}
\end{figure}

\paragraph{IPC}
We have implemented and verified a single-copy
inter-process communication (IPC) protocol using condition variables
and the FIFO Blocking Bounded Queue.
Additionally, we have verified a
zero-copy IPC for user programs that is built on top of the
shared memory infrastructure.
\ignore{
\begin{figure}
\lstinputlisting [language = C, multicols=2] {source_code/ipc.c}
\caption{Implementation of Single-Copy IPC}
\label{fig:exp:ipc}
\end{figure}

\vspace{-3pt}
\paragraph{Trap handler}
\label{sec:base:trapm}

\ignore{
\begin{figure}
\begin{center}
\includegraphics[scale=0.33]{figs/pagefault2}	
\caption{Call graph of page fault handler}
\label{fig:base:pagefault}
\end{center}
\vspace*{-14pt}
\end{figure}
}
specifies the behaviors of exception handlers
(\eg, page fault handler), interrupt handlers (\eg, serial port),
and system calls (\eg. IPC). It also specifies the behavior of ring switch and saving/restoring trap frames.
}
\ignore{
For example, a page fault at the user level traps into the kernel, saves the
current trap frame (done by both the hardware and software), and then jumps
to the page fault handler.
The page fault handler reserves a page for \texttt{PFLA} (if necessary)
and returns to the user level by restoring the saved user context.
The verification of the page fault handler depends on objects
introduced at various abstraction levels. 
\ronghui{So what? Everybody knows what is a page fault?}
% (see Fig.~\ref{fig:base:pagefault}).
}

\ignore{
Therefore, the behavior of the page fault handler is interpreted by
the concrete first-class code pointer until all the dependent layer
objects are introduced.  Then the handler code is verified and
the behavior is interpreted using its abstract atomic specification.
}


To further simplify the reasoning about user code, we have implemented and
verified the user level system call libraries directly in the user space.
Since our machine semantics models hardware behaviors
like paging and ring switch, the specifications of user system call
libraries closely corresponds to the real execution model in the actual
hardware. With this \emph{atomic} system call semantics in the user level,
the user code can be proved much more easily.

The top layer of \mCTOSbase{} offers a set of system calls for user programs, 
such as IPC calls and calls to trigger the scheduler.
The specifications of system calls are defined and verified at the user level
by wrapping the system call handler's specification
with the ring switch specification.
We can reason about user-level programs directly with these atomic system calls' specifications.


\ignore{
\newman{Already in Section 2}
\subsection{Other properties}
Except for the above features, we also prove the following properties of \mCTOSbase{}:
\begin{itemize}
\item Since the contextual refinement is termination sensitive, we prove the total
correctness of our kernel, meaning that our kernel will not get stuck
and all system calls for user program will terminate.
\item There is no integer overflow inside the kernel.
\item There is not stack overflow inside the kernel. (Statically checked by the analysis tool,
refer to Quentin's work)
\item All the pointers stored in the kernel objects are valid.
\end{itemize}
}

%\section{Layered Concurrent Programming}
\label{sec:prog}

\ignore{
\paragraph{outline} 1) C-level vs assembly-level programming
2) layer refinement for composing multiple events?
3) spinlocks (ticket lock, MCS lock)
4)thread management
5) queuing locks
6) starvation-free condition variable
}


In this section,\ignore{we instantiate our framework with a
C-like language ClightX and an assembly level language LAsm,
and we introduce the concurrent layer interfaces for
a single active CPU.
To demonstrate the modularity and practicality of our framework
for concurrent program reasoning,}
we show how to verify spinlock implementations, shared queues protected by spinlocks,
thread scheduling primitives, and a queuing lock,
using concurrent abstraction layers.
All layers are built upon the single-core machine $\LAsm{(L(c,\oracle))}$.


\paragraph{Scheduler primitives and multi-thread composition}
It is common for multi-threaded programs to perform explicit synchronization
of threads by calling scheduler primitives, e.g., $\yield$, $\sleep$, and $\wakeup$.
Verification of such concurrent programs is extremely challenging.
Previous work either directly models the abstract scheduler primitives at a high level
(e.g., simple switch in the thread ID~\cite{xu16}),
or verifies the scheduler functions with a low level small-step semantics
(e.g., saving and restoring the register set) \cite{dscal15}.
Both approaches have severe limitations. The first approach provides no formal connection
between the specification and underlying C and assembly implementation of those
scheduler functions. In the second approach, the specifications are too low level to
be useful in verifying the actual programs calling these primitives. Furthermore,
some portion of scheduler functionality is implemented in assembly (e.g., context switch).
Thus the semantics does not satisfy the C calling convention, meaning that it 
is impossible to reason about programs calling these assembly functions at the C level.

In this example, we build concurrent layers for the assembly part
of a scheduler in a small-step manner, but later lift the specifications
to big-step ones satisfying the C calling conventions. 
This allows us to perform \emph{thread-local} reasoning of programs
at the C level, where the proved properties of each thread can be linked formally
to obtain a global claim about the whole set of threads running on a processor.
To the best of our knowledge, this is the first attempt to support modular verification
of multi-threaded applications with explicit synchronization using verified
low level scheduler primitives.

\vspace{3pt}
\noindent\textbf{Step 1 (assembly-level specification $L_5$)} 
Figure~\ref{fig:exp:sched} shows the implementation
of selected scheduling functions.
The $\yield$ function first polls a pending
thread from the per-CPU pending queue
(\ie, $\comm{pendq}(c)$),
then selects the next running thread
from the CPU-local ready queue
(\ie, $\comm{rdq}(c)$),
and finally switches the kernel context.
The $\sleep$ function is similar: it
puts the current running thread
on the sleeping queue (\ie, $\comm{slpq}(q)$) and
then context switches.
The $\wakeup$ function moves the head of the sleeping queue
into the corresponding CPU's pending queue.
The context switch function $\mathsf{cswitch}$
saves the current thread's kernel context (including the 
\emph{stack pointer}),
and loads the context of the target thread.
This function $\mathsf{cswitch}$ can only be implemented at the assembly level,
and its specification does not satisfy the C calling convention.
To verify these low level scheduler functions, we introduce an
assembly-style concurrent layer interface $L_5$.
\begin{figure}
\lstinputlisting [language = C, multicols=2] {source_code/scheduling.c}
\vspace{-10pt}
\caption{Pseudocode of scheduling implementation.}
\label{fig:exp:sched}
\vspace{-10pt}
\end{figure}


At layer $L_5$, we introduce three new events
$c.\yield$, $c.\sleep(i)$, and $c.\wakeup(i)$,
which map to $L_4$'s events
$c.\deq(\comm{pendq}(c))$,
$c.\enq(\comm{slpq}(i))$,
and $c.\enq(\comm{pendq}(c))$, respectively.
Since the mapping is one-to-one, the transformation functions for
the log ($f_{l4}$) and environment context ($f_{\oracle4}$)
can be defined trivially.
The specifications of $\yield$ and $\sleep$
are defined as follows:
\begin{small}
\begin{mathpar}
\inferrule{
l_0 = l \cons \oracle (l) \\
\replay_{\comm{sched}} (l_0, c) = (tid, tdqp, tcbp) \\
a' = a\set{\comm{kctxt}(tid) : \regs[\comm{ra},
\comm{ebp}, \comm{ebx}, \comm{esi}, \comm{edi}, \comm{esp}]} \\
l' = l_0 \cons c.\yield/c.\sleep(i) \\
\replay_{\comm{sched}} (l', c) = (rtid, \any, \any) \\
\regs' = \regs\leftarrow a.\comm{kctxt}(rtid)  
}{
 \oracle, c\vdash_{\comm{Asm}}  \spec_{\yield/\sleep}([]/[i],\regs, m, a, l,\any, \regs',  m, a', l')
}
\end{mathpar}
\end{small}%
The abstract states $\comm{tid}$,
$\comm{tdqp}$, and $\comm{tcbp}$
are hidden at $L_5$, since they can always be reconstructed
by the replay function $\replay_{\comm{sched}}$ given the current log.
The replay function $\replay_{\comm{sched}}$
is defined inductively.
Here we only present the case for the $\sleep$ event:
\begin{small}
\begin{mathpar}
\inferrule{
\replay_{\comm{sched}} (l, c) = (tid, tdqp, tcbp) \\
\comm{tdqp} (\comm{rdq}(c)) = r\cons q \\
\comm{tdqp} (\comm{slpq}(i)) = q' \\
\comm{tdqp}'  = \comm{tdqp}
\set{\comm{rdq}(c): q}
\set{\comm{slpq}(i):q'\cons tid}\\
\comm{tcbp}'  = \comm{tcbp}
\set{tid: \comm{SLEEP}}
\set{r:\comm{RUN}}
}{
\replay_{\comm{sched}} (l \cons c.\sleep (i), c) = (r, tdqp', tcbp') 
}
\end{mathpar}
\end{small}%
\noindent\textbf{Step 2 (multi-threaded machine $\TAsm$)} 
The  layer interface $L_5$ is defined for the whole
set of threads for CPU $c$ and does not support thread-local reasoning.
Ideally, we would like to reason about each thread running on the CPU 
locally, and later formally combine the reasoning to obtain a global
property for the full set of threads.
To support this, we need a machine model that gives semantics to
a partially-composed set of threads.

Let $T_c$ denote the whole set of threads running over CPU $c$.
From $\LAsm(L_5(c,\oracle))$, we construct a new 
multi-threaded \emph{partial} machine $\TAsm$,
and instantiate it with a layer interface $L_6$.
The result machine 
$\TAsm(L_6(c), A)\langle \oracle \rangle$ is 
parameterized over an active thread set $A \subseteq T_c$,
and can be instantiated into a regular machine for
an environment context $\oracle$.
The machine state of $\TAsm$ is defined as $s:=(\tid, f_\regs, m, f_a, l)$.
Here, $\tid$ is the current thread ID, and
$f_\regs$ and $f_a$ are partial functions
from IDs of threads in $A$ to a register set $\regs$
and a per-thread abstract state $a$, respectively (\ie,
we divide the register set and the abstract state
into separate ones for each thread).
The relation between abstract states
can be easily defined, but the relation of the register set between $L_5$ and $L_6$ is non-trivial.
For the currently-running thread, its local register set ($s_6.f_\regs(s_6.\tid)$) is equal
to the register set of $L_5$ ($s_5.\regs$).
For other threads of $L_6$, the local register set is equal
to the corresponding saved kernel context of $s_5$ for the registers
$[\comm{ra}, \comm{ebp}, \comm{ebx}, \comm{esi}, \comm{edi}, \comm{esp}]$.
In other words, the register set is decomposed in a way such that the $\regs$ of the
currently-running thread is equal to the machine's register set, while the $\regs$ of
other threads is equal to the corresponding kernel context saved by context switch.

\ignore{
However, $m$ is using
the CompCert memory model \cite{leroy08},
which is not compositional \ronghui{reference here?Tahina?}.
We introduce a new \emph{algebraic memory
model} to make the CompCert memory model
modular (\cf Sec.~\ref{sec:comp} for more details).
Two machines are composable only if they have the same
logical log $l$.
}

For instructions and primitives that do not change the current thread
ID, their semantics are lifted from $L_5$ to $L_6$
by instead operating on the state $(f_\regs(\tid), m, f_a(\tid), l)$.
The scheduling functions (which may change the thread ID to a thread outside of $A$)
are part of the definition of $\TAsm$
and may trigger environment steps by generating events
that will switch control to a thread outside of $A$.
For instance,
the $\yield$ operation of $\TAsm$
will generate a $c.\yield$ event that will switch control to the next thread
and set the machine's $\tid$ accordingly.
If the new thread is in $A$,
then the $\TAsm$ machine will simply keep running local steps of the new thread.
If the new thread is outside of $A$,
then the machine will take environment steps
until the log indicates that control switches back to  $A$.

When $A$ is the entire thread set $T_c$ of core $c$,
it will never switch to environment threads and
we have the following refinement:
\begin{lemma}{\small
$\forall \oracle\ ,~
\LAsm(L_5(c,\oracle)) 
\refines
\TAsm(L_6(c), T_c)\langle \oracle \rangle$}%
\label{thread_composition}
\end{lemma}

However,
the partial machine $\TAsm$ can also be decomposed using the linking operator $\Join$,
which enables the thread-local reasoning:
\begin{lemma}
{\small
$
\TAsm(L_6(c), T_c)
\Refrel_\id
\bigJoin_{t \in T_c}
\TAsm(L_6(c), \{t\})
$}
\label{thread_compose}
\end{lemma}

%-----------------------------
\ignore{
For scheduling functions (which may change the thread ID to one for a thread not in $A$),
it may trigger the environment step by querying the environment
context $\oracle_T$, which capture the behavior of \emph{inactive} threads (\ie, the threads in $T_c - A$).

We define the specifications of scheduling functions using $\oracle_T$.
For example, the $\yield$ primitive at $L_6$ is specified as:
\begin{small}
\begin{mathpar}
\inferrule{
l_0 = l \cons \oracle (l) \\
(l', \tid') =  \comm{yield\_back}
(A, \oracle_T, l_0 \cons c.\yield) \\
f_\regs' = f_\regs\set{\tid': 
\text{undefined value except for }[\comm{ra},
\comm{ebp}, \comm{ebx}, \comm{esi}, \comm{edi}, \comm{esp}]}
}{
 A, \oracle_T, \oracle, c\vdash_{\comm{Asm}}  \spec_{\yield'}([],\tid,f_\regs, m, f_a, l,\any, \tid', f_\regs',  m, f_a, l')
}
\end{mathpar}
\end{small}%

Note that $\yield$ changes the $\tid$ to $\tid'$.
After that, the execution operates on
the state $f_\regs'(\tid')$ and $f_a(\tid')$.
When $\tid'$ is different from $\tid$, this simulates the behavior of context switch
within the active thread set $A$. 
If the active thread set $A$ is $T_c$, we have:
}

\ignore{
If the active thread set $A$ is the whole
thread set $T_c$, we have:

Then, we can prove the simulation
relation:
\begin{lemma}[$\LAsm$ refines $\TAsm$]
\label{thread_composition}
{\small
\[
\forall \oracle,~
\LAsm(L_5(c,\oracle)) 
\refines
\TAsm(L_6(c,\oracle, T_c, \any))
\]}%
\end{lemma}



\begin{lemma}[CPU-local machine refines multi-thread machine]
{\small
\[
\forall M~ \oracle, 
\sem{ }{M}L_5(c,\oracle) \Refrel_{R_5} 
\sem{}{M}L_6(T_c, \oracle)
\]}
\label{whole_thread_composition}
\end{lemma}}

\ignore{
\textbf{Proof sketch}:
For whole thread set,
the environment context only carries
the information of the context CPUs.
Thus, $\oracle_{T_c} = \oracle$.
For the simulation relation $R_5(s_5,s_6)$, 
the only non-trivial part is the relation
for register set function $s_6.\regs p$.
For the current thread,
$s_6.\regs p(s_6.\tid)= s_5.\regs$.
For other threads,
the register set of $s_6$ is equal
to the kernel context of $s_5$
for the register list
$[\comm{ra},
\comm{ebp}, \comm{ebx}, \comm{esi}, \comm{edi}, \comm{esp}]$.
\ignore{
which is defined as:
1) ;
and 2)$\forall i \neq s_6.\tid,
s_6.\regs p(i)= s_5.a.\comm{kctxt}$.}
}
%-----------------------------

\noindent\textbf{Step 3 (thread-local layer interface $L_7$)} 
Finally, we introduce a per-thread layer interface $L_7$, which is parameterized
over a single active thread $t\in T_c$.
Thus, $\yield$ always yields back to itself,
and there is no longer any ``internal'' low level context switch among active threads.
Thus, we obtain the following big-step specification of $\yield$ that actually
satisfies the C calling convention:
\begin{small}
\begin{mathpar}
\inferrule{
l_0 = l \cons \oracle (l)\\
(l', \tid) =  \comm{yield\_back}
(\tid, \oracle, l_0 \cons c.\yield)
}{
 \tid, \oracle, c, \oracle \vdash  \spec_{\cyield}([],\regs, m, a, l,\any, \regs,  m, a, l')
}
\end{mathpar}
\end{small}%
Here, the auxiliary function $\comm{yield\_back}$ 
specifies the behavior of repeatedly querying the environment context
$\oracle$ until the control flow yields back to the thread of interest $\tid$.
It appends all the events triggered by the inactive threads to the log.
To prove termination-sensitive contextual refinement,
we prove that $\comm{yield\_back}$ actually terminates,
by showing that the scheduler is \emph{fair} and every running
thread gives up the CPU within a finite number of steps.

On top of this thread-local machine $\LAsm(L_7(c, t, \oracle))$,
we can reason about a thread's program 
locally at the C level by building per-thread layers.
We show that $L_6$ for a single thread is refined $L_7$:
{\small
\[ \TAsm(L_6(c), \{t\})\langle \oracle \rangle \Refrel \LAsm(L_7(c, t, \oracle)) \]
}%

% ------------------------------------------------------------------
% Let's not talk about linking partial machines here
\ignore{
By Lemma~\ref{lemma:mono},
per-thread layers can be composed using
``$\Join$", and propagated down to the $L_6$ machine.
Once the layers are composed for the full thread set $T_c$, we can apply Lemma~\ref{thread_composition}
to transfer the guarantee down to $L_5$
and link with per-CPU layers.
}
% ------------------------------------------------------------------

\paragraph{Queuing Lock} is an algorithm whereby waiting threads are put to sleep,
so that busy waiting is avoided.
Verification of this complex locking algorithm is particularly
challenging since its C implementation utilizes both
spinlocks and low level scheduler primitives ($\sleep$ and $\wakeup$).
In our framework, a queuing lock implementation is verified
by building multiple layer interfaces on top of $L_7$.

\ignore{
We first introduce two atomic operations (using $\comm{CAS}$)
that set and clear the busy bit of the lock.
Then we prove that the implementations
of the lock and unlock operations
satisfy their atomic specifications
(which generate single $\comm{acq\_q}$ and $\comm{rel\_q}$ events, respectively).
This is achieved by proving a simulation between layer interfaces together
with starvation-freedom, using a strategy similar to verification of
the ticket lock implementation.
}

\ignore{

 is a general
lock algorithm that 
allows sleeping for the lock, instead of busy waiting.
Since queuing lock
relies on the spinlocks and thread scheduling
and is implemented at C-level
(\cf Fig.~\ref{fig:exp:queue_lock}),
it is hard to reason about
and has never been verified in previous works.

In our framework, queuing lock can be easily verified
based on $L_7$.
We first introduce two atomic operations
to set the busy bit if the lock is free (\ie,
$\comm{CAS\_qlock}$, implemented
using $\comm{CAS}$),
and to clear the busy bit (\ie, $\comm{clear\_qlock}$).
Then, we prove the implementation 
of lock and unlock operations
satisfy the atomic specifications
(\ie, $\comm{acq\_q}$ and $\comm{rel\_q}$ events)
by showing the simulation relation
and the starvation-freedom.
The verification details are similar to the ticket lock.

\begin{figure}
\lstinputlisting [language = C, multicols=2] {source_code/queue_lock.c}
\vspace{-5pt}
\caption{Pseudocode of queuing lock implementation.}
\label{fig:exp:queue_lock}
\vspace{-10pt}
\end{figure}

}
      % Layered Concurrent Programming (3-3.5 


\section{Certified Compilation and Linking}
\label{sec:comp}

In our approach, interaction between threads happens only through the 
abstract log, and concrete memory is always thread-local.
In Sec.~\ref{sec:prog} we already saw how this yields a big
improvement over the previous state of the art, because it lets us
give a \emph{C level} specification to primitives that handle
communication between threads. In particular, we can write the
implementation of the $\yield$ primitive in assembly, and because the
behavior is deterministic (given the log in the abstract state) we can
write a sequential-style specification in which $\yield$ immediately
returns with the abstract state suitably modified.
In order to take advantage of this, we also need a C compiler which
can compile the programs that call $\yield$. 
In this section we show how to adapt Gu et al.'s CompCertX verified
separate compiler \cite[\S 6]{dscal15} for this task.

In fact, most of the compiler can be reused as is. This is because we
do not need to consider parallel composition of thread behaviors
\emph{at the C level}. As the source language, we reuse Gu et al.'s
ClightX \cite[\S 4]{dscal15}, which is the CompCert
Clight subset of C parameterized with layer interfaces.
Each program is compiled into a
target program in the per-thread TAsm language, and we prove that the
compiled code refines its source. Then our concurrent linking
theorems (Lemmas~\ref{lemma:mono},~\ref{thread_composition}, and~\ref{thread_compose})
state that the individual threads compose into a
single, concurrent TAsm program. %, and that theorem can be stated entirely in terms
% of assembly code.

The final thread composition is subtle even at the assembly level,
because each thread's stack and thread-local memory must be combined
into a single global memory state in the CompCert memory model. To
prove that this is possible, we have to maintain a detailed invariant
about how the memory is partitioned, and to extend the abstract state
with extra information to make block allocations deterministic.

%%%%%%%%%%%%%%

% These paragraphs are Tahina's previous section introduction, can 
% reuse as needed.
\ignore{
If the specification of the primitive being implemented already
involves any need for communication with other threads, then such need
is recorded as events and operations on the abstract state of the
overlay interface, which the implementation will need to refine. We
saw in Section~\ref{sec:prog} that, if the primitive is implemented in
assembly, then it is enough to prove refinement between this primitive
and its assembly implementation calling primitives of the underlay
interface where communication with other threads is also recorded as
events and operations on the abstract state of the underlay interface.

In this section, we argue that this principle is still valid if the
primitive is implemented in C rather than assembly. Indeed, we reuse
Gu et al.'s ClightX \cite[\S 4]{dscal15}, the formal semantics of the
CompCert Clight subset of C parameterized with layer interfaces, in
the same way as for sequential C code, with abstract state and
primitives also serving to model concurrency and thread
communication. Then, similarly to the sequential setting by Gu et al.,
we prove that, on the one hand, each C function refines the primitive
it claims to be implementing. Then, on the other hand, we argue that
Gu et al.'s CompCertX verified separate compiler \cite[\S 6]{dscal15}
can be reused in the same way here, to prove that the compiled
assembly code refines its C source. Thus, by transitivity, the
compiled assembly code will refine the primitive its source C code
claims to implement.

This way, just like an assembly thread, we can consider the behavior
of a C thread as if it were running standalone sequentially, by
abstracting other threads away through the abstract state and
primitives (logs, etc.): the $\yield$ primitive can be called by a
C function, and it is \emph{deterministic} (given the log in the
abstract state) and returns with the abstract state suitably modified
by other threads (thus allowing communication between threads.)

However, contrary to assembly code, we do not need to consider any
parallel composition of thread behaviors \emph{at the C level}, thanks
to verified separate compilation and linking, since it is enough to
link the behaviors of assembly threads for the purposes of refining
our concurrent abstraction layers.
}

\paragraph{Verified separate compilation and linking with CompCertX} 
Replacing Gu et al.'s LAsm target with our per-thread TAsm assembly
language, we first recall Gu et al.'s CompCertX correctness statement
pictured in Fig.~\ref{fig:compcertx}: starting from a memory state $m$
(provided by the caller of the primitive being implemented), if the
ClightX function $f$ runs on top of a layer $L$ and produces a
concrete memory state $m'$, then the compiled function $f$ in TAsm
runs and produces a memory state $m''$, introducing some \emph{memory
  injection} $j$ between $m'$ and $m''$ (written
$\inject{j}{m'}{m''}$). This memory injection is a memory
transformation \cite[\S 5.4]{leroy08} due to the different handling of
stack frames between ClightX and TAsm.


\begin{figure}
\[
\xymatrix@R=3pt@C+=3pt{
& v, m', a'
\ar@<-3ex>@{_{(}-->}[dd]^j
\ar@{_{(}-->}[dd]^j
\ar@<3ex>@{==}[dd]
\\
\genfrac{}{}{0pt}{0}{l}{\rho}, m, a
\ar[ur]^{\llbracket f \rrbracket_{\text{ClightX}}(L)}
\ar@{-->}[dr]^{ \llbracket f \rrbracket_{\text{TAsm}}(L)} \\
l \approx m(\rho(\textsf{ESP}))
& \rho', m'', a'
}
\]
\vspace{-10pt}
\caption{CompCertX per-thread correctness statement} \label{fig:compcertx}
\vspace{-10pt}
\end{figure}

Then, to implement the primitives of an overlay interface $L_2$ on top
of an underlay interface $L_1$ by some ClightX module $M_C$ for some
primitives of $L_2$ and some TAsm module $M_{\text{Asm}}$ for other
primitives of $L_2$, for any fixed thread $t$ and environment context
$\mathcal E$, we follow Gu et al., first splitting $L_2$ into C-style
and assembly-style primitives by writing $L_2 = L_{2, \text{C}} \oplus
L_{2, \text{Asm}}$ , then implementing $L_{2, \text{Asm}}$ with TAsm
code $M_{\text{Asm}}$, and $L_{2, \text{C}}$ with ClightX code
$M_{\text{C}}$, so as to prove per-thread refinement at the assembly
level using forward downward simulations as described in \cite[\S
  3.3]{dscal15} and Definition~\ref{def:popl15-layers}, $L_{2,
  \text{Asm}} \leqslant_R \llbracket M_{\text{Asm}} \rrbracket L_1$
(and thus $L_{1} \vdash_R M_{\text{Asm}} : L_{2, \text{Asm}}$), and at
the C level, $L_{2, \text{C}} \leqslant_R \llbracket M_{\text{C}}
\rrbracket L_1$. Then, we compile $M_{\text{C}}$ into TAsm using
CompCertX, thus obtaining $\llbracket M_{\text{C}} \rrbracket L_1
\leqslant_{j} \llbracket \text{CompCertX}(M_{\text{C}}) \rrbracket
L_1$ for some memory injection $j$, and thus by transitivity $L_{2,
  \text{C}} \leqslant_{j \circ R} \llbracket
\text{CompCertX}(M_{\text{C}}) \rrbracket L_1$, which can be rewritten
into $L_{1} \vdash_R \text{CompCertX}(M_{\text{C}}) : L_{2, \text{C}}$
since $j \circ R = R$. Finally, by horizontal composition, we link the
obtained assembly codes to obtain per-thread layer refinement, for any
fixed thread $t$ and environment context $\mathcal E$: $L_1 \vdash_R
M_{\text{Asm}} \oplus \text{CompCertX}(M_{\text{C}}) : L_2$.

\paragraph{Thread-safety of CompCertX: parallel composition and concrete memory states}

We then have to show that the output of CompCertX is compatible with
parallel composition of threads at the TAsm level, and in particular
that the per-thread specifications of concurrent primitives such as
$\yield$ are compatible with verified compilation a la CompCert.  This
might sound trivial: as we presented it so far, $\yield$ only modifies
the abstract state, which should not interact with the compiled
assembly code. But unfortunately, the exact semantics of primitive
calls actually also have to modify the concrete memory state. This is because
of a small snag which we glossed over until now: stack frames.

\ignore{
We argue that Gu et al.'s CompCertX is sound to be used for per-thread
verified separation \vilhelm{what's ``verified separation''?} and linking, in a way compatible with parallel
composition of threads at the TAsm level. The key reason is due to the
fact that the per-thread specifications of concurrent primitives such
as $\yield$ are in fact compatible with verified compilation a la
CompCert, which we are explaining in this paragraph.
}

Modeling the $\yield$ primitives when performing parallel
composition at the assembly level (from TAsm to LAsm, Lemma~\ref{thread_composition}) requires
maintaining an invariant on the concrete global memory state $m$ of
the LAsm machine and the per-thread concrete memory state $m_i$ of
each thread $i$'s TAsm machine. The concrete memory state of a thread
only contains thread-private memory, in particular its stack. Whenever
a function is called, a fresh memory block has to be allocated\footnote{We use the CompCert memory model \cite{leroy08}(notations in Fig.~\ref{fig:mem}), where a memory state consists in finitely many blocks within which finitely many memory locations can be accessed in such a way that pointer arithmetics can be performed only within one memory block at a time.} in the
concrete memory for its stack frame. This means that, at the TAsm
level, a function called within a thread will allocate its stack frame
into the thread-private memory state, and conversely, a thread is
never aware of any newer memory blocks allocated by other
threads. However, at the LAsm level, all stack frames have to be allocated
in the global memory regardless of which thread they belong to;
thus, in the TAsm to LAsm parallel composition proof, we need to account
for all such stack frames.

One possible way could be to consider maintaining memory injections
from each TAsm per-thread memory state to the LAsm global memory state
(so that per-thread $\yield$ would not modify the thread-private
memory at all); but proofs involving memory injections are notoriously
hard since they involve transformations of pointer values.

Thus, we instead chose an easier solution \cite[\S 5.2]{leroy08}. Contrary to memory injections,
memory extensions (written $\extends{m_1}{m_2}$, following the
notation in \cite[\S 5.2]{leroy08}) are memory transformations that do
not require any transformation on the values
stored in memory states. The main issue is that if there is a memory
extension between $m_i$ and $m$, they need to have the same number of
memory blocks allocated, even though the actual permissions of memory
locations within those memory blocks may differ (more permissions in
$m$ than in $m_i$) \cite{compcert-mem-v2}. Thus, to make our parallel composition proof go
through, we need to modify the semantics of per-thread $\yield$ so
that it allocates enough empty memory blocks (within which no memory
locations have any permissions) corresponding to the stack frames
allocated by other threads. So, for each thread $i$, we only need to
maintain memory extension between $\liftnextblock{m_i}{\nextblock{m} -
\nextblock{m_i}}$ (the thread-local memory $m_i$ adjusted by
allocating enough empty blocks to match the number of blocks in $m$)\footnote{for any memory state $m'$,
$\nextblock{m'}$ is the total number of blocks allocated in $m'$, and
$\liftnextblock{m'}{n}$ allocates $n$ empty blocks in $m'$},
and the global memory state $m$.

\ignore{
\tahina{Here, introduce the new per-thread semantics of $\yield$ with
  the accurate handling of $m$ with $\liftnextblockOP$. --- More
  generally, can we try to describe the correctness proof of parallel
  composition (TAsm to LAsm) in Sec. \ref{sec:layer} and/or
  \ref{sec:prog}, first \emph{pretending} that $\yield$ would not
  change the thread-private concrete memory state (and so, that no
  further memory blocks are ever allocated either in thread-privates
  or in the global memory), then here in this section, ``refine'' our
  explanation by showing how that correctness proof changes
  wrt. conveying the $\nextblockOP$ into the log events?}
}

More formally, in Fig.~\ref{fig:algmem}, we introduce a notion of \emph{algebraic memory model}
to model the relationship $\disjointunion{m_1}{m_2}{m}$ between the private memory states $m_1, m_2$ of two disjoint
thread sets and the global memory state $m$ after parallel composition.
\begin{definition}[Algebraic memory model] \label{def:algmem}
A memory model based on the CompCert memory model \cite{leroy08} is
said to be \emph{algebraic} if, and only if, there is a \emph{disjoint
  union} predicate $\disjointunion{m_1}{m_2}{m}$ satisfying
the axioms of Fig.~\ref{fig:algmem}.
\end{definition}

\begin{figure}
\begin{small}
\begin{tabular}{ll}
$\load{m}{\ell} = \some{x}$ & reading from location $\ell$ in memory $m$ succeeds and value $x$ is read \\
$\store{m}{\ell}{v} = \some{m'}$ & storing value $v$ to location $\ell$ in memory $m$ succeeds and yields memory $m'$ \\
$\alloc{m}{l}{h}$ & memory state obtained by allocating a new memory block in $m$ with permissions for offsets from $l$ to $h$ within the new block \\
$\nextblock{m}$ & number of memory blocks already allocated in $m$ \\
$\liftnextblock{m}{n}$ & memory state obtained by allocating $n$ empty memory blocks (satisfies $\liftnextblock{m}{n+1} = \liftnextblock{\alloc{m}{0}{0}}{n}$ and $\liftnextblock{m}{0} = m$) \\
$\extends{m}{m'}$ & memory extension between $m$ and $m'$ \cite[\S 5.2]{leroy08} \\
$\inject{j}{m}{m'}$ & memory injection between $m$ and $m'$ \cite[\S 5.4]{leroy08} \\
\end{tabular}
\end{small}
\caption{The CompCert memory model \cite{leroy08}: notations} \label{fig:mem}
\end{figure}

\begin{figure}
\begin{scriptsize}
\[
\inferrule{
  \disjointunion{m_1}{m_2}{m}
}{
  \nextblock{m} = \max(\nextblock{m_1}, \nextblock{m_2})
}(\textsc{Disj-Nb})
~ \inferrule{
  \disjointunion{m_1}{m_2}{m}
}{
  \disjointunion{m_2}{m_1}{m}
}(\textsc{Disj-Comm})
\]
\begin{mathpar}
\inferrule{
  \disjointunion{m_1}{m_2}{m} \\
  \nextblock{m_1} \leq \nextblock{m_2}
}{
  \disjointunion{m_1}{\liftnextblock{m_2}{n}}{\liftnextblock{m}{n}}
}(\textsc{Disj-Liftnb-Right})
\\
\inferrule{
  \disjointunion{m_1}{m_2}{m} \\
  \nextblock{m_1} \leq \nextblock{m_2}
}{
  \disjointunion{\liftnextblock{m_1}{n}}{m_2}{\liftnextblock{m}{n - (\nextblock{m} - \nextblock{m_1})}}
}(\textsc{Disj-Liftnb-Left})
\\
\inferrule{
  \disjointunion{m_1}{m_2}{m} \\
  \nextblock{m_1} \leq \nextblock{m_2} 
}{
  \disjointunion{m_1}{\alloc{m_2}{l}{h}}{\alloc{m}{l}{h}}
}(\textsc{Disj-Alloc})
\end{mathpar}
\[
\inferrule{
  \disjointunion{m_1}{m_2}{m}
}{
  \disjointunion{m_1}{\store{m_2}{\ell}{v}}{\store{m}{\ell}{v}}
}(\textsc{Disj-Store})
\inferrule{
  \disjointunion{m_1}{m_2}{m} \\
  \load{m_2}{\ell} = \some{v}
}{
  \load{m}{\ell} = \some{v}
}(\textsc{Disj-Load})
\]
\end{scriptsize}
\vspace{-10pt}
\caption{Algebraic memory model} \label{fig:algmem}
\vspace{-10pt}
\end{figure}

Rules in Fig.~\ref{fig:algmem} define the notion of algebraic memory
we use to prove the correctness of single-processor parallel
composition: any memory operation in a per-thread memory state shall
reflect in the corresponding global memory state. The condition
$\nextblock{m_1} \leq \nextblock{m_2}$ ensures that a non-empty memory
block can be allocated only in the memory state of the \emph{active}
thread, so that if a thread calls $\yield$, then, once it regains
control, it is guaranteed that all new memory blocks were allocated by
other threads. Similarly to (\textsc{Disj-Store}), $\disjointunionOP$
is compatible with $\mathsf{free}$, which only clears the permissions
of memory locations without changing the number of memory blocks.

Rule \textsc{Disj-Liftnb-Left} is critical to our parallel composition
proof, since it allows to maintain the disjoint union invariant when a
thread yields and gets back control, allocating empty memory blocks in
its private memory state $m_1$ to account for blocks allocated in $m$
by other threads.

Based on our disjoint union for two memory states, we then use rule
\textsc{Disj-Liftnb-Right} to generalize to $N$ threads by saying that
$m$ is a disjoint union of the private memory states $m_1, \dots, m_N$
of $N$ threads (on a single processor) if, and only if, there exists a
memory state $m'$ such that $m'$ is a disjoint union of $m_1, \dots,
m_{N-1}$ and $\disjointunion{m_N}{m'}{m}$ holds.

\begin{lemma}
Let $m_1, m_2$ two partial memory states containing the thread-private memories of two disjoint thread sets. Then, we say that global memory state $m$ is \emph{a disjoint union} of $m_1$ and $m_2$ (written $\disjointunion{m_1}{m_2}{m}$) if, and only if, all the following conditions hold:
$\mathsf{nb}(m) = \max(\mathsf{nb}(m_1), \mathsf{nb}(m_2))$; $\forall i$, memory extension from $\mathsf{liftnb}(m_i, \mathsf{nb}(m) - \mathsf{nb}(m_i))$ to $m$; and no valid common memory locations in both $m_1$ and $m_2$; then we satisfy the algebraic memory model axioms of Fig.~\ref{fig:algmem}.
\end{lemma}

Remember that, similarly to CompCert, CompCertX can only compile
ClightX code that calls primitives that are deterministic and are
preserved by memory transformations (extensions and injections)
introduced by compilation passes. To maintain per-thread determinism
of $\yield$ while the exact behaviors of other threads is not
known, it is necessary to record the number of missing blocks
allocated by other threads in each per-thread log in its abstract
state, so that per-thread $\yield$ can remain
deterministic. Finally, having $\yield$ allocate a number of
blocks only determined by the per-thread log instead of any concrete
memory state makes it compatible with memory transformations
introduced by CompCertX. Thus, $\yield$ is valid to be used in
ClightX code being compiled by CompCertX.
      % Certified Compilation and Linking (1-1.5 pages)


\paragraph{Verification Effort and Lessons Learned}
Our team completed verification of the \cCTOS{} kernel in about 2 person years.
The layered approach is key to the scalability and feasibility of such a
large-scale verification effort.
One benefit of our approach is that
concrete and highly optimized (and thus complex) implementations can be abstracted
into much simpler logical specifications that are easier to reason about,
\eg, abstracting a queue implemented as a doubly-linked list into a simple logical list,
abstracting two-level page tables into a logical map from virtual addresses
to physical addresses with permissions, {\it etc}.
Besides this straightforward benefit, the layered approach shines in many other
aspects of complex system verification.

Layering allows us to perform incremental refinement of machine
  models. Real world CPUs are far from ideal for reasoning
purposes, so we abstract the realistic
machine model into a simpler one.
As indicated in Figure~\ref{fig:layer_diagram}, the per-CPU layers are built on top
of an abstract machine model that supports CPU-local reasoning. Through
multiple contextual refinements, we have proved that the underlying
nondeterministic hardware machine model refines the abstract model. Furthermore, 
because each layer in our framework
is an abstract machine, we can do this not just at
the bottom, but at any point within the verification stack. For example, 
during the \cCTOS{} verification, we first abstract the two-level page table structure
in memory into a logical map from virtual addresses to physical addresses
with permissions. After doing this, we \emph{then} abstract the low level machine 
model that implements page-based virtual memory as described in the hardware 
manual into a simpler model using our abstract logical map.

Layering eases code verification in the concurrent setting.
It allows us to separate code verification from logical reasoning.
Concurrent program verification can be treated as a process of building
verified atomic objects. Each method of an atomic object is implemented using
locks and other atomic objects introduced at lower layers.
When we verify the function body of such a method, we first simply treat it as 
if it were sequential; this results in a non-atomic specification that allows
multiple events to occur. Then, through a separate contextual refinement
that does not involve any code verification, we abstract the environment
context to obtain a new, atomic specification generating only a single event.
In this way, we manage to cleanly separate concurrency reasoning 
(\eg, interleaved executions) from code verification.

\ignore{
{\it Layering facilitates the invariant preservation proof.}
Invariant preservation proofs represent one of the most expensive
components of large-scale verification projects \cite{klein2009sel4}.
They are expensive because they need to be proved not only locally, but
also for the global execution of the whole program. Additionally, many low 
level data manipulation functions temporarily break invariants and reestablish them
later. In our layered approach, every global in-memory data
structure refines one or more isolated abstract logical states in
higher layers. This results in automatic isolation guarantees, despite the
fact that underlying implementations at lower layers still manipulate pointers to
the data structure. Furthermore, each abstraction layer may have a different set
of invariants in our framework, so invariants are introduced incrementally.
For example, consider a queue implemented as a doubly-linked list in
memory. If we directly impose the invariant that the queue is always
well-formed, then we need to prove that no other pointer manipulation in the kernel 
breaks this invariant, while also dealing with the situation where low level 
code temporarily breaks but then reestablishes the invariant. 
Instead, we first abstract
the concrete implementation of the queue into a simple logical list, hiding
concrete pointer manipulation beneath abstract read/write primitives.
Then at higher layers, we gradually introduce and verify the
abstract \texttt{enqueue} and \texttt{dequeue} operations that utilize these
read/write primitives. The well-formedness invariant is only introduced at this 
layer where the \texttt{enqueue} and \texttt{dequeue} operations are atomic and 
all the lower level primitives are already hidden. 
}


\ignore{
\newman{Don't read further. Below are just some random sentences got cut from the OSDI version.}

Furthermore, note that we did not reach the current working solution
in one shot. We first spent about 3 person months developing an
unsuccessful version of the framework for composing multi-threaded
execution on a single CPU.  In that version, thread-local execution
was modeled using a \emph{time stamp} index into a global system
log. We eventually realized that the exact time stamps were too
cumbersome and revealed too much information about the underlying
implementation (\eg, the number of software yields within a function
body), so we spent another month developing a new system that uses
local logs (lists of events) instead.\ignore{ of time stamps, and the
  ability to shuffle and merge the events in the local logs to hide
  unnecessary nondeterminism or implementation details.}  Our initial
multicore machine model also did not work out very well when we
developed the multicore linking framework; we spent 3 person weeks to
improve the initial design through multiple iterations.  The main
challenge was finding the right invariants for the environment
context, such that we could successfully establish starvation-freedom.

\paragraph{Abstraction Layers}
\newman{I am gonna write some summary about the layered approach after I read
the overview section to see how much of the concepts are already covered.}
\begin{itemize}
\item Abstraction of data representation: doubly-linked list --> logical list
\item Stronger invariants
\item Abstraction of primitive specification: hiding log implementation by linking and merging events
\item Refinement of machine models: from a realistic machine model to an ideal machine model that is suitable for our reasoning purpose, e.g., change in machine/memory/interrupt model
\end{itemize}

The verification effort roughly falls into three categories: layer
design with specification and invariants, refinement proofs between
the layers, and verification of C and assembly code with respect to
the specifications. The time needed for each of the categories depends
largely on the layer.  For instance, at the boundary of physical and
virtual memory management (\texttt{MPTIntro}), almost all effort
is in the refinement proof, due to the proof for the refinement between
two completely different memory models. More effort went into the
refinement proof when we introduced the Intel \emph{virtual machine
memory model}, where we proved the refinement between the concrete
four level extended page table structure in memory and the abstract
mapping from the guest addresses to the host addresses.
In contrast, for the layer \texttt{MATOp},
which initializes physical memory allocation,
most of the time was spent on verifying
the non-trivial nested loops present in the C code,
while the refinement proofs were derived automatically. 

The proofs were facilitated by automation tools for C
code, layer design patterns, and tactics libraries developed in
recent years \cite{dscal15}. These tools have greatly
reduced the amount of work needed to verify extensions of the kernel.


Verification not only
should not hinder application of similar performance optimizations,
but instead provide a safety net for more aggressive optimizations, if
it is required for application scenarios of the kernel we have in
mind.
}
