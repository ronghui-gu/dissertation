
\section{Appendix}

\subsection{Failure Sampling Algorithm}
\label{subsec-failure}

The pseudocode of our proposed failure sampling algorithm
is presented in Algorithm~\ref{alg-monteCarlo}.

\begin{algorithm}[h]
\begin{footnotesize}
\KwIn{\ft $G$ and the number of samples $N$}
\KwOut{RG}
\For{$i \leftarrow 1$ to $N$}{
  \ForEach{$event_i \in G$ by reversed breadth-first traversal}{
    \If{$event_i$ is {\em basic event}}{
      $event_i.occurrence \leftarrow 0$ or $1$ based on random flipping
a fair coin}
    \Else{
      $event_i.occurrence \leftarrow 0$\;
      \If{$event_i.InputGate$ is {\em OR}}{
        \ForEach{$c_i \in event_i.ChildList$}{
          \If{$c_i.occurrence$ is $1$}{
            $event_i.occurrence \leftarrow 1$\;
            break\;
          }
        }
      }
      \Else(\tcc*[f]{$event_i.InputGate$ is {\em AND}}){
        \ForEach{$c_i \in event_i.ChildList$}{
          \If{$c_i.occurrence$ is $0$}{
            break\;
          }
        }
        $event_i.occurrence \leftarrow 1$\;
      }
    }
  }
  \If{$root.occurrence$ is $1$}{
    $TmpSet \leftarrow \emptyset$\;
    \ForEach{$event_i \in G$}{
      \If{$event_i.occurrence$ is $1$}{
        $TmpSet.append(event_i)$\;
      }
    }
    RG.append(TmpSet)\;
  }
}
{\bf return} RG\;
\end{footnotesize}
\caption{Failure sampling algorithm}
\label{alg-monteCarlo}
\end{algorithm}


\subsection{Kissner and Song Protocol}
\label{subsec-ks}

In order to compare \pia, we implemented a typical private
set intersection cardinality protocol, 
Kissner and Song (KS)~\cite{kissner05privacy}.

\paragraph{KS protocol}
KS allows a group of $k$ parties with
multisets $S_1,...,S_k$ to compute privately $|\cap_i S_i|$, the number
of elements they have in common,
without learning the specific elements in $\cap_i S_i$.
In addition, KS needs to use a homomorphic encryption
scheme such as the Paillier cryptosystem~\cite{paillier99public}.
At the initial phrase of the protocol, 
the $k$ parties use a homomorphic cryptosystem
to share a secret key $sk$ amongst themselves, while the
corresponding public key $pk$ is known to all parties.
With the above keys in hands, the protocol computes $|\cap_i S_i|$
as follows.  
First, each party $p_i$ encrypts a polynomial $P_i$ whose
roots are the elements of its local input data set $S_i$.
The encrypted polynomials are essentially added together, thus
yielding a polynomial $P$ whose roots are the elements in the
union sets of all the parties.
Each party $p_i$, then, evaluates $P$ on the elements $e_{ij}$ of its
local data set $S_i$, yielding values $v_{ij} = P(e_{ij})$; however,
since $sk$ is shared, no individual party can decrypt the $v_{ij}$.
The $k$ parties securely re-randomize and shuffle the $v_{ij}$ based on
the approach~\cite{neff01verifiable}, such that each party 
learns all the $v_{ij}$
but cannot tell which party it came from.
Finally, The $k$ parties jointly decrypt the $v_{ij}$.  
If there are $n$ elements in
the intersection, this would yield $n \cdot k$ zeros; thus, each party
can compute the final result by dividing the number of 
zeros by $k$.

\paragraph{Implementation of KS.}
Following the above description, we have implemented 
KS for comparing with our system in Java.  
Based on our knowledge, this is the first effort
implementing a practical system based on KS.
In our implementation, we used MD5 for hashing operations,
and use the thep library~\cite{thep} to implement
Paillier cryptosystem.
