\chapter{Problem Statement and Challenges}
\label{chap-prob}

This chapter first clarifies our target 
problem ($\S$\ref{sec-prob}) and then
lists associated technical challenges ($\S$\ref{sec-chal}).

\section{Problem Statement}
\label{sec-prob}

Under the assumption of failure independence,
cloud providers typically leverage redundancy techniques to
reduce the likelihood of failures,
thus enhancing the reliability of their services.

\paragraph{Risk group.}
For a given service or application's redundant system,
we define a {\em risk group}~\cite{kaminow97optical}, \rg, 
of this redundant system as a collection of the underlying components 
whose simultaneous failures would lead to the service outage.%
\footnote{Failure of some component means the component does not function
due to software, network or hardware problems.}
For example suppose a cloud storage service employs
three-way redundancy configuration upon its storage components.
Its purpose is for all \rgs to be of size at least three.
In other words, at least three components must fail simultaneously 
to result in the service outage.
However, if these three higher-level components unexpectedly 
depend on some {\em single} lower-level component,
\eg, an aggregation switch, then this lower-level component
represents a \rg of size one in that its failure
would make the whole service become unavailable.
Heading off {\em unexpected \rgs}, defined as
smaller than expected \rgs, is our goal.

\paragraph{Unexpected \rgs within the clouds.}
Cloud service outages in reality have resulted from
unexpected \rgs due to common dependencies~\cite{potharaju13network}.
Well-known cloud IaaS services, 
such as Amazon EC2 and Microsoft Azure, 
try to utilize redundancy to ensure
service reliability, \eg, by introducing backup 
servers and switches in their data centers~\cite{abu-libdeh10symbiotic, 
chuanxiong08dcell, mysore09portland, 
zhong08replication}.
However, additional redundancy may not mitigate 
the likelihood of failure due to a failed \rgs,
derailing providers' efforts to improve reliability%
~\cite{he13next}.
Amazon, for example, experienced a significant
disruption in the Northern Virginia EC2 data center
due to correlated failures resulting from 
incorrect configuration on a few 
aggregation switches~\cite{EC2Crash11}.
While Amazon repaired the failure after it occurred,
the general problem, \ie, an unexpectedly failed \rg, 
would likely recur in the future.

\paragraph{Unexpected \rgs across multiple cloud providers.}
As the cloud diversifies, some cloud service providers 
no longer depend upon only a single cloud service and
have begun redundantly using other cloud providers
for enhanced reliability~\cite{bessani11depsky}. 
For example, as a well-known cloud storage and computing platform,
iCloud~\cite{iCloud} redundantly builds its service upon 
infrastructure services from both 
Amazon EC2 and Microsoft Azure. 
Zynga~\cite{Zynga}, developer of many Facebook games, 
uses both EC2 and an internal ``cloud'' for redundancy.
Despite the efforts, these providers may be 
unaware and unable to mitigate
failures due to unexpected \rgs shared by distinct cloud providers.
For instance, a recent ferocious lightning storm
in Northern Virginia~\cite{storm12, powerOut} 
took out local primary and backup power supplies,
resulting in unavailability of all the IaaS services in the region. 
%This accident caused the outages of cloud
%services in this area redundantly using these IaaS services. 

\section{Technical Challenges}
\label{sec-chal}

Several technical challenges appear in detailed analysis of
our target problem.  The following list 
focuses on those we view as critical toward 
building a practical solution on our problem.

\paragraph{Challenge~1: Acquiring dependencies.}
In order to discover and eliminate 
unexpected \rgs within a cloud service deployment,
we must be able to acquire the information about relevant components 
and their associated dependencies
supporting this service.
Because such infrastructures underlying the
cloud tend to be very complex in practice,
it is likely to be infeasible for the 
service provider or administrator to populate this information manually.
Therefore, acquiring dependencies 
automatically becomes an important and challenging problem.
Existing efforts towards this target found in monitoring and
diagnosis systems have been limited to networking ignoring
hardware and software level dependencies~\cite{chen08automating}.
Recent reports have shown unsuitability of networking alone,
as failures resulting from software and hardware become
rather commonplace~\cite{Top10-2013}.

\paragraph{Challenge~2: Determining and evaluating \rgs.}
Even with a set of components and their dependencies,
it is still not obvious how best to determine unexpected \rgs and 
evaluate their importance in a useful way.
Within this challenge, there exists the need to construct an useful
dependency graph and instrumenting it with potential failures.
Determining \rgs collected from target services provides 
a difficult challenge due to potentially complex dependencies.
Existing efforts~\cite{bahl07highly, kompella05ip} 
have tried to solve similar problems with
diagnosis analysis; however, these approaches may fail to 
offer accurate results when confronted with complex dependencies%
~\cite{wu12netpilot}.

\com{
Even with a set of components and their dependencies,
it is not obvious how best to determine \rgs
and evaluate their importance in a useful way.
For example, while a reasonable first step is 
simply to rank \rgs by size --- %
number of components that must fail 
simultaneously to create an outage --- %
a more sophisticated ranking might account for differences
in anticipated failure probabilities of different components.
Such an improvement, however, would have the cost
of requiring dependency acquisition mechanisms to obtain
not just an accurate dependency graph,
but also realistic failure probability,
for each component.
Existing efforts~\cite{bahl07highly, kompella05ip}
have tried to locate the root cause for a given
service outage with diagnosis approaches,
but they do so after the fact in response to a failure,
and in many cases cannot produce accurate results
in the face of complex dependencies~\cite{wu12netpilot}.
}

\paragraph{Challenge~3: Privacy Preservation.}
For a cloud service that rents infrastructure
from other different cloud providers,
performing structural auditing is almost impossible in practice.
Since the infrastructure information 
is high secret proprietary information
to every cloud provider, no provider is willing to share it.
Auditing such multi-level cloud services, therefore,
introduces another complex challenge:
performing auditing of the target service
without compromising the privacy of its 
cloud infrastructure providers.
Two possible solutions would be:
1) to introduce a trusted third party,
who collects dependency information from
multiple cloud providers 
and performs auditing for the service;
or 2) using secure multi-party 
computation (SMPC)~\cite{yao82protocols}
to compute the overlap privately,
without exposing any information from which those results are computed.
Unfortunately, in the former case cloud
providers may be hesitant to trust a third-party,
while the latter option has been proven 
to be time-consuming and does not scale well%
~\cite{zhai13auditing, xiao13structural}.

