
\section{Layered Programming in LAsm}
\label{sec:seq:lasm}

In this section, we describe LAsm, the \emph{Layered
  Assembly language}, and the extended machine model which LAsm is based on.

The reason we are interested in assembly code and behavior is
threefold.  First of all, even though we provide ClightX to write most
code, we are still interested in the actual assembly code running on
the actual machine. In Section \ref{sec:seq:comp}, we will provide a
verified compiler to transport all proofs of code written in ClightX
to assembly.

\begin{figure}[t]\centering
\lstinputlisting [language = C, multicols=2] {source_code/seq-cswitch.c}
\caption{Kernel context switch verified in LAsm}
\label{fig:contextswitch}
\hrulefill
\end{figure}

Secondly, there are parts of software that have to be manually written
in assembly for various reasons. For example, the standard
implementation of kernel context switch, shown in
Figure~\ref{fig:contextswitch}, 
modifies the stack pointer register
\textsf{ESP}, which does not satisfy the C calling convention and has
to be verified in assembly.  A linker will be defined in Section
\ref{sec:seq:comp} to link them with compiled C code.

Last but not least, we are interested not only in the behavior of our
code, but also in the behavior of the \emph{context} that will call
functions defined in our code. To be as general as possible, we allow
the context to include all valid assembly code sequences. To this end,
it is necessary to transport per-function refinement proofs to a
whole-machine \emph{contextual refinement} proof.


  \subsection{LAsm and layer interfaces}

We start from the syntax and formal semantics of the 32-bit x86
assembly subset specified in CompCert.
CompCert x86 assembly is
modeled as a state machine with a register set and a memory state. 
The register set
consists of eight 32-bit general-purpose registers and eight XMM registers
designated as scalar double-precision floating-point operands.
The memory state is same as the one in Clight.
In particular, each function executes with its stack frame modeled in its
own memory block, so that the stack is not a contiguous piece of
memory.
Another anomaly regarding function calls in CompCert x86 assembly is that
the return address is stored in pseudo-register $\mathsf{RA}$ instead of
being pushed onto the stack, so
that the callee must allocate its own stack frame and
store the return address 
using the \textsf{Pallocframe} and
\textsf{Pfreeframe} pseudo-instructions.

\begin{figure}[t]\centering
\[
\begin{array}{llll}
\mathit{ri}  & ::= & \mathsf{ESP} \, | \mathsf{EBP} \, & \text{Stack registers} \\
& | & \mathsf{EAX} \, | \mathsf{EBX} \, | \mathsf{ECX} \, | \mathsf{EDX} & \text{Integer registers} \\
\mathit{rf} & ::= & \mathsf{FP0} \, | \mathsf{XMM0} \, | \mathsf{XMM1} \dots | \mathsf{XMM7} & \text{Floating-point registers} \\
r & \in & \textsf{preg} & \text{Registers} \\
 & ::= & \mathit{ri} \, | \mathit{rf} \\
& | & \mathsf{EIP} \, | \mathsf{RA} & \text{Prog counter, return addr} \\
\end{array}
\]
\[
\begin{array}{llll}
\mathit{ti} & ::= & \mathit{ri} & \text{Direct integer register access} \\
& | & (\mathit{ri})n & \text{Indirect load/store via int register} \\
& | & x+n & \text{Load/store to global variable} \\
\mathit{si} & ::= & \mathit{ti} & \text{Dereference} \\
& | & \$n & \text{Constant integer} \\
& | & \$x+n & \text{Pointer to global variable} \\
\\
\mathit{tf} & ::= & \mathit{rf} \, | (\mathit{ri})n \, | x+n & \text{Floating-point targets} \\
\mathit{sf} & ::= & \mathit{tf} \, | \$q \, & \text{Floating-point sources} \\
\\
\mathit{I} & ::= & \mathtt{movi} ~ \mathit{si}, \mathit{ti} & \text{Integer move/load/store} \\
& | & \mathtt{leai} ~ \mathit{si}, \mathit{ti} & \text{Integer load/store from address} \\
& | & \mathtt{movf} ~ \mathit{sf}, \mathit{tf} & \text{Floating-point move/load/store} \\
& | & \mathtt{addi} ~ \mathit{ri}_d, \mathit{ri}_o & \text{Arithmetic} \\
& | & \mathtt{subi} ~ \mathit{ri}_d, \mathit{ri}_o \, | \dots \\
& | & \mathtt{call} ~ \mathit{ti} \, | \mathtt{ret} & \text{Function call and return} \\
& | & \mathtt{push} ~ \mathit{ri} \, | \mathtt{pop} ~ \mathit{ri} & \text{Regular register push/pop} \\
& | & \mathsf{Pallocframe} ~ n & \text{Allocate stack frame} \\
& | & \mathsf{Pfreeframe} ~ n & \text{Free stack frame}
\end{array}
\]
\caption{The syntax of LAsm}
\label{fig:seq:lasm:syntax}
\hrulefill
\end{figure}

Similarly to ClightX, we extend the machine state with an abstract
state, which will be modified by primitives. This yields LAsm, whose
syntax (\cf Figure~\ref{fig:seq:lasm:syntax}) is the same as that of CompCert x86 assembly, except that the
semantics will be parameterized over the type of abstract states and
the specifications of primitives.  Most notably, primitive calls are
syntactically indistinguishable from normal function calls, yet depend
on the specifications semantically.

Moreover, in our Coq formalization, the semantics of LAsm is also
equipped with \emph{memory accessors} for address translation in order to handle both
kernel memory linear mapping and user space virtual memory.
In the latter case, plain integers can be treated
as pointers to user memory, as opposed to kernel memory modeled as the
CompCert-style concrete memory state. 

\begin{definition}[LAsm Modules]
A LAsm module is a finite map from identifiers to arrays of LAsm
instructions.
\end{definition}

\ignore{We define the semantics of LAsm in small-step form. The machine
state is $(\rho, m, a)$ where $\rho$ contains the values of registers,
$m$ is the concrete memory state and $a$ is the abstract state. Let
$M$ be an LAsm module, which is a finite map from identifiers to arrays of LAsm
instructions, we write $\Gamma, L, M \vdash (\rho, m, a)
\rightarrow (\rho', m', a')$ a transition step in the LAsm machine. The full syntax and formal semantics of LAsm is described in the
  companion technical report.}

\ignore{
\begin{figure}[t]\scriptsize
$$
\begin{array}{l|l}
\begin{array}{l}
\verb+kctxt_switch:+\\
\verb+	leai	0(%eax,%eax,2), %eax+\\
\verb+	leai	KCtxtP(,%eax,8), %eax+\\
\verb+	movi	%esp, 0(%eax)+\\
\verb+	movi	%edi, 4(%eax)+\\
\verb+	movi	%esi, 8(%eax)+\\
\verb+	movi	%ebx, 12(%eax)+\\
\verb+	movi	%ebp, 16(%eax)+\\
\verb+	pop	%ecx+\\
\verb+	movi	%ecx, 20(%eax)+\\
\verb+	leai	0(%edx,%edx,2), %edx+
\end{array}
&
\begin{array}{l}
\\
\verb+	leai	KCtxtP(,%edx,8), %edx+\\
\verb+	movi	0(%edx), %esp+\\
\verb+	movi	4(%edx), %edi+\\
\verb+	movi	8(%edx), %esi+\\
\verb+	movi	12(%edx), %ebx+\\
\verb+	movi	16(%edx), %ebp+\\
\verb+	movi	20(%edx), %ecx+\\
\verb+	push	%ecx+\\
\verb+	xor	%eax, %eax+\\
\verb+	ret+\\
\verb+......+
\end{array}
\end{array}
$$
\caption{Kernel context switch verified in LAsm}
\label{fig:contextswitch}
\end{figure}
}

%\vspace*{-3pt}
\paragraph{Assembly layer interfaces}
The semantics of LAsm is parameterized over a layer interface.
Different from C-style primitives (\cf Definition~\ref{def:c-prim}), which are defined using
argument list and return value, primitives implemented in LAsm
often utilize their full control over the
register set and are not restricted to a particular calling convention (\eg, context switch).
Therefore, it is necessary to extend the structure of layer
interfaces to allow assembly-style primitives modifying the register set.

\begin{definition}[Assembly-style primitive]
An assembly-style primitive specification $P$ over the abstract state type
$\abst$ is a predicate on $((\textsf{preg} \rightarrow \textsf{val}) \times \textsf{mem} \times \abst) \times
((\textsf{preg} \rightarrow \textsf{val}) \times \textsf{mem} \times \abst)$. $P(\rho, m, a,
\rho', m', a')$ says that the primitive $P$ takes
register set $\rho$, memory state $m$ and abstract state $a$ as
arguments, and returns register set $\rho'$, memory state
$m'$ and abstract state $a'$ as result.
\end{definition}

By ``\emph{style},'' we mean the calling convention,
not the language in which they are actually implemented.
C-style primitives may very well be implemented as hand-written
assembly code at underlay.

We can then define assembly layer interfaces by replacing the primitive
specification with our assembly-style one in Definition~\ref{def:c-layer}.
But, to make reasoning simpler, when defining assembly layer interfaces,
we distinguish C-style from assembly-style primitives.
First, C-style primitives
can be refined by other C-style primitives. 
Second and most importantly, it
becomes possible to instantiate the semantics of ClightX with an
\emph{assembly} layer interface by just considering C-style primitives and ignoring
assembly-style primitives (which might not follow the C calling convention).
In this way, ClightX code is only allowed to call C-style
primitives, whereas LAsm can actually call both kinds of primitives.

\begin{definition}[Assembly layer interface] \label{def:asm-layer}
An assembly layer interface $L$ is a tuple
$ L = (\abst, \primt_{\mathrm{ClightX}}, \primt_{\mathrm{LAsm}}, \invt, \mat) $
where:
\begin{itemize}
\item $(\abst, \primt_{\mathrm{ClightX}}, \invt)$ is a C layer interface (\cf
Definition~\ref{def:c-layer})
\item $\primt_{\mathrm{LAsm}}$ is a finite map from identifiers to assembly-style
primitive specifications over the abstract state $\abst$. The domains of
$\primt_{\mathrm{ClightX}}$ and $\primt_{\mathrm{LAsm}}$ shall be disjoint.
\item $\mat$ is the memory accessor used for address translation.
\end{itemize}
\end{definition}


\paragraph{Small-step semantics}

We write $\llbracket \mathit{ti} \rrbracket^\triangleleft(\Gamma, \rho)$
(\emph{resp}. $\llbracket \mathit{tf} \rrbracket^\triangleleft(\Gamma, \rho)$)
to denote the evaluation of integer (\emph{resp}. floating-point) registers
and global variables as targets for \texttt{mov} operations, where
$\Gamma$ is a mapping from global variables to memory block
identifiers, and $\rho$ is a mapping from registers to values.

\[
\begin{array}{lll}
\llbracket (\mathit{ri})n \rrbracket^\triangleleft(\Gamma, \rho) & = (n' + n) & \text{if} ~ \rho(\mathit{ri}) = n' \\
\llbracket (\mathit{ri})n \rrbracket^\triangleleft(\Gamma, \rho) & = (b, \mathit{ofs}+n) & \text{if} ~ \rho(\mathit{ri}) = (b, \mathit{ofs}) \\
\llbracket x + n  \rrbracket^\triangleleft(\Gamma, \rho) & = (b, n) & \text{if} ~ \Gamma(x) = b
\end{array}
\]

To evaluate indirect load through registers and global variables as
source operand for $\mathtt{mov}$ operations, we write $\llbracket s
\rrbracket^\triangleright(\Gamma, \rho, m)$ the evaluation of
\texttt{mov} operands.

\[
\begin{array}{ll}
\llbracket s \rrbracket^\triangleright(\Gamma, \rho, m)  = m(v) &
\text{if }  \llbracket s \rrbracket^\triangleleft(\Gamma, \rho) = v 
\\
\llbracket r \rrbracket^\triangleright(\Gamma, \rho, m)= \rho(r) &
\llbracket \$ n \rrbracket^\triangleright(\Gamma, \rho, m)  = n 
\\
\llbracket \$ q \rrbracket^\triangleright(\Gamma, \rho, m)  = q 
\\
\llbracket \$ x+n \rrbracket^\triangleright(\Gamma, \rho, m)  = (b, n)  &
\text{if }  \Gamma(x)  = b 
\end{array}
\]

For any register set $\rho$, we write 
\[\mathsf{nextEIP}(\rho) =
\rho[\mathsf{EIP} \leftarrow (b, \mathit{ofs}+1)] \text{ if }\rho(\mathsf{EIP})
  = (b, \mathit{ofs})\]%
The register $\mathsf{EIP}$ is assumed to contain a
  pointer $(b, \mathit{ofs})$ where $b$ is the memory block
  corresponding to the current function, and $\mathit{ofs}$ is the
  index of the current instruction within this function; in the exact
  same way as in CompCert x86 assembly, we assume that all
  instructions are of size 1.

Function call and return only modify the value of registers \textsf{EIP} and \textsf{RA}.

Similarly to CompCert x86 assembly, \textsf{Pallocframe} allocates a
new stack frame memory block, stores the return address and a back
link to the parent stack frame, and makes the stack pointer point to
the new stack frame. Conversely, \textsf{Pfreeframe} restores the
previous values of the stack pointer and return address, then frees
the current stack block.

We define the semantics of LAsm in small-step form. The machine
state is $(\rho, m, a)$ where $\rho$ contains the values of registers,
$m$ is the concrete memory state and $a$ is the abstract state.  Let
$M$ be a module, we write $\Gamma, L, M \vdash (\rho, m, a)
\rightarrow (\rho', m', a')$ a transition step in the LAsm machine.

\begin{figure}[t]
\begin{mathpar}
\inferrule{
\llbracket s \rrbracket ^ \triangleright(\Gamma, \rho, m)  = v 
\\
\rho'  = \mathsf{nextEIP}(\rho[r \leftarrow v])
}
{\llbracket \mathtt{mov}~ s, r \rrbracket(\Gamma, L; \rho, m, a)  = (\rho', m, a)}
\and
\inferrule{
\llbracket s \rrbracket ^ \triangleright(\Gamma, \rho, m) = v_s
\\
\llbracket t \rrbracket ^ \triangleleft(\Gamma, \rho) = v_t\\
m' = m[v_t \leftarrow v_s]\\
\rho_2  = \mathsf{nextEIP}(\rho_1)
}
{\llbracket \mathtt{mov}~ s, t \rrbracket(\Gamma, L; \rho, m, a) = (\rho_2, m', a') }
\and
\inferrule{
\rho(r_s)  = n 
\\
\rho(r_t) = (b, \mathit{ofs}) 
\\
\rho' = \mathsf{nextEIP}(\rho[r_t \leftarrow (b, \mathit{ofs}+n)])
}
{\llbracket \mathtt{addi}~ r_s, r_t \rrbracket(\Gamma, L; \rho, m, a) = (\rho', m, a) }
\and
\inferrule{
\rho(r_s) = (b, \mathit{ofs}_s) 
\\
\rho(r_t) = (b, \mathit{ofs}_t) 
\\
\rho' = \mathsf{nextEIP}(\rho[r_t \leftarrow (\mathit{ofs}_t - \mathit{ofs}_s)])
}
{\llbracket \mathtt{subi}~ r_s, r_t \rrbracket(\Gamma, L; \rho, m, a) = (\rho', m, a) }
\and
\inferrule{
\llbracket t \rrbracket^\triangleleft(\Gamma, \rho)  = v
\\
\rho' = \rho[\mathsf{EIP} \leftarrow v][\mathsf{RA} \leftarrow \rho(\mathsf{EIP})] 
}
{\llbracket \mathtt{call} ~ t \rrbracket(\Gamma, L; \rho, m, a) = (\rho', m, a)}
\and
\inferrule{
\rho' = \rho[\mathsf{EIP} \leftarrow \rho(\mathsf{RA})][\mathsf{RA} \leftarrow \bot]
}
{\llbracket \mathtt{ret} \rrbracket(\Gamma, L; \rho, m, a) = (\rho', m, a) }
\and
\inferrule{
m_1  = \mathsf{alloc}(n)(m) 
\\
m_2  = m_1[(\mathsf{next}(m), 0) \leftarrow \rho(\mathsf{RA})]
\\
m' = m_2[(\mathsf{next}(m), 4) \leftarrow \rho(\mathsf{ESP})]
\\
\rho' = \rho[\mathsf{ESP} \leftarrow (\mathsf{next}(m), 0)]
}
{\llbracket \mathsf{Pallocframe} ~ n \rrbracket(\Gamma, L; \rho, m, a) = (\rho', m', a) }
\and
\inferrule{
\rho(\mathsf{ESP})  = (b, 0)
\\
m(b, 0) = \mathit{ra}
\\
m(b, 4) = \mathit{esp}
\\
m' = \mathsf{free}(b, n)(m)
\\
\rho'  = \rho[\mathsf{ESP} \leftarrow \mathit{esp}][\mathsf{RA} \leftarrow \mathit{ra}] 
}
{\llbracket \mathsf{Pfreeframe} ~ n \rrbracket(\Gamma, L; \rho, m, a) = (\rho', m', a)}
\end{mathpar}
\caption{Execution step of LAsm instructions}
\label{fig:seq:lasm:sem}
\hrulefill
\end{figure}


For any instruction $I$, we write $\llbracket I \rrbracket (\Gamma, L;
\rho, m, a) = (\rho', m', a')$ for its execution step.
For the sake of presentation, in Figure~\ref{fig:seq:lasm:sem},
we only show a simplified version of
the execution step of LAsm instructions,
where memory accesses only use the CompCert-style kernel memory (\ie, the address translation is an identity function). 
\ronghui{FIx}

For an internal function, we look into the module for the
instruction to execute.
\[
\inferrule{
  \rho(\mathsf{EIP}) = (b, \mathit{ofs}) \\
  \Gamma(f) = b \\
  M(b)(\mathit{ofs}) = I \\
  \llbracket I \rrbracket (\Gamma, L; \rho, m, a) = \rho', m', a'
}{
  \Gamma, L, M \vdash (\rho, m, a) \rightarrow (\rho', m', a')
}
\]

For external function calls, there are two cases, one for the case of
assembly-style primitives, and the other for the case of the C-style primitives.

For assembly-style primitives, we directly take the step from the
specification of the primitive in $L$.
\[
\inferrule{
  \rho(\mathsf{EIP}) = (b, 0) \\
  \Gamma(f) = b \\
  \Gamma \vdash L.\primt_{\mathrm{LAsm}}(f)(\rho, m, a, \rho', m', a')
}{
  \Gamma, L, M \vdash (\rho, m, a) \rightarrow (\rho', m', a')
}
\]

For C-style primitives, we take the specification of the primitive
from the C-style primitive component of $L$, but we have to wrap it into the
calling convention. We write $\mathsf{arguments}(\mathit{args}, m, v)$
to denote the fact that the arguments are stored in memory $m$, at the
stack pointer $v$; for primitives returning integer or
pointer values, we store the value to the \texttt{EAX} register. We
write $\mathsf{eraseNonCalleeSave}(\rho)$ to set to $\bot$ the
contents of all non-callee-save registers in $\rho$.
\[
\inferrule{
  \rho(\mathsf{EIP}) = (b, 0) \\
  \Gamma(f) = b \\
 \Gamma \vdash L.\primt_{\mathrm{ClightX}}(f)(\mathit{args}, m, a, \mathit{res}, m', a') \\
  \rho(\mathsf{ESP}) \neq \bot \\
  \mathsf{arguments}(\mathit{args}, m, \rho(\mathsf{ESP})) \\
  \rho(\mathsf{RA}) \neq \bot \\
  \rho' = \mathsf{eraseNonCalleeSave}(\rho)[\mathsf{EAX} \leftarrow \mathit{res}][\mathsf{EIP} \leftarrow \rho(\mathsf{RA})][\mathsf{RA} \leftarrow \bot]
}{
  \Gamma, L, M \vdash (\rho, m, a) \rightarrow (\rho', m', a')
}
\]


\paragraph{Whole-machine semantics and contextual refinement}

Based on the relational transition system which we just defined for LAsm,
we can define the whole-machine semantics including not only the code
that we wrote by hand or that we compile, but also the \emph{context}
code that shall call our functions. To this end, it suffices to equip
the semantics with a notion of initial and final state, in a way
similar to the CompCert x86 whole-program assembly semantics.

In CompCert, the initial state consists of an empty register set
with only \textsf{EIP} (instruction pointer register) pointing to the \texttt{main} function of the
module, and the memory state is constructed by allocating a memory
block for each global variable of the program. We follow the same
approach for LAsm, except that we also need an initial abstract
state, provided by the layer interface, so
we need to extend its definition:

\begin{definition}[Whole-machine layer interface] \label{def:whole-machine-layer}
A whole-machine layer interface $L$ is a tuple
$L = (\abst, \primt_{\mathrm{ClightX}}, \primt_{\mathrm{LAsm}}, \invt, \mat, a_0)$
where:
\begin{itemize}
\item $ (\abst, \primt_{\mathrm{ClightX}}, \primt_{\mathrm{LAsm}}, \invt, \mat) $ is an assembly layer interface
\item $a_0 : A$ is the initial abstract state.
\end{itemize}
\end{definition}

\begin{definition}[Whole-machine initial state]
The whole-machine LAsm initial state for layer $L$ and module $M$ is
the LAsm state $(\rho_0, m_0, a_0)$ defined as follows:
\begin{itemize}
\item $\rho_0(r) = \left\{ \begin{array}{ll}
(\Gamma(\mathtt{main}), 0) & \text{if} ~ r = \mathsf{EIP} \\
%\ifTR{0 & \text{if} ~ r = \mathsf{RA} \\}{}
0 & \text{if} ~ r = \mathsf{RA} \\
\bot & \text{otherwise}
\end{array}
\right. $
\item $m_0$ is constructed from the global variables of $\Gamma, L, M$
\item $a_0$ is the whole-machine initial state specified in $L$
\end{itemize}
\end{definition}

\begin{definition}[Whole-machine final state]
A whole-machine LAsm state $(\rho, m, a)$ is final with return code
$n$ if, and only if, $\rho(\mathsf{EAX}) = n$ and $\rho(\mathsf{EIP}) =
0$.
\end{definition}

Notice that $\rho(\mathsf{EIP})$ contains the \emph{integer} 0,
which is also the initial return address and is not a valid pointer.
This ensures that executions do not
go beyond a final state, following the CompCert x86 whole-program
semantics: \texttt{main} has returned to its ``caller'', which does
not exist. Thus, the final state is uniquely determined (there can be
no other possible behavior once such a state is reached), so the
whole-machine semantics is deterministic once the primitives are.

\begin{definition}[Whole-machine behavior]
Let $\Gamma$ be a mapping of global variables to memory blocks. Then, we say that
\begin{itemize}
\item $\mathit{LAsm}(\Gamma, L, M)$ \emph{diverges} if there is an
  infinite execution sequence from the whole-machine initial state for $L$
\item $\mathit{LAsm}(\Gamma, L, M)$ \emph{terminates with return code} $n$ if there is a finite execution sequence from the whole-machine initial state for $L$ to a whole-machine final state with return code $n$
\item $\mathit{LAsm}(\Gamma, L, M)$ \emph{goes wrong} if there is a finite execution sequence from the whole-machine initial state for $L$ to a non-final state that can take no step.
\end{itemize}
$M$ is said to be \emph{unsafe} in layer interface $L$ if $\mathit{LAsm}(\Gamma, L, M)$ goes wrong.
\ignore{
\tahina {TODO: should we also have traces?}
}
\end{definition}

Then, we are interested in \emph{refinement} between whole machines:
\begin{definition}[Whole-machine refinement]
Let $L_{\text{high}}, L_{\text{low}}$ be two whole-machine assembly
layer interfaces, and $M_{\text{high}}, M_{\text{low}}$ be two LAsm
modules. Then, we say that $M_{\text{low}} @ L_{\text{low}}$ refines
$M_{\text{high}} @ L_{\text{high}}$, and write $M_{\text{low}} @
L_{\text{low}} \sqsubseteq M_{\text{high}} @ L_{\text{high}}$ if,
and only if, for any $\Gamma$ such that $
\mathrm{dom}(L_{\text{high}}) \cup \mathrm{dom}(M_{\text{high}}) \cup
\mathrm{dom}(L_{\text{low}}) \cup \mathrm{dom}(M_{\text{low}})
\subseteq \mathrm{dom}(\Gamma)$ and $\mathit{LAsm}(\Gamma, L_{\text{high}},
M_{\text{high}})$ does not go wrong, then
(1) $\mathit{LAsm}(\Gamma, L_{\text{low}}, M_{\text{low}})$ does not go wrong;
(2) if $\mathit{LAsm}(\Gamma, L_{\text{low}}, M_{\text{low}})$
  terminates with return code $n$, then so does $\mathit{LAsm}(\Gamma,
  L_{\text{high}}, M_{\text{high}})$;
(3) if $\mathit{LAsm}(\Gamma, L_{\text{low}}, M_{\text{low}})$
  diverges, then so does $\mathit{LAsm}(\Gamma, L_{\text{high}},
  M_{\text{high}})$.
\end{definition}

In our Coq implementation, we actually formalized the semantics of
LAsm with a richer notion of observable behaviors involving
CompCert-style events such as I/O. Thus, we define the
whole-machine behaviors and refinement using event traces 
{\em a la} CompCert \cite[3.5 sqq.]{Leroy-backend}: if the higher machine does
not go wrong, then every valid behavior of the lower machine is a
valid behavior of the higher.

Finally, we can define \emph{contextual refinement} between layer interfaces through a module $M$:

\begin{definition}[Contextual refinement]
We say a module $M$ implements an overlay $L_{\text{high}}$ on top of an
underlay $L_{\text{low}}$, and write $ L_{\text{low}} \vDash M :
L_{\text{high}} $ if, and only if, for any module (\emph{context}) $M'$ disjoint from $M, L_{\text{low}}, L_{\text{high}}$,
we have $(M \oplus M')@L_{\text{low}} \sqsubseteq M'@L_{\text{high}}$.
\end{definition}

\paragraph{Per-module semantics}
As for ClightX, we can also specify the semantics of an LAsm module as
a layer interface. However, a major difference between ClightX and
LAsm is that it is not possible to uniquely characterize the
``per-function final state'' at which function execution should
stop. Indeed, as in LAsm there is no control stack, when considering
the per-function semantics of a function $f$, it is not possible to
distinguish $f$ exiting and returning control to its caller, from a
callee $g$ returning to $f$. 
For some typical functions in an operating system implementation,
there is either no \verb+ret+ in the code body, or the actual 
\verb+ret+ statement may not return directly to their callers (\eg, context switch).


Thus, even though both the step relation of the LAsm 
semantics and the primitive
specifications (of a layer interface) are deterministic, 
the semantics of a function could still be non-deterministic.

\begin{definition}
Let $L$ be an
assembly layer interface, and $M$ be an LAsm module. The
module semantics $\llbracket M \rrbracket L$ % of $M$ in $L$ 
is then the
assembly layer interface $\llbracket M \rrbracket L = (L.\abst, \emptyset, \primt,\emptyset,\emptyset)$,
where the assembly-style primitive specification $\primt$ is defined for each $f
\in \mathrm{dom}(M)$ using the small-step semantics of LAsm as follows:
\[
\begin{array}{ll}
\multicolumn{2}{l}{\primt(f)(\rho, m, a, \rho', m', a')} \\
\Leftrightarrow & \Gamma(f) = b \land \rho(\mathsf{EIP}) = (b, 0)  \land \Gamma, L, M \vdash (\rho, m, a) \rightarrow^+ (\rho', m', a')
\end{array}
\]
\end{definition}%
\noindent{}Notice that we use ``$\rightarrow^+$"
to state the fact that the assembly function $f$ can return at any time.

\paragraph{Soundness of per-module refinement}

In this thesis, we aim at showing that the layer calculus given in
Section~{\ref{sec:layer}} is a powerful device to prove contextual
refinement: instead of proving the whole-machine contextual refinement directly, 
we only need to prove the \emph{downward simulation} relations about
individual modules, notated as $L_{\text{low}} \vdash_R M : L_{\text{high}}$, 
and apply the soundness theorem to get the contextual refinement properties
at the whole-machine level.

\begin{lemma}[Downward simulation diagram] \label{lem:forward}
Let $(L_{\text{low}}, M, L_{\text{high}})$ be a certified layer,
such that $L_{\text{low}} \vdash_R M : L_{\text{high}}$. Then,
for any module $M'$, we have the following downward simulation diagram:
\[
\xymatrix{
s_{\text{high}} \ar[rr]_{\Gamma, L_{\text{high}}, M'}^1 \ar@{-}[d]^R & & s'_{\text{high}} \ar@{.}[d]^R \\
s_{\text{low}} \ar@{.>}[rr]_{\Gamma, L_{\text{low}}, M \oplus M'}^{+} & & s'_{\text{low}}
}
\]
\end{lemma}

\begin{theorem}[Soundness]\label{thm:sound}
Let $(L_{\text{low}}, M, L_{\text{high}})$ be a certified layer. If the
primitive specifications of $L_{\text{low}}$ are deterministic and if
$L_{\text{low}} \vdash_R M : L_{\text{high}}$, then $L_{\text{low}}
\vDash M : L_{\text{high}}$.
\end{theorem}
\begin{proof}
Since the whole machine {\small $LAsm(\Gamma, L_{\text{low}}, M)$} is
deterministic, we can flip the downward simulation given by Lemma
\ref{lem:forward} to an upward one, hence the whole-machine refinement.
\end{proof}

Since the per-function semantics is non-deterministic due to its final
state not being uniquely defined, we can only flip the downward
simulation to contextual refinement at the whole-machine level.


\subsection{Layered programming and verification in LAsm}
For the functions which do not satisfy the C calling convention, we verify them in 
LAsm. Figure \ref{fig:contextswitch} shows the implementation of the context switch function
\verb"cswitch" in LAsm for switching two kernel thread stacks. This function violates
the C calling convention, which assumes the stack or continuation stays unchanged after
a function call.
Recall that the semantics of LAsm functions is non-deterministic, while the
corresponding  specifications are deterministic.
Hence, it is not possible to prove the contextual refinement relation 
directly at the function level. 
Therefore, we only prove the downward simulation relation for each LAsm function
and flip to the contextual refinement relation proof into an upward simulation
at the whole-machine level. 






