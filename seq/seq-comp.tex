\section{Certified compilation and linking}
\label{sec:seq:comp}

We would like to write most parts of our kernel in ClightX rather than
in LAsm for easier verification.  This means that, for
each layer interface $L$, we have to compile our ClightX($L$) source code to the
corresponding LAsm($L$) assembly language in such a way that all
proofs at the ClightX level are preserved at the LAsm
level.

This section describes how we have modified the CompCert compiler to
compile certified C layers into certified assembly layers. It also talks about
how we link compiled certified C layers with other certified assembly layers. 

\subsection{The CompCertX verified compiler}

To transport the proofs at ClightX down to LAsm, we adapt the CompCert
verified compiler to parameterize all
its intermediate languages over the layer interface $L$ similarly to
how we defined ClightX($L$), including the assembly language. This gives rise to
$\CompCertX{L}$ (for ``CompCert eXtended'', where external functions
are instantiated with layer interface $L$).

CompCertX goes from ClightX to the similarly parameterized AsmX and
then to LAsm. We retain all features and optimizations of CompCert,
including function inlining, dead code elimination, 
constant
propagation\footnote{With the exception of \texttt{const} global
  variable unfolding},
common subexpression elimination, and tail call
recognition.

% We skip execLoad, execStore for the moment

\paragraph{Kernel mode}

In addition to compiling $\ClightX{L}$ to the assembly
language,  
%(which we obtain by similarly parameterizing the CompCert
%assembly language), 
we have to ``compile'' --- in fact, reinterpret
--- the obtained assembly code to the LAsm($L$) assembly language
instantiated by the layer interface as well. However, remember that, contrary to
ClightX, the semantics of memory accesses in LAsm is entirely
parameterized by the layer interface $L$. Thus, it becomes necessary to prove
that, for every access (\ie, load and store)
to the memory state in ClightX, the
corresponding semantics of $L.\mat.\mathit{load}$ and
$L.\mat.\mathit{store}$ specified by the assembly layer  $L$
will actually operate on the memory state rather than on the abstract
state.

To this end, we require each layer interface to define a predicate
$\textsf{kernel\_mode}$ on the type $L.\abst$ of abstract state specified by
the layer  $L$, with the following properties:

\begin{itemize}
\item for code in kernel mode, load and store memory accesses are
  performed in the memory, and the abstract state is unchanged
  (and hence, stays in kernel mode):
\[
\begin{array}{ll}
 \multicolumn{2}{l}{
L.\textsf{kernel\_mode}(a)} \\
\Rightarrow & L.\mat.load(v; \rho, m, a)  = m(v) \\
& \land  \quad L.\mat.store(v_s, v; \rho, m, a) = (\rho, m[v_s \leftarrow v], a)
\end{array}
\]

\item every C-style primitive preserves kernel mode:
\[
\begin{array}{l}
 L.\primt_{\mathrm{ClightX}}(\mathit{args}, m, a, \mathit{res}, m', a')
\land {L.\textsf{kernel\_mode}(a)} \\
\Rightarrow  {L.\textsf{kernel\_mode}(a')}
\end{array}
\]
\end{itemize}%
These requirements illustrate the fact that address translation is an
identity function during kernel code execution.


\paragraph{Compiler correctness for CompCertX} 

Because CompCert only proves semantics preservation for whole
programs, the major challenge is to adapt the semantics preservation
statements of all compilation passes (from Clight to assembly) 
to per-function semantics.

The operational semantics of all CompCert languages are given through
small-step transition relations equipped with sets of whole-program
initial and final states, so we have to redesign those states to
per-function setting. For the initial state, whereas CompCert
constructs an initial memory and calls \verb+main+ with no arguments,
we take the function pointer to call, the initial memory, and the list
of arguments as parameters. For the final state, we take not
only the return value, but also the memory state when we exit the
function.

Consequently, the compiler correctness proofs have to
change. Currently, CompCert uses a downward simulation diagram
\cite[2.1]{Leroy-backend} for each pass from Clight, then, thanks to
the fact that the CompCert assembly language is deterministic (up to
input values given by the environment), CompCert composes all of them
together before turning them to a single upward simulation which
actually entails that the compiled code refines the source code.

In this work, we follow a similar approach: for each individual pass,
we prove per-function semantics preservation in a downward simulation
flavor.
We do not, however, turn it
into an upward simulation, because the whole
layer refinement proof is based on downward simulation,
which is in turn turned into an upward simulation at
whole-machine contextual refinement thanks to the determinism (up to
the environment) of LAsm$(L)$.


\paragraph{Memory state during compilation}

The main difference between CompCert and CompCertX lies in the memory
given at the beginning of a function call.

In the whole-program setting, the initial state is the same
across all languages, because it is uniquely determined by
the global variables (which are preserved by compilation).
On the other hand, in the middle of the execution
when entering an arbitrary function,
the memory in Clight is different from its assembly counterpart
because CompCert introduces \emph{memory transformations} such as
memory injections or extensions \cite[5.4]{leroy08}
to manage the callees' stack frames.
This is actually advantageous for compilation
of handling arguments and the return address.

For CompCertX, within the module being compiled, the same memory state
mismatch also exists.
At module entry, however, we cannot assume much about the
memory state because it is given as a parameter to
the semantics of each function in the module. In fact, this memory
state is determined by the caller, so it may very well come from
non-ClightX code (e.g., arbitrary assembly user code), thus we have to
take the same memory as initial state across all the languages of
CompCertX. It follows then that the arguments of the function already
have to be present in the memory, following the calling convention
imposed by the assembly language, even though ClightX does not read
the arguments from memory.

Another difference between CompCert and CompCertX is the treatment of
final memory states. In CompCert, only the return value
of a program is observable at the end; the final memory state
is not. By contrast, in CompCertX, the final memory state
is passed back to the caller hence observable. Thus, it is
necessary to account for memory transformations when relating
the final states in the simulation diagrams.

\paragraph{Compilation refinement relation}
Finally, the per-function compiler correctness statement of CompCertX
can be roughly summarized as this commutative diagram and formally defined below.
\[
\xymatrix@R=3pt@C+=3pt{
& v, m', a'
\ar@<-3ex>@{_{(}-->}[dd]^j
\ar@{_{(}-->}[dd]^j
\ar@<3ex>@{==}[dd]
\\
\genfrac{}{}{0pt}{0}{\mathit{args}}{\rho}, m, a
\ar[ur]^{L_\text{C}.\primt(f)}
\ar@{-->}[dr]^{L_\text{Asm}.\primt(f)} \\
\mathit{args} \approx m(\rho(\textsf{ESP}))
& \rho', m'', a'
}
\]

\begin{definition}
Let $L_\text{C}$ be a C layer interface, and $L_{\text{Asm}}$ be an
assembly layer interface. We say that $L_C$ is simulated by
$L_{\text{Asm}}$ by compilation, written $L_{\text{C}}
\leqslant^{\textrm{comp}} L_{\text{Asm}}$, if and only if, for any $\Gamma$, and for any execution $L_C.\primt(f)(\mathit{args}, m, a, v, m', a')$ of a primitive $f$ of $L_C$ for some list $\mathit{args}$ of arguments and some return value $v$, from  memory state $m$ and abstract state $a$ to $m'$ and $a'$, and for any register map $\rho$ such that the following requirements hold:
\begin{enumerate}
\item the memory $m$ contains the arguments $\mathit{args}$ in the stack pointed to by $\rho(\mathsf{ESP})$
\item $\mathsf{EIP}$ points to the function $f$ being called: $\rho(\mathsf{EIP}) = (\Gamma(f), 0)$
%% \item $kernelMode(a)$ holds
\end{enumerate}
Then, there is a primitive execution $L_{\text{Asm}}.\primt(f)(\rho, m, a,
\rho', m'', a')$ and a memory injection $j$ from $m'$ to $m''$
preserving the addresses of $m$ such that the following holds:
\begin{itemize} %\itemsep 0pt
\item the values of callee-save registers in $\rho$ are preserved in $\rho'$;
\item $\rho'(\mathsf{EIP})$ points to return address $\rho(\mathsf{RA})$;
\item the return value contained in $\rho'(\mathsf{EAX})$ (for integers/pointers) or $\rho'(\mathsf{FP0})$ (for floating-points) is related to $v$ by $j$;
\end{itemize}
\end{definition}

\begin{theorem}
\label{theorem:compcorect}
Let $L$ be an assembly layer interface with all C-style primitives
well-behaved. Then, for any $M$:
\[
\llbracket M \rrbracket L \leqslant^{\textrm{comp}} \llbracket \textrm{\emph{CompCertX}}(M) \rrbracket L
\]
\end{theorem}

%{We will write 
%$\llbracket M \rrbracket L \leqslant^{\textrm{comp}} \llbracket \mathrm{CompCertX}(M) \rrbracket L$
%for a relation of this shape.}



It follows that the way abstract state changes must remain intact during
compilation.
In other words, while a pointer to a stack-allocated
variable can be passed to a primitive, it cannot be ``stored'' into
the abstract state, as such pointer changes during compilation.

\paragraph{Well-behaved primitives}

To be able to use CompCertX, ClightX code can only call
\emph{well-behaved} external functions. The operational semantics of
ClightX described in Section~\ref{sec:seq:clight} already
guarantees that ClightX code will not call assembly-style primitives. Thus,
ClightX, similarly to CompCert Clight, does not support primitives
that do not follow a clear function call/return discipline, such as
context switching, or primitives that would switch from kernel mode to
user mode.

However, this restriction is not enough. Just like external functions
in CompCert, C layer primitives have to satisfy conditions for their
specifications to be stable by compilation by CompCertX. In
particular, they have to be stable by memory transformations
introduced by the compiler, such as memory injections or extensions.



\paragraph{Memory model and treatment of abstract state}

It is interesting that the correctness proofs  of CompCertX are agnostic to
the semantics of external functions (besides their compilability
requirements): in fact, they are not even aware of the presence of
abstract state along with physical memory, because abstract state can
be accessed only through primitives, and all calls to primitives are
exactly preserved ``as is'' by each pass. In terms of Coq
implementation, we did not even need to specify anything about the
abstract state at all in compiler correctness proofs. We proceed in four steps:

\begin{enumerate}
\item We first axiomatically specify the memory model through its
  operations (load, store, alloc, free) and requirements on their
  semantics.
\item Then, we parameterize the compiler correctness proofs of
  CompCertX over such memory model by means of type classes.
\item Independently of compiler correctness proofs, we provide a
  way to lift any memory model along a lens and show that, if 
  $\textsf{mem}'
  \stackrel{\pi}{\longrightarrow} \textsf{mem}$ is a lens, then, $\textsf{mem}'$ is a valid
  memory model whenever $\textsf{mem}$ is a valid memory model.
\item Finally, once the compiler correctness is obtained, we
  instantiate it for any abstract state type $\abst$ with the memory model $\textsf{mem} \times \abst$
  where $\textsf{mem}$ is the concrete memory model implementation of CompCert,
  thanks to the fact that $\textsf{mem} \times \abst
  \stackrel{\pi_1}{\longrightarrow} \textsf{mem}$ is a lens.
\end{enumerate}


\paragraph{Stack issues} 
We have to guarantee that the compiled code must not modify the stack
of its caller, including its arguments and its spilling locations. To
this end, we actually restrict the semantics of ClightX by not writing
to memory blocks that may be the stack locations of the caller.  This
restriction is necessary because of one pass in the compiler, namely the
stack layout pass, relies on the fact that the produced code must not
modify the locations corresponding to the function arguments, which
are actually located in the caller's frame. Whereas the compiler
can maintain this guarantee within a module, the arguments of the
function called at module entry are in the initial memory state, so
those locations have to be protected.\footnote{In whole-program
  CompCert, there are no such function arguments in the initial memory
  state, because \texttt{main} takes no arguments.}

We successfully investigated two ways to ensure this guarantee:
\begin{itemize}
\item In an early version, we introduced tags on memory blocks to
  distinguish stack from global variables, then we forbade ClightX
  from writing to memory blocks labeled as stack blocks. The major
  drawback of this method is that it very invasively modifies the
  CompCert memory model.
\item To avoid such deep changes, we instead disable writing on blocks
  that are valid at module entry but do not correspond to global
  variables. A further advantage of this method is that blocks that will be
  created by the module being compiled (\eg, local stack-allocated
  variables) may still be written to.
\end{itemize}


Then, the statement execution rules of ClightX become parameterized
over the per-module initial memory $m_0$ received at module entry. Thus,
the memory assignment rule becomes:
\[
\inferrule{
\llbracket{} e_1 \rrbracket^\triangleleft(\localvariable, \tau, m) = (b, \mathit{ofs}) \\
\llbracket{} e_2 \rrbracket^\triangleright(\localvariable, \tau, m) = v \\
b \not\in \mathrm{dom}(m_0) \backslash \mathrm{dom}(\Gamma) \\
m' = m[ (b, \mathit{ofs}) \leftarrow v ]
}{
  \Gamma, L, M, m_0, \localvariable \vdash e_1 = e_2 : (\tau, m, a) \downarrow
  (\cdot; \tau, m', a)
}
\]

and the semantics of a ClightX module $M$ becomes:
\[
\inferrule{
  f \in \mathrm{dom}(M) \\
  \Gamma, L, M, m \vdash f : (\mathit{args}; m, a) \Downarrow (\mathit{res}; m', a')
}{
  (\llbracket{}M\rrbracket{}L).\primt(f)(\mathit{args}, m, a, \mathit{res}, m', a')
}
\]

Similarly, we have to impose that memory reads in such blocks still be
the same before and after any call to a C-style primitive defined in $L$:
their semantics also becomes parameterized over $m_0$ in a similar way, with an analogous requirement:
\[
\forall b \in \mathrm{dom}(m_0) \backslash \mathrm{dom}(\Gamma), m'(b) = m(b)
\]


\subsection{Linking compiled code with assembly code} \label{sec:linking}

Contrary to traditional separate compilation, we target compiling
ClightX functions that may be called by LAsm assembly code.
Since the caller may be arbitrary LAsm code, not necessarily well-behaved
code written in or compiled from ClightX,
we have to assume that the memory we are given follows LAsm layout.
When reasoning about memory states that involve compiled code,
we then have to accommodate memory injections introduced by the compiler.

During a whole-machine refinement proof, the two memory states of the
overlay and the underlay are related with a simulation relation $R$.
However, consider when the higher (LAsm) code calls an overlay primitive,
that, in the underlay, is compiled from ClightX.
Because during the
per-primitive simulation proofs we ignored the effects of the compiler,
the memory injection introduced by the compiler may become a source of
discrepancy.  That is why we encapsulate, in $R$,
a memory injection between the higher memory state and the
lower memory state. This injection is identity until the lower state
calls a compiled ClightX function. Then, at every such call, the layer
simulation relation $R$ can ``absorb'' compilation refinement on its
right-hand side:

\begin{lemma} \label{lem:sim-absorb-comp}
If $L'$ and $L_\text{C}$ are C overlays and $L_{\text{Asm}}$ is
an assembly underlay, with $L' \leqslant_R L_C$ and $L_\text{C} \leqslant^{\textrm{comp}} L_{\text{Asm}}$, then $L' \leqslant_R L_{\text{Asm}}$.
\end{lemma}
\begin{proof} 
If $R$ encapsulates a memory injection $j_0$, and compilation introduces a
memory injection $j$, then, the simulation relation $R$ will still hold with
the composed memory injection $j \circ j_0$.
\end{proof}

\paragraph{Summary of the refinement proof with compilation and linking}

\begin{figure}[t]
\[
\xymatrix@R=12pt@C=3pt{
%
L_{2, \text{C}} \ar[d]^{\leqslant_R}_{\text{\tiny 2.}} &
\bigoplus^{\text{\tiny 1.}} &
L_{2, \text{Asm}} \ar[d]^{\leqslant_R}_{\text{\tiny 2.}} &
= L_2 \\
%
L'_{1, \text{C}} \ar[d]^{\leqslant_\id}_{\text{\tiny 3.}} & &
L'_{1, \text{Asm}} \ar[dd]^{\leqslant_\id}_{\text{\tiny 3.}} \\
%
\llbracket M_\text{C} \rrbracket L_1
\ar[d]^{\leqslant^{\textrm{comp}}}_{\text{\tiny 4.}} \\
%
\llbracket \textrm{CompCertX}(M_\text{C}) \rrbracket L_1 &
\bigoplus \ar[d]^{\leqslant_\id}_{\text{\tiny 6.}} &
\llbracket M_\text{Asm} \rrbracket L_1 \\
%
& \save[]+<-37.90pt,0pt>*\txt{
$\llbracket M \rrbracket L_1 =
 \llbracket \textrm{CompCertX}(M_\text{C}) \oplus M_\text{Asm} \rrbracket L_1$
}
\ar@{<-} `_r[rr] [uuuurr]_{\leqslant_R}^{\txt{\tiny 5.\\ \tiny 7.}}
\restore
& &
}
\]
\caption{Proof steps of layer refinement $L_1 \vdash_R M : L_2$}
\label{fig:layer-refinement-proof}
\hrulefill
\end{figure}


Finally, the outline of proving layer refinement $L_1 \vdash M : L_2$,
where $M = \textrm{CompCertX}(M_\text{C}) \oplus M_\text{Asm}$ is the union
of a compiled ClightX module and an LAsm module,
is summarized in the following steps, also shown in
Figure~\ref{fig:layer-refinement-proof}:
\begin{enumerate}   
%\begin{compactenum}
\item Split the overlay $L_2$ into two layer interfaces
  $L_{2, \text{C}}$ and $L_{2, \text{Asm}}$ where $L_{2, \text{C}}$ is a C layer interface
  containing primitive specifications to be implemented by
  ClightX code (necessarily C-style) and $L_{2, \text{Asm}}$ is an assembly layer interface
  containing all other primitives (implemented in LAsm), so that $L_2 = L_{2, \text{C}} \oplus
  L_{2, \text{Asm}}$. (In particular, $L_{2, \text{Asm}}$ will contain the
  memory accessors and the whole-machine initial state specification).
\item For each such part of the overlay, design an intermediate layer interface
  $L'_{1, \text{C}}$ and $L'_{1, \text{Asm}}$ with the same abstract state type as
  $L_1$ (\cf Section \ref{sec:clightx-prog}), and
  prove $L_{2, \text{C}} \leqslant_R L'_{1, \text{C}}$ and $L_{2, \text{Asm}} \leqslant_R
  L'_{1, \text{Asm}}$ independently of the implementation. In
  particular, the proof of $L_{2, \text{Asm}} \leqslant_R
  L'_{1, \text{Asm}}$ will also involve the refinement proofs for memory
  accessors and whole-machine initial states.
\item For both intermediate layer interfaces, prove that they are implemented by
  modules $M_\text{C}$ and $M_\text{Asm}$ on top of
  $L_1$ respectively, i.e.  $L'_{1, \text{C}} \leqslant_\id \llbracket M_\text{C} \rrbracket L_1$ and
  $L'_{1,\text{Asm}} \leqslant_\id \llbracket M_{\text{Asm}} \rrbracket L_1$.
\item Then, compile $M_\text{C}$:
$\llbracket M_\text{C}
  \rrbracket L_1 \leqslant^{\textrm{comp}} \llbracket
  \textrm{CompCertX}(M_\text{C}) \rrbracket L_1$~~~via Theorem~\ref{theorem:compcorect}.

\item
Using \rulename{LLe-Trans} and \rulename{LLe-Mon} to combine 2.\ and 3., we have:
$
L_{2, \text{C}} \oplus L_{2, \text{Asm}}
\leqslant_R
  L'_{1, \text{C}} \oplus L'_{1, \text{Asm}}
\leqslant_\id
  \llbracket M_\text{C} \rrbracket L_1 \oplus
  \llbracket M_{\text{Asm}} \rrbracket L_1
$.

On the C side (left of $\oplus$), Lemma~\ref{lem:sim-absorb-comp} shows that
$\leqslant_R$ absorbs $\leqslant^{\textrm{comp}}$.  By 4.:
$
L_{2, \text{C}} \oplus L_{2, \text{Asm}} \leqslant_R
  \llbracket \textrm{CompCertX}(M_\text{C}) \rrbracket L_1 \oplus 
  \llbracket M_{\text{Asm}} \rrbracket L_1
$.

\item From the soundness of 
\rulename{Hcomp}, 
and the fact that  $M = \textrm{CompCertX}(M_\text{C}) \oplus M_\text{Asm}$,
we have:
$\llbracket \textrm{CompCertX}(M_\text{C}) \rrbracket L_1 \oplus
\llbracket M_\text{Asm} \rrbracket L_1 \leqslant_\id \llbracket
M \rrbracket
L_1$.
\item
Finally, by combining 5. and 6., we have 
$L_{2, \text{C}} \oplus L_{2, \text{Asm}}
\leqslant_R  \llbracket M \rrbracket L_1$.  
Since~ $L_2 = L_{2, \text{C}} \oplus L_{2, \text{Asm}}$, 
by using \rulename{LLe-Ub-Left}
and \rulename{LLe-Comm},
we have: 
$L_2 \leqslant_R  \llbracket M \rrbracket L_1
\leqslant_\id \llbracket M \rrbracket L_1 \oplus L_1
\leqslant_\id L_1 \oplus \llbracket M \rrbracket L_1$,
thus we get ~$L_1 \vdash_R M : L_2$.
%%
%\[\begin{array}{r@{\;}c@{\;}l@{}r}
%L_{2, \text{C}} \oplus L_{2, \text{Asm}}
%&\leqslant_R&
%  \llbracket M \rrbracket L_1
%  & \text{\small{}5.\ \& 6.} \\
%&\leqslant_\id& \llbracket M \rrbracket L_1 \oplus L_1
%  & \text{\small\rulename{LLe-Ub-Left}} \\
%&\leqslant_\id& L_1 \oplus \llbracket M \rrbracket L_1
%  & \text{\small\rulename{LLe-Comm}}
%\end{array} \]
%\end{compactenum}
\end{enumerate} 



\subsection{Linking across layers before compiling}

Instead of compiling each module within its own layer interface, our layer
calculus also provides ways to first link code at the C level across
layers before compiling. Let $L_3$ be an overlay, $L_1$ be an
underlay, and $L_2$ be an intermediate layer interface. Let us assume that we
split each layer interface $L_i$ into two layer interfaces $L_i^C \oplus
L_i^{\mathrm{Asm}}$ and that we proved $L_{i}^{\mathit{lang}} \vdash_{R_i}
M_i^{\mathit{lang}}: L_{i+1}^{\mathit{lang}}$ for each $i \in \{1,2
\}$ and $\mathit{lang} \in \{ C, \mathrm{Asm} \}$.

Then, on each side, we can do vertical composition and obtain
$L_1^{\mathit{lang}} \vdash_{R_2 \circ R_1} M_1^{\mathit{lang}} \oplus
M_2^{\mathit{lang}} : L_3$. We can then follow the process described
in Section~\ref{sec:linking} with $H = L_3$, $L = L_1$ and
$M^{\mathit{lang}} = M_1^{\mathit{lang}} \oplus M_2^{\mathit{lang}}$, so that we obtain:
\[
L_1 \vdash_{R_2 \circ R_1} \mathrm{CompCertX}(M_1^C \oplus M_2^C) \oplus M_1^{\mathrm{Asm}} \oplus M_2^{\mathrm{Asm}}: L_3
\]
This shows that compiling before linking is possible as soon as
ClightX supports vertical composition with refinement. The
corresponding Coq proof is underway. Thus, because CompCertX already
supports function inlining (within a single module) just like CompCert
does, if an intermediate layer interface defining ``getter-setters'' is to be
implemented through simple functions performing only a single memory
access, then compilation before linking will allow suppressing the
function call overhead for these simple functions, as they are linked
within a single module before compilation, so that memory accesses
will be inlined in the overlay implementation.

